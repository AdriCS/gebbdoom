\section{Collision Detection} \label{blockmapdetails}
Collision detection is a significant part of the engine's activity. Each moving object (player, monster or projectile) must check for collisions before it moves. Line of sight also depends on an efficient collision system. Enemies which inflict direct hit damage also need to check if they have a clear line of fire.\\
\par
Collisions could have been detected via the BSP. However it was only after Bruce Naylor visited id Software that John Carmack became aware this was possible. By then, \doom{} had already shipped with its collision detection data structure called the blockmap.\\
\par
\vspace{10pt}
\drawing{E1M1_lines}{E1M1 sectors and lines}
\par
There is one blockmap per map which was generated via \cw{doombsp} preprocessing on NextStations. Saved in a lump audaciously named \cw{BLOCKMAP}, it is used at runtime to lower the number of lines to test intersections with.\\
\par
The work done by \cw{doombsp} is simple: divide the map into 128x128 axis-aligned blocks. For each block a list is made of every line that passes through it. At the end of the process, an index is constructed based on blockmap coordinates (in 128x128 units) pointing to the list of lines. Notice that a line can be present in multiple blocks. In the case of map E1M1, the result is visible in figure \ref{E1M1_blockmap}.

\drawing{E1M1_blockmap}{E1M1 lines indexed via blockmap. Note that empty blocks are not drawn}

All map traversals are done with an abstract method \cw{P\_PathTraverse} which takes as argument two coordinates making up a line to check collisions with and a function pointer to call when a hit is detected (a.k.a How to fake OOP with C).\\
\par
\ccode{P_PathTraverse.c}
\par
Function \cw{P\_AimLineAttack} which is used for punching and sawing uses \cw{P\_PathTraverse} with flag = \cw{PT\_ADDLINES|PT\_ADDTHINGS} so that only lines and things are considered during traversal. The block coordinates of any map coordinate are easy to obtain via modulo 128. To detect collisions with things, their block coordinates are updated each time they change position.\\
\par
\ccode{P_AimLineAttack.c}
