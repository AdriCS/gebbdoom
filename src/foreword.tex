In many ways, \doom{} was almost a "perfect" game.\\
\par
With hindsight and two decades more skill building, I can think of better ways to implement almost everything, but even if I could time machine back and make all the changes, it wouldn't have really mattered.  \doom{} hit a saturation level of success, and the legacy wouldn't be any different if it was 25\% faster and had a few more features.\\
\par
 The giant aliased pixels make it hard to look at from a modern perspective, but \doom{} felt "solid" in a way that few 3D games of the time did, largely due to perspective correct, subpixel accurate texture mapping, and a generally high level of robustness.\\
\par
Moving to a fully textured and lit world with arbitrary 2D geometry let designers do meaningful things with the levels.  Wolfenstein 3D could still be thought of as a "maze game", but \doom{} had architecture, and there were hints of grandeur in some of the compositions.\\
\par
 Sound effects were actually processed, with attenuation and spatialization, instead of simply being played back, and many of them were iconic enough that people still recognize them decades later.\\
\par
 The engine was built for user modification from the ground up, and the synergy of shareware distribution, public tool source release, and early online communities led to the original game being only a tiny fraction of the content created for it.  Many careers in the gaming industry started with someone hacking on \doom{}.\\
\par
 Blasting through the game in cooperative mode with a friend was a lot of fun, but competitive FPS deathmatch is one of the greatest legacies of the game.  Seeing another player running across your screen, converging with the path of the rocket that you just launched, is something that still makes millions of gamers grin today.\\
\par
 There was a lot of clever smoke and mirrors involved in making \doom{} look and feel as good as it did, and it is a testament to the quality of the decisions that so many people thought it was doing more than it actually was.  This remains the key lesson that still matters today: there are often tradeoffs that can be made that gets you a significant advantage in exchange for limitations that you can successfully cover up with good design.\\
\par
-- John Carmack\\
\par
