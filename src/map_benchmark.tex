 As of 2018, it has been twenty five years since the last machine came out of \NeXT{}'s Redwood City factory in 1993. It has become a rare occurrence to find one of these pieces of black hardware in working condition.\\
 \par
 Since this book strives to be historically accurate, it was paramount to find an actual NeXTStation Turbo -- first and foremost to document the development condition of the time, but also to witness the full game pipeline in motion. Even though passionate and dedicated programmers have produced a gorgeous emulator called "Previous", the performance numbers would not have been accurate.\\
 \par
  I lucked out on eBay and found exactly the configuration I needed. The machine was in working condition but the SCSI hard-drive was making clinking noises, a sign that it was about to die. Additionally, the MegaDisplay colors had faded out\footnote{This would have been easily fixable by replacing the capacitors on the monitor control board, but it would not have solved the issue of the weight.} and its 50 lbs (23 kg) made it difficult to move.\\
  \par
  Thanks to Rob Blessin, owner and founder of Black Hole Inc., I was able to replace the HDD with a SD card SCSI2SD providing similar access time. It was difficult to find a screen compatible with \NeXTns{}'s exotic "sync on green" but thanks to the wonderful people at \cw{www.nextcomputers.org} I was pointed to a NEC MultiSync 1980SX which worked flawlessly.\\
  \par
  Words cannot convey how it felt to hear the humming of the machine's fan. To witness \cw{Doom.app}, \cw{DoomED} and, \cw{doombsp} compile flawlessly. To witness this NeXTStation Turbo Color (serial \cw{\#ABC0053943}) come back to life. The machine did not cure cancer as Jobs wished for but it did provide happiness to countless developers.





\section{Developing The Game}
On this double page is recreated the typical developer desktop setup. Notice "Interceptor VGA Console" which gives away \cw{libinterceptor.a}, a private library provided by \NeXTns's engineers to punch a hole in Display Postscript and bypass the "slow" compositor.\\
\par
\cscaledimage{0.9}{doom_on_next.png}{NeXTSTEP development setup (left part of the screen)}

The MegaDisplay resolution of 1120x832 was so high that id Software had to implement a 2x software zoom for the game window. Without it, the \doom{} window looked like a tiny stamp with barely any pixels visible.\\
\par
\vspace{18.5pt}
\cscaledimage{0.92}{doom_on_next2.png}{NeXTSTEP development setup (right part of the screen)}
\pagebreak

\section{Compiling Maps}
Benchmarks for \cw{doombsp} run time for each level in \doom~and \doomii\footnote{Based on \cw{.map} files released by John Romero on 2015-04-22.}.\\
\par
 \begin{minipage}[t]{0.45\textwidth}
 \begin{figure}[H]
\centering  
\begin{tabularx}{\textwidth}{ L{0.3} | R{0.7} }
  \specialrule{1pt}{0pt}{0pt}
  \textbf{Map} & \textbf{\cw{doombsp} runtime (s)} \\
  \specialrule{1pt}{0pt}{0pt}
E1M1 &     8.2 \\ 
E1M2 &       32.0 \\
E1M3 &       26.2\\
E1M4 &       18.4\\  
E1M5 &       19.9\\
E1M6 &       44.0\\
E1M7 &       22.3\\
E1M8 &        6.9\\
E1M9 &       15.4\\
E2M1 &        6.0\\
E2M2 &        55.4\\
E2M3 &        19.6\\
E2M4 &        36.0\\
E2M5 &        46.8\\
E2M6 &        32.5\\
E2M7 &        60.8\\
E2M8 &         2.5\\
E2M9 &         1.5\\
E3M1 &        2.5\\
E3M2 &        9.2\\
E3M3 &       38.1\\
E3M4 &       23.7\\
E3M5 &       34.5\\
E3M6 &       22.5\\
E3M7 &       23.4\\
E3M8 &        1.9\\
E3M9 &        8.9\\
   \specialrule{1pt}{0pt}{0pt}
\end{tabularx}
%\caption{Video system interface}
\end{figure}
\end{minipage}
\hspace{1cm}
\begin{minipage}[t]{0.45\textwidth}
 \begin{figure}[H]
\centering  
\begin{tabularx}{\textwidth}{ L{0.3} | R{0.7} }
  \specialrule{1pt}{0pt}{0pt}
  \textbf{Map} & \textbf{\cw{doombsp} runtime (s)} \\
  \specialrule{1pt}{0pt}{0pt}
MAP01 &       6.1  \\
MAP02 &       6.6 \\
MAP03 &       8.7 \\
MAP04 &       8.5 \\
MAP05 &       17.6\\
MAP06 &       25.0\\
MAP07 &       1.9 \\
MAP08 &       15.2\\
MAP09 &       16.3\\
MAP10 &       34.0\\
MAP11 &        15.7 \\
MAP12 &        15.2\\
MAP13 &        31.5\\
MAP14 &        44.7\\
MAP15 &        66.0\\
MAP16 &        16.2\\
MAP17 &        36.2\\
MAP18 &        17.2\\
MAP19 &        45.8\\
MAP20 &        29.2\\
MAP21 &        5.7 \\
MAP22 &        9.4 \\
MAP23 &        7.5 \\
MAP24 &       30.5 \\
MAP25 &       21.1 \\
MAP26 &       18.8 \\
MAP27 &       26.2 \\
MAP28 &       19.6 \\
MAP29 &       45.8 \\
MAP30 &        1.0 \\
MAP31 &       16.4 \\
MAP32 &        2.7 \\
MAP33 &        6.6 \\
MAP34 &        9.3 \\
MAP35 &        0.3 \\
   \specialrule{1pt}{0pt}{0pt}
\end{tabularx}
%\caption{Video system interface}
\end{figure}

\end{minipage}
\\

\section{Running The Game}
\fixme{TODO TABLE HIGH LOW details framerate here.}\\
\par

On \NeXTns{}, the video system is very different to DOS. The implementation disregards update signals from \cw{I\_} and defers all work to \cw{I\_FinishUpdate} where the full content of framebuffer \#0 is blitted to \cw{NSWindow}. As a result it does not benefit as much as it should from "reduced canvas size".\\
\par

\section{Framebuffer Non-distortion}
Contrary to what happens on DOS due to the VGA, there is no distortion when the framebuffer is displayed onto the screen. The 320x200 content is not stretched to 320x240 and therefore appears vertically squashed. It is particularly noticeable when the splash screen is displayed.\\
\par
\cfullimage{doom_crushed}{\doom{} on NeXTSTEP. Content appears vertically squashed}

