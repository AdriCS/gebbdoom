\vspace{-10pt}
\section{Programming}
Migrating from the Borland C++ editor on DOS to TextEdit on NeXTSTEP was a trade-off. On the one hand, convenient features such as syntax highlighting were lost. On the other hand, apps never crashed. Precious hours of work were never lost. TextEdit also had markers (\cw{//}) allowing to "fold" code to improve navigation and readability.\\
\par
 The higher resolution (1120 x 832) of the MegaDisplay allowed seeing much more code vertically and up to three DOS 80 column windows side by side. Notice how Borland's IDE (in its default mode\footnote{Borland's C++ IDE could be set to use 80 column mode to get 50 lines but readability suffered greatly.}) shows 21 lines of code while TextEdit can show 57 lines.\\
\par
\fullimage{development.png}\\

\vspace{-4mm}
\cfullimage{TextEditApp.png}{TextEdit.app allowed section folding via special markers}


In figure \ref{TextEditApp.png}, TextEdit tags and folding show close to 700 lines of \cw{d\_main.c}'s' 1181 lines. This feature allowed developers to think in terms of systems instead of functions.

\subsection{Interface Builder, OOP and Objective-C}
The list of tools would not be complete without mentioning \NeXTns{}'s crown jewel which many considered to be NeXTSTEP's killer app, Interface Builder.\\
\par
"IB" was first written in Lisp by Jean-Marie Hullot in 1984 and commercialized in 1986 under the name "SOS Interface"\footnote{Source: "A Brief History of Human Computer Interaction Technology".}. Hullot was hired by NeXT, Inc. where along with a team he created a similar tool focused around Objective-C.\\
\par
The NeXTSTEP version he managed to produce would reduce the construction cost of building GUIs by a factor of 5-10x\footnote{Source: "NeXT vs Sun: A world of a difference", 1991 promotional video.}.\\
\par
\fullimage{ib.png}



\vspace{-4mm}
With IB, all of the drudge involved in GUI authoring was done with a mouse in the blink of an eye. The "tedious" part involving writing code was only mandatory for the actual meat of the application in the business logic layer. Creating the GUI was a two-step process. First draw all the elements, then connect the UI elements to the Object models.\\
\par
The first part was, as Steve Jobs qualified it, "insanely easy" since building an interface was done with a series of drag-and-drop operations from a palette of GUI elements onto a canvas. An inspector allowed seeing the properties of an element. Everything from min/max of a slider to the default value of a text field could be adjusted easily.\\
\par
The second part, connecting the visual elements to the business logic objects, was also done with the mouse, by connecting visual boxes to targets/actions. The target could be a boolean property in the case of a checkbox UI element or it could be a method in the case of a button UI element.
%\par
%The developer could focus its time on implementing only the business logic.\\

\subsubsection{Object Oriented Programming}
Beyond its revolutionary design, IB was nicely complemented by the OOP (Oriented Object Programming) design of a programmer-friendly language called Objective-C.\\
\par
\fq{In my 20 years in this industry, I have never seen a revolution as profound as [object-oriented-programming]. You can build software literally 5 to 10 times faster, and that software is much more reliable, much easier to maintain and much more powerful... All software will be written using this object technology someday. No question about it.}{Steve Jobs, Rolling Stone, June 16, 1994.}\\
\par
OOP's encapsulation, inheritance and polymorphism allowed pushing back the limits of complexity a human programmer could deal with. A program would be conceptualized as a collection of potentially nested sub-systems. The mental image did not have to be a complex monolithic block. It could be decomposed in smaller, easier to summarize opaque systems.\\

\par
\vspace{-10pt}
\subsubsection{Objective-C}
Objective-C was developed around the same time as \verb!C++!. However, as co-creator Brad Cox recalls, his creation and Bjarne Stroustrup's brainchild had opposing philosophies. Where C++ placed performance first, Objective-C valued the programmer's productivity first.\\
\par
\fq{
	Back in 1980 when both our languages were under construction, I came down and met with Bjarne Stroustrup. We had radically different views on how our languages would be designed. The key concept was the relative importance of machine efficiency vs programmer efficiency. Eventually, we agreed to disagree.
} {Brad Cox (Interview for "The NeXT Bible" book)}\\
\par


\par
\vspace{-5pt}
The version shipping with NeXT computers featured an important library called Foundation Kit. One of its components, \cw{NSObject}, freed developers from the error-prone memory management burden by offering reference counting via its \cw{retain} and \cw{release} methods. A collection of swiss army knife containers -- \cw{NSArray}, \cw{NSDictionary}, \cw{NSSet} and \cw{NSData} -- further allowed focusing on the core functionality instead of losing time on infrastructure.\\
\par
\trivia{Initially using the prefix NX, all objects in Foundation were renamed with a leading NS for OpenSTEP where "NS" stands for the alliance between NeXT and Sun Microsystems. All these objects are still at the core of macOS and iOS today; the prefix NS was never removed.}\\
\par
The very core of Objective-C architecture, based on messages routed via a dispatching method \cw{objc\_msgSend}, allowed programs to be more resilient to errors. By far the most amazing feat (at least to a C++ developer) was the ability to send a message to \cw{nullptr}. When a developer sent a message to an object via the following syntax.\\ \par
\objccode{obj.message}
What really happened behind the scenes was a call to the dispatcher.\\
\par
\objccode{objc_msgSend.m}
The highly-optimized\footnote{Source: "Dissecting objc\_msgSend on ARM64" by Mikea Ash.}, hand written assembly was called millions of times by the time a NeXT had booted. Despite the complex mutable nature of ObjC objects (a method can be added at runtime, changing its duck type) \cw{objc\_msgSend} was able to follow an inheritance chain at runtime to find the proper target. More importantly, it was also able to detect a \cw{nullptr} and simply return the value \cw{0} instead of bringing down the entire process.\\
\par

Altogether, the four pillars of NeXT development (Unix, IB, OOP, and Obj-C) sped up development of DoomED, doombsp, and wadlink to a level that DOS-based tools could not even compare\footnote{Many years later, while evangelizing static code analysis, John Carmack would rant a bit about how the sloppy programming permitted by dynamic languages like ObjC was Not A Good Thing.}.

