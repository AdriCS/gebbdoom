\begin{wrapfigure}[4]{r}{0.4\textwidth}{
\centering \scaledimage{0.4}{sega_x32_logo.png}}
\end{wrapfigure}
In January 1994, Sega was in a delicate position. The Genesis, its 16-bit moneymaker was loosing ground in Japan. It  1993 sales had placed the machine third, behind Nintendo's Super Famicom and NEC's PC Engine. To make things worse SEGA now had to deal with two new competitor whom had entered the game in 1993 with Atari's' Jaguar and Panasonic's 3DO. The consensus at Sega Of Japan (SOJ) was that the company should put all of its available resources into the 32-bit Saturn project.\\
\par
While SOJ work on the Saturn was moving forward, it was feared it would be a while before it was finished. In the US, the Genesis had been selling well (32 millions units as of late 1993) and Sega of America (SOA) was eager to point out the financial opportunity to create a Genesis "booster".\\
\par
During CES '94 Las Vegas, then SOJ CEO Hayao Nakayama summoned Sega of America (SOA) executives, Joe Miller -- Head of R\&D, Marty Franz -- SOA Technical Director, and Scot Bayless -- Senior Producer into a conference call\footnote{Source: Retrogamer \#77. All quotes in this section are also from Retrogamer interviews.}.\\
\par 
They were give the green light for project "Mars" with the goal to release a Genesis Booster within nine months. Incredibly they managed to reach their target. The Sega 32X was released in November 1994.\\
\par
\fullimage{consoles/32X.png}%{Sega Genesis (a.ka. MegaDrive) with a 32X booster on top of it.}
\par
%The 32X is inserted into the Genesis like a standard game cartridge.
%\pagebreak
The 32X, as it would be marketed, was to be inserted like a cartridge. 32X games were inserted on top of the overall assembly. Games had access to everything including the Genesis's 7.6 MHz Motorola 68000 and the 3.58 MHz Zilog Z80.\\

\fq{After the call ended, Marty Franz grabbed one of those little hotel notepads and drew a couple of Hitachi SH2 processors, each with its own frame buffer. That's pretty much where the 32X started.}{Scot Bayless}\\
\par

\drawing{x32_arch}{What the notepad may have looked like.}

\par
\fq{The design of the graphics subsystem was brilliantly simple; something of
a coder's dream for the day. It was built around two central processors feeding independent frame buffers with twice the depth per pixel of anything else out there. It was a wonderful platform for doing 3D in ways that nobody else was attempting outside the workstation market.}{Scot Bayless}\\
\par

Besides the dual SH-2, the 32X was gifted an impressive audio chip from QSound. Capable of Pulse Wave Modulation, it added extra channels and even had a multidimensional sound capability allowing a regular stereo audio signal to approximate the 3D sounds heard in everyday life. Also present, a graphic chip name "VDP" in charge of double buffering (to avoid tearing) also capable to clear the framebuffers rapidly.
\par

\pngdrawing{x32_fullarch}{The 32X system as summarized in the developer documentation.}

The design beard similarities with the Saturn (which also used two SuperH) but with a different philosophy.\\
\par
\fq{The Saturn was essentially a 2D system with the ability to move the
four corners of a sprite in a way that could simulate projection in 3D space, It had the advantage of doing the rendering in hardware, but the rendering scheme also tended to create a lot of problems, and the pixel overwrite rate was very high; much of the advantage of dedicated hardware was lost to memory access stalls. The 32X, on the other hand, did everything in software but gave two fast RISC chips tied to great big frame buffers and complete control to the programmer.}{Scot Bayless} \\
\par
With engineers pouring their heart into the system and DevRel doing amazing work to have a decent portfolio of games for launch date, the 32X managed to sell 665,000 units at the end of 1994. A promising start followed by a sad story.

A story well summarized by Damien McFerran, reporter at RetroGamer.\\
\par
\fq{How do you take half a decade's worth
of critical and commercial success and  flush it down the toilet?\\
\par
 Easy: you release a device like the Sega 32X.}{Reporter for RetroGamer \#77}\\
\par

What crippled Mars was Saturn. During 1994 in SOJ, work on the 32-bit system had gone well. So well Sega decided to release it in Japan way ahead of what was originally scheduled on November 1994. The same month the 32X was to be released in USA.\\

\par

\fq{Not surprisingly, word got out quickly in the West, US and EU consumers immediately started asking the obvious question: 'Why should I buy 32X when Saturn is only a few months away?' Sadly, the best answer Sega could come up with was that 32X was a 'transitional device' – that it would form a bridge from Mega Drive to Saturn.\\
\par
 Frankly it just made us look greedy and dumb to consumers, something that a year earlier I couldn't have imagined people thinking about us. We were the cool kids.}{Scot Bayless}\\
 \par This poor timing made the 32X almost dead on arrival. Not only the Saturn was around the corner, the  Playstation 1 was released one month after the 32X on December 3, 1994. By the end of 1995, inventory was sadly liquidated at \$19.95 per unit.\\
\par
Looking back on this era and reading interviews is a bitter feeling when you keep in mind that up to that point in history Sega had been a colossal competitor to Nintendo. It had a cool image which it had taken five years to build\footnote{At some point, Sega was the biggest advertiser on MTV channel.}. From this point it seems the company focused on making one wrong decision after an other. Sega's last console, the Dreamcast released in 1998, ironically would end up being widely popular. Sales however were not enough and the Japanese company stopped manufacturing hardware to focus on programming games instead.\\ 





Scot Bayless testimony on the death march necessary to ship within nine months is eloquent.\\
\par
\fq{Games
in the queue were effectively jammed into a box as fast as possible, which meant massive cutting of corners in every conceivable way. Even from the outset, designs of those games were deliberately conservative because of the time crunch. By the time they shipped they were even more conservative; they did nothing to show off what the hardware was capable of.}{Scot Bayless}\\
\par
His analysis of deeper issued explains later mistakes rooted in Sega, Inc culture.\\
\par
\fq{
The 32X is a great case study in two things:\\
\par
First, messaging: your number one job in marketing
is to establish the value proposition. Even with all the rushed hardware and late software, if Sega had been able
to convince people that the 32X was really worth having, it might have had
a chance to succeed. But we never did that; we never managed to explain to anyone in any credible way what was so unique and worthy about the 32X. The result is exactly what you'd expect: Sony ate our lunch.\\
\par
 Second: honesty; not in the legal sense, nor in the public sense, but internally. I remember when I arrived at Microsoft in 1998 I attended an executive orientation briefing on my first day. The VP who met with us said: 'The one thing we demand of every one of you guys is to say what you think.' That attitude was what kept Microsoft vibrant, healthy and successful for
more than 20 years.
 Sega, by contrast, lacked that ruthless honesty. Nobody wanted to hurt anyone's feelings. Even when everybody knew the 32X and Saturn were way behind the power curve, nobody was willing to stand
up and say so. And it wasn’t just the hardware; during the same period, Sega published some of the oddest games it ever released. Games that were deeply flawed. Games that completely failed
to connect. And all the while everyone was smiling and saying, 'Gosh, aren't we great?'' I wasn’t able to articulate all this at the time, but I know I felt it intuitively. I knew there was something wrong, that we were losing our way.
}{Scot Bayless}





\subsection{Doom On 32X}
If porting \doom to Jaguar had been a tour de force, repeating the feat on a system even less powerful was to require nothing short than a miracle. Once again John Carmack invested himself totally in the project.\\
\par
\fq{I spent weeks working with Id Software's John Carmack, who literally camped out at the Sega of America building in Redwood City trying to get Doom ported. That guy worked his ass off and he still had to cut a third of the levels to get it done in time.\\ 
\par
What amazes me now is that with all that going on, nobody at Sega was willing to say "Wait a minute, what are we doing? Why don't we just stop?" Sega should have killed the 32X in the spring of 1994, but we didn't. We stormed the hill, and when we got to the top we realized it was the wrong damn hill.\\
\par
Looking back now I'd say that really was the beginning of the end for Sega's credibility as a hardware company.}{Scot Bayless}\\
\par
To make the game fit in the little RAM the 32X had, even more features than the Jaguar version had to be cut. An other enemy, the Spectre, had to be removed. All eight poses of each monsters was removed except for the one facing the player. Since they could no longer point towards each others, monster infighting was also removed. There were no save games, players instead manually selected the starting level instead. The cartridge only had enough room for seventeen heavily edited maps. Since none of theme had the BFG9000, the weapon is unavailable without using cheat codes.\\
\par
There was also a significant problem of performances. Even with its twin SuperH, the machine was unable to render at the original resolution.\\
\par
\fq{I liked the 32X -- it was basically two decent 32 bit processors (SH2) and a framebuffer, so you programmed like on a PC, but with SMP long before it was mainstream on PC.  It was still pretty underpowered compared to even a 386, so resolution was low.}{John Carmack}\\
\par
\cfullimage{consoles/x32_screenshot.png}{E1M1 legendary entrance hall}

In figure \ref{consoles/x32_screenshot.png} notice how, like on Jaguar, E1M1 blue floor texture had to be replaced with a brown one to limit RAM consumption. Likewise, the number of steps on the stairs was also lowered in order to reduce the number of visplanes generated.\\
\par
The game ran at a resolution of 320x224 but CPUs struggled so much that the active window was reduced to 128x144 (with horizontal line doubled to reach 256x144, leaving 100 vertical pixels for the status bar and the brown filling background. With all these compromises the framerate managed to remain within the 15-20 fps range which was a pleasant experience\footnote{Source}.\\
\par
\trivia{Sega named all its project after a planet of the solar system. Besides Saturn and Mars, two others are known of. Neptune was a two-in-one Genesis and 32X console which Sega planned to release in fall 1995. It was cancelled because of fears that it would dilute their marketing for the Saturn while being priced too close to the Saturn to be a viable competitor. Jupiter was rumored to have been a Saturn without a CD drive.}



\fullimage{32xe1m1.png}
\par
Map rework details. Above, E1M1 main room was stripped of many textures (compare PC version on page \pageref{complex_scene_plain_light.png}). Below, the "pit" of E1M3 have to be flattened.\\
\par
\fullimage{32xe1m2.png}






