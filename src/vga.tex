\section{VGA}
At first sight, the video system may have looked like it had not evolved much.  The video system was still the same weird VGA with its maze of 300 registers and VRAM made of four banks of 64 KiB. 
\par
DRAWING
\par
Programmed via mapped memory from \cw{0xA0000} to \cw{0xAFFFF}. It still had to be put used in 320x200 in unchained Mode-Y mode to make games.\\
\par
DRAWING
\par
And of course since the framebuffer aspect ratio did not match the CRT aspect ratio, the result was still distored.
\par
DRAWING
\par
But a closer insepction unveiled tremendous improvements. The early 90s saw the dawn of GUIs with IBM's OS/2 and Microsoft Windows 3.1. It was a demanding task to transit from pushing 4,000 bytes of information\footnote{2,000 bytes for the characters, and 2,000 bytes for screen attributes} in text mode to 300,000 bytes\footnote{256 colors at 640x480}.\\
\par
\cfullimage{Windows311workspace.png}{Windows 3.11 for Workgroup by Microsoft.}
\par
Two majors improvement occured. One came from VGA card manufacturers themselves. The other came from a Association of manufacturers defining a new technical standard which they called VLB\footnote{VESA Local Bus}.\\ 
\par


 
RAMDAC\footnote{Random Access Memory Digital-Analog Converter.}
\par
ARK 2000PV\\
ET4000\\
\par
Cirrus Logic VLB (fabless microbalbla)\\
Trident he rapid introduction of 3D graphics caught many graphics suppliers off guard, including Trident\\
\par
\par
40 to 100 times faster than ISA\\
\par
\subsection{VGA Chip manufacturer}
Fabless\footnote{Fabless manufacturing is the design and sale of hardware devices and semiconductor chips while outsourcing the fabrication.} semiconductor supplier started to pick interested in video systems. Companies such as Cirrus Logic, Matrox and ATI competed to produce the fastest chips, using SRAM in their RAMDACs. One of the most revered brand Tseng Labs manufactured probably one of the best VGA chipset of the time. The ET4000 chip was in high demand.\\
\par
\cfullimage{vlb_cards/tseng_lans_et4000_w32_vesa_local_illetw32.png}{ILLETW32 Britek Electronics. Photo courtesy \cw{http://www.amoretro.de/}}
\cfullimage{vlb_cards/hercules_vl_bus_dynamite_tseng_labs_et4000_w32p_2mb_vlb_vesa_local}{Hercules Dynamite. Photo courtesy \cw{http://www.amoretro.de/}}
\cfullimage{vlb_cards/tseng_labs_et4000_w32_vesa_local_vlb.png}{Photo courtesy \cw{http://www.amoretro.de/}}
\par


\subsection{VL-Bus}
The most intersting performance boost would not come from the graphic chips (this was a time before GPU demanding T\&L\footnote{Transformation \& Lightning}) but from the VESA (Video Electronics Standards Association). After more than ten years of service\footnote{The ISA was introduced in 1981}, the 16-bit ISA running at 8.3 Mhz, was starting to show its age. The VL-BUS was designed to hinder the 486 as little as possible and be on par with memory speed.
\pngdrawing{bus_isa}{ISA Bus}
\par
\pngdrawing{bus_vlb}{VLB Bus}

The speed of the system data bus is based on the clock rate of the motherboard's crystal. During the heyday of the VLB, this was usually 33 MHz, and VLB cards usually ran at half that rate, far outpacing the ISA bus. Some cards ran as fast as 50 MHz, using the full speed of the souped-up system bus. That often caused system crashes, because 50 MHz was outside the VLB specification.\\
\par
The chip design for the VLB controller was relativity simple, because many of the core instructions were hosted by the ISA circuits already on the motherboard, but the actual data passes were on the same local bus as the one used by the CPU.\\
\par
The design specification provides two other performance-boosting features: burst mode and bus mastering. In burst mode, VLB devices gain complete control of the external data bus for up to four bus cycles, passing up to 16 bytes (128 bits) of data in a single burst. Bus mastering allows the VLB controller to arbitrate data transfers between the external data bus and up to three VLB devices without assistance from the CPU. This limit of three devices also limited the maximum number of VLB slots to three and called for the use of a coprocessor. Display-system design is covered in more detail in Tutorial 11, "The Display System: Monitors and Adapters."\\
\par

BANDWIDTH CHART, ISA, ISA-16 bits and VLBUS.