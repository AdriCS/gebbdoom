The \textit{Game Engine Black Book: \doom{}} was first published on December 10th 2018, exactly twenty-five years after the December 10th 1993 release of the game. Within that timespan, it would be an understatement to say the world of \doom{} has flourished.\\
\par
\doom{} I was a colossal success, beloved by critics and gamers. At \$9 per unit the game quickly found itself making \$100,000/day. According to Sandy Petersen, the game "sold a couple of hundred thousand copies during its first year". Experts estimate that the game sold approximately 2-3 million physical copies from its release through 1999. In 1995 it was estimated \doom{} was installed on more computers than the Microsoft Windows operating system.\\
\par
The sequel, \doomii{}: Hell on Earth, was released in 1994. It was equally well received and managed to exceed sales expectations. More than 600,000 units where shipped to stores in preparation for the launch but it found itself sold out within a month. The game was the United States' highest-selling computer title of 1994. It placed 10th for 1996, with 322,671 units sold and \$12.6 million earned in the region that year alone.\\
\par
id Software took a break to develop its Quake brand for a few years until they released \doom{} III in August 2004. Featuring technological prowess such as dynamic shadows, the gameplay was slowed down to match the ambiance of a horror movie. Once again the title received favorable reviews from critics, and went on to become another successful title for id. In two years, 760,000 copies were sold for a total of \$32.4 million.  By 2007, the title had gone on to sell over 3.5 million copies, making it id's most successful project to date.\\
\par
 \doom{} 4 development started in 2007 but remained in limbo for several years. After a tumultuous development process and rumors of cancellation, id Software released \textit{Doom} (named the same as the first title in the series) in 2016. It was praised for its return to a fast pace and innovative game mechanics. It was the second best-selling retail video game in the US in May 2016, reaching 500,000 copies. By July 2017, the game reached 2 million copies sold on PC.\\
\par

\thispagestyle{plain} % Trick to remove header




The \doom{} team from 1993 parted ways over the years, although the legacy and status associated with the project continued to follow each of their careers. Before \doom{}, John Romero famously shared his ambition to reach a level of success similar to Scott Miller and George Broussard from Apogee. He reportedly said to John Carmack: \\
\par
 \rawfq{They're driving bad-ass cars while we drive ass cars. It is time to kick ass.}\\
 \par
After \doom{}, \textit{ass car} driving was no more. Romero's Ferrari Testarossa (modified with a COM port connected to the engine) and John Carmack's Ferrari F-40 were notorious in both the gaming and programming worlds.\\
\par
Beyond sales and fame, \doom{} reached a new dimension when its source code was open sourced on December 23, 1997. Hundreds of ports ensued, some still actively developed to this day, among them: LinuxDoom, DOOM 95, DosDoom, Chocolate Doom, ZDoom, BOOM, EDGE, Doom Legacy, Doom Retro, Crispy Doom, Doomsday Engine, GZDoom, csDoom, MBF, PrBoom, 3DGE, Risen3D, QZDoom, Skulltag, ZDaemon, Odamex, SMMU, PrBoom+, Zandronum and Eternity Engine -- and these are only the most famous.\\
% \par
% In an amusing turn of events, id Software used the \textit{PrBroom} port project code to release \textit{Doom Classic} on iOS.\\
\par
\vspace{10pt}
\b{Personal Note:}\\
\par
\doom{} has a special place in my heart. As a 24-year old immigrant in Toronto knowing only Java, this was the codebase I used to learn C and build up my skills. It is the quality level I set myself to emulate. It is thanks to the "University of id Software" as I like to call it that I ended up being noticed by Google and eventually landed a job offer for what was my dream job at the time. Something I once deemed impossible to achieve.\\
\par
The title of the book \textit{Game Engine Black Book} is an homage to Michael Abrash. The explanations in his \textit{Graphics Programming Black Book} unlocked the most difficult parts of Quake. Michael's book features a quote that resonated with me. I have tried to live by it and so far it has served me well. Maybe you will also find it inspiring and it will guide you the same way it has guided me in times of discouragement.\\
\par
\fq{If you do what you love, and do it as well as you can, good things will eventually come of it. Not necessarily quickly or easily, but if you stick with it, they will come.}{Michael Abrash}\\
\fullimage{ports.png}
\thispagestyle{plain} % Trick to remove header
