\section{Filesystem}
\doom barely interacts with the operating system file system. During a typical gaming session, the engine \cw{DOOM.EXE} only needs to open \cw{DOOM.WAD} in order to access its assets. Therefore, the engine does not deal with files but rather with something called \cw{lump}s which are the atomic unit of a \cw{.wad} archive.\\ 
\par
\trivia{The extension name was coined by Tom Halls, when John Carmack asked him "How do you call a file where's all the data?". Tom's answer was a "WAD" (Where's All the Data).}\\
\par
Among all version of \doom distributed, the engine executabler was always the same\footnote{Apart for a few bug fixes.}. Only the asset file name and content changed.\\
\par
 \begin{figure}[H]
\centering  
\begin{tabularx}{\textwidth}{ L{1.3}  L{0.7} R{1}}
  \toprule
  \textbf{Game} &  \textbf{Archive name} & \textbf{Size in bytes}\\

  \toprule 
  Doom Shareware & \cw{DOOM1.WAD} &  4,196,020 \\
  Doom Registered & \cw{DOOM.WAD} & 11,159,840 \\
  Doom II: Hell on Earth & \cw{DOOM2.WAD} & 14,604,584\\
  Ultimate Doom & \cw{UDOOM.WAD} & 12,408,292\\
  The Plutonia Experiment & \cw{PLUTONIA.WAD} & 17,420,824\\
TNT: Evilution &  \cw{TNT.WAD} & 18,195,736\\
French Doom II & \cw{DOOM2F.WAD} & 14,607,420\\
   \toprule
  Heretic Shareware & \cw{HERETIC1.WAD} & 5,120,920\\
    Heretic Registered & \cw{HERETIC.WAD} & 14,189,976\\
   \toprule
  Hexen Shareware & \cw{HEXENDEMO.WAD} & 10,644,136\\
  Hexen Registered & \cw{HEXEN.WAD} & 20,083,672\\
  Hexen: Deathkings & \cw{HEXEDD.WAD} & 4,440,584\\
   \toprule
\end{tabularx}
\caption{WAD files in the many versions of doom\protect\footnotemark}
\end{figure}
\par
\footnotetext{Source: doomgod.com "Internal War Allocation Daemons"}

Lumps are identified by an unique name on eight characters (which conveniently matches DOS filename length limitation). The smallest wad, \cw{DOOM1.WAD} is made of 1264 lumps.\\
\par
\ccode{lumpinfo_t.c}{}
\pagebreak

\begin{figure}[H]
\centering  
\begin{tabularx}{\textwidth}{ L{0.2}  L{0.8}}
  \toprule
  \textbf{Lump Name} &  \textbf{Usage} \\
   
  \toprule 
  \cw{PLAYPAL} & The fourteen palettes used at runtime. Detailed on page \pageref{label_palettes} \\
  \cw{COLORMAP} & Translation tables to simulate 32 shades of each 256 colors. Detailed on page \pageref{diminishedlightning}. \\
  \cw{DEMO?} &  Record of game sessions by id Software members. Played when the game startup as demo.\\
  \toprule
  \cw{MAP??} & Zero sized lump serving as marker for the beginning of a series of map lumps. The first ? is the episode number and the second ? is the map id.\\
  \cw{THINGS} & All monsters, weapons, ammo and sprite contained in the map designed by the last map marker.\\
  \cw{LINESDEFS} & All lines referenced by \cw{SECTORS}.\\
  \cw{SIDEDEFS} & All side referenced by \cw{LINESDEFS}. A line can have up to two sides.\\
  \cw{VERTEXES} & All vertices in the current map.\\
  \cw{NODES} & A binary tree allowing to sort line segments efficiently. \\
  \cw{SSECTORS} &  The Sub-Sectors, leaves of the binary tree in \cw{NODES}.  \\
  \cw{SEGS} &  The segments pointed to by the \cw{SSECTORS} lumps.\\
  \cw{SECTORS} &  Referenced by subsectors. Details ceiling/floor height, texture, and lighting properties.\\
  \cw{REJECT} &  An A.I acceleration datastructure to detect if monsters can see a player.\\
  \cw{BLOCKMAP} & A collision detection acceleration structure slicing the map 128x128. Provide fast access to all LINESDEFS neighboring any point on the map. Detailed on page \pageref{blockmapdetails} \\
  \toprule
  \cw{DP.*} &  SFX in PC Speaker format.\\
  \cw{DS.*} &  SFX in PCM Mono, 8-bit 11,025Hz.\\
  \cw{D\_.*} & Music in MUS format (a slightly altered MIDI format).\\
  \toprule
  \cw{ENDOOM} & Text-mode exit screen to entice players to by the full version. \\
  \cw{DMXGUS} & Translation table to match a MIDI instrument with a Gravis Ultra Sound sample file.\\
  \cw{GENMIDI} &  Bank of instrument data to play MIDI music with an OPL audio chip.\\
  \cw{PNAMES} &  Lists all lump names used as wall patches.\\
  \cw{TEXTURE1} &  A dictionary of all walls textures lumps referenced by \cw{SIDEDEFS}. Used to speed up access and allocation at runtime.\\  
  \cw{F\_START} &  Zero sized lump marking the beginning of flats textures.\\  
  \cw{F\_END} &   Zero sized lump marking the end of flats textures.\\  
  \cw{S\_START} & Zero sized lump marking start of item/monster "sprite" section. \\  
  \cw{S\_END} & Zero sized lump marking end of item/monster "sprite" section. \\  
  \cw{P\_START} & Zero sized lump marking the beginning of wall textures.\\
  \cw{P\_END} & Zero sized lump marking the end of wall textures.\\
  \cw{.*} &  Many others, fonts, \cw{TITLEPIC}, \cw{HELP} screens, intermission screens, \cw{VICTORY} screen... \\  
   \toprule
\end{tabularx}
%\caption{WAD lump types}
\end{figure}
\par
\pagebreak






\subsection{Lumps}
The lump system is the less glamoruos part of the engine but one of the coolest in its implementation and what it had to offer to players.\\
\par
When it starts, the system looks at every wad archive provided and indexes every single lump found into a gigantic array of \cw{lumpinfo\_t} cunningly named \cw{lumpinfo}. 
Additional wad archives can be provided thanks to \cw{-file} command-line parameter. In this case, lumps belonging to official id Software (\cw{DOOM1.WAD}, \cw{DOOM2.WAD},...) are added to \cw{lumpinfo} first.\\
\par
\vspace{4in}
\par
The wad lumpsystem provide access to each lump prodideda lump name and a hint indicating how it will be used. The first task is to find out where the lump in in the \cw{wad}. To this effect a lump dictionary takes care of translating a \cw{char[8]} into a lumpinfo index. The search is sequencial but to speed up comparaision the string of eight is treated as two integers of 32-bit.\\
\par
\ccode{W_CheckNumForName.c}{}\\
\par


At startup the wad file is loaded and the lump directory parsed so all lumps are known. At runtime when an asset is needed it is requested and refered to by its lump name.\\
\par
\ccode{W_CacheLumpName.c}


\subsection{Inheritance system}
Upon starting the engine automatically looked for known wads such as \cw{DOOM1.WAD}, \cw{DOOM.WAD}, or \cw{DOOM2.WAD}. It also accepted a parameter \cw{-file} allowing to load additional wad. These additional WADs took precedence over the stock wads in the sense that when a lump was requested the engine looked it that directory first. If that failed, the lookup went down the wad chain.\\
\par
\trivia{Additional wad had to use the magic number \cw{PWAD} for Player WAD while the magic number IWAD was reserved for Id Software WADs.}\\
\par
 This elegant design allowed modders to overwrite some aspect of stock wad while keeping the rest. For mappers this meant only shipping a small \cw{PWAD} with the map lumps and still be able to use the textures from the \cw{IPWAD}.\\
 \par