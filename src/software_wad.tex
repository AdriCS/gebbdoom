\section{Filesystem}
\doom barely interacts with the operating system file system. During a typical gaming session, the engine \cw{DOOM.EXE} only needs to open \cw{DOOM.WAD} in order to access its assets. Therefore, the engine does not deal with files but rather with something called \cw{lump}s which are the atomic unit of a \cw{.wad} archive.\\ 
\par
\trivia{The extension name was coined by Tom Halls, when John Carmack asked him "How do you call a file where's all the data?". Tom's answer was a "WAD" (Where's All the Data).}\\
\par
Among all version of \doom distributed, the engine executabler was always the same\footnote{Apart for a few bug fixes.}. Only the asset file name and content changed.\\
\par
 \begin{figure}[H]
\centering  
\begin{tabularx}{\textwidth}{ L{1.3}  L{0.7} R{1}}
  \toprule
  \textbf{Game} &  \textbf{Archive name} & \textbf{Size in bytes}\\

  \toprule 
  Doom Shareware & \cw{DOOM1.WAD} &  4,196,020 \\
  Doom Registered & \cw{DOOM.WAD} & 11,159,840 \\
  Doom II: Hell on Earth & \cw{DOOM2.WAD} & 14,604,584\\
  Ultimate Doom & \cw{UDOOM.WAD} & 12,408,292\\
  The Plutonia Experiment & \cw{PLUTONIA.WAD} & 17,420,824\\
TNT: Evilution &  \cw{TNT.WAD} & 18,195,736\\
French Doom II & \cw{DOOM2F.WAD} & 14,607,420\\
   \toprule
  Heretic Shareware & \cw{HERETIC1.WAD} & 5,120,920\\
    Heretic Registered & \cw{HERETIC.WAD} & 14,189,976\\
   \toprule
  Hexen Shareware & \cw{HEXENDEMO.WAD} & 10,644,136\\
  Hexen Registered & \cw{HEXEN.WAD} & 20,083,672\\
  Hexen: Deathkings & \cw{HEXEDD.WAD} & 4,440,584\\
   \toprule
\end{tabularx}
\caption{WAD files in the many versions of doom\protect\footnotemark}
\end{figure}
\par
\footnotetext{Source: doomgod.com "Internal War Allocation Daemons"}

Lumps are identified by an unique name on eight characters (which conveniently matches DOS filename length limitation). The smallest wad, \cw{DOOM1.WAD} is made of 1264 lumps.\\
\par
\ccode{lumpinfo_t.c}{}
\pagebreak

\begin{figure}[H]
\centering  
\begin{tabularx}{\textwidth}{ L{0.2}  L{0.8}}
  \toprule
  \textbf{Lump Name} &  \textbf{Usage} \\
   
  \toprule 
  \cw{PLAYPAL} & The fourteen palettes used at runtime. Detailed on page \pageref{label_palettes} \\
  \cw{COLORMAP} & Translation tables to simulate 32 shades of each 256 colors. Detailed on page \pageref{diminishedlightning}. \\
  \cw{DEMO?} &  Record of game sessions by id Software members. Played when the game startup as demo.\\
  \toprule
  \cw{MAP??} & Zero sized lump serving as marker for the beginning of a series of map lumps. The first ? is the episode number and the second ? is the map id.\\
  \cw{THINGS} & All monsters, weapons, ammo and sprite contained in the map designed by the last map marker.\\
  \cw{LINESDEFS} & All lines referenced by \cw{SECTORS}.\\
  \cw{SIDEDEFS} & All side referenced by \cw{LINESDEFS}. A line can have up to two sides.\\
  \cw{VERTEXES} & All vertices in the current map.\\
  \cw{NODES} & A binary tree allowing to sort line segments efficiently. \\
  \cw{SSECTORS} &  The Sub-Sectors, leaves of the binary tree in \cw{NODES}.  \\
  \cw{SEGS} &  The segments pointed to by the \cw{SSECTORS} lumps.\\
  \cw{SECTORS} &  Referenced by subsectors. Details ceiling/floor height, texture, and lighting properties.\\
  \cw{REJECT} &  An A.I acceleration datastructure to detect if monsters can see a player.\\
  \cw{BLOCKMAP} & A collision detection acceleration structure slicing the map 128x128. Provide fast access to all LINESDEFS neighboring any point on the map. Detailed on page \pageref{blockmapdetails} \\
  \toprule
  \cw{DP.*} &  SFX in PC Speaker format.\\
  \cw{DS.*} &  SFX in PCM Mono, 8-bit 11,025Hz.\\
  \cw{D\_.*} & Music in MUS format (a slightly altered MIDI format).\\
  \toprule
  \cw{ENDOOM} & Text-mode exit screen to entice players to by the full version. \\
  \cw{DMXGUS} & Translation table to match a MIDI instrument with a Gravis Ultra Sound sample file.\\
  \cw{GENMIDI} &  Bank of instrument data to play MIDI music with an OPL audio chip.\\
  \cw{PNAMES} &  Lists all lump names used as wall patches.\\
  \cw{TEXTURE1} &  A dictionary of all walls textures lumps referenced by \cw{SIDEDEFS}. Used to speed up access and allocation at runtime.\\  
  \cw{F\_START} &  Zero sized lump marking the beginning of flats textures.\\  
  \cw{F\_END} &   Zero sized lump marking the end of flats textures.\\  
  \cw{S\_START} & Zero sized lump marking start of item/monster "sprite" section. \\  
  \cw{S\_END} & Zero sized lump marking end of item/monster "sprite" section. \\  
  \cw{P\_START} & Zero sized lump marking the beginning of wall textures.\\
  \cw{P\_END} & Zero sized lump marking the end of wall textures.\\
  \cw{.*} &  Many others, fonts, \cw{TITLEPIC}, \cw{HELP} screens, intermission screens, \cw{VICTORY} screen... \\  
   \toprule
\end{tabularx}
%\caption{WAD lump types}
\end{figure}
\par
\pagebreak






\subsection{Lumps}
The lump system is the less glamorous part of the engine but one of the coolest in its implementation and what it had to offer to players.\\
\par
Upon starting up, it looks at every wad archive provided and indexes every lumps found into a gigantic array of \cw{lumpinfo\_t} cunningly named \cw{lumpinfo}. 
If additional archives are provided via \cw{-file} command-line parameter, lumps belonging to official id Software (\cw{DOOM1.WAD}, \cw{DOOM2.WAD},...) are added to \cw{lump info} first.\\
\par
In the following example, \doom was started with the command-line:\\
\par
\fakedosoutput{dos_doom_wad}
\par
Which results in the following lump dictionary:\\
\par
\pngdrawing{lumpsManager}{For simplicity of drawing, \cw{DOOM1.WAD} is represented with only three lumps and the two additional WAD archive have only one lump each.}
\par
Request to the lump system are sent via a \cw{char[8]} name. The first task to make the lump available is to find out in which wad it is and at which offset. The dictionary is searched sequentially in function \cw{W\_CheckNumForName} which in order to speed up comparison uses a cool trick. Instead of comparing eight character each times, it compares two integers of 32-bit.\\
\par
\ccode{W_CheckNumForName.c}{}\\
\par
In the code listing above, you will notice that the dictionary is searched starting from the end. This done intentionally to embrass the modding community and allow fan made wad archives to overwrite id software lumps.\\
\par
The beauty of this system means that the original \cw{DOOM1.WAD} never had to be modified or patched. Any of the assets could be overwritten, maps, music, sfx, to graphics\footnote{With the exception of A.I which is backed in the executable} could be swapped with a simple command-line.\\
\par
\trivia{To differentiate official wad from fan made ones, the magic number at the beginning of a WAD archive was different. \cw{IWAD} was reserved for id Software while fan made were requested to use \cw{PWAD}.}\\
\par
Once a lump location has been found, a memory block is requested from the memory allocator. The content of the lump is copied to RAM and returned to the caller. \\
\par
The lump system features a fuzzy caching system. The \cw{lumpinfo} array is mirrored by a \cw{lumpcache} array. Every time a lump is requested, the cache is looked up first. The lump cache slot set itself as the \cw{user} of the memory block which means the cache is automatically invalidated if the block is freed.\\
\par
DRAWING\\
\vspace{7cm}
DRAWING\\
\par
\ccode{W_CacheLumpName.c}



Some fan used the overwritting system to replace almost every aspect of the original game. These mods were know as "Total Conversion". The most notorious of them all, was name Alien Total Conversion.\\
\par
Released in December 1994, it took Justin Fisher one year of hard work. Amusingly the result ended up fulfilling the aliens motion picture theme id Software briefly considered before going for demons.\\
\par
 \fullimage{alientc.png}{}
 \par
 The list of modifications was impressive at the time. Many sound effects such as doors, weaponry, and explosions were straight from the movie. Actors signatures "Let's rock" screams had been digitized. All demons had been replaced with aliens, eggs, facehuggers and even the alien queen.
The weapons also received the treatment. The Pulse Rifle, the Grenade Launcher, the Smart-Gun are there and even the chainsaw was replaced with the "Caterpillar P-5000 Work Loader".\\
\par
Map design was far from being neglected. AlienTC did a breathtaking job at replicating Aliens worrisome and intriguing atmosphere. In order to build up paranoia, the first level features no enemies.
