\section{Graphic assets}
% Everything, from walls to floors and ceiling would textured. 
\subsection{Sprites}
Given the ambitions, there was a lot of artwork to produce. Weapons were animated when firing. Monsters had to have eight poses depending on the viewing angle. Wall surfaces, ceilings, and floors were to be textured. And this is not to mention all the "utility" art for the menus, intermission and final screens. It was an immense undertaking for a team of only two artists working with just Deluxe Paint.\\
\par
%They solved the problem with an interesting combinations of skills, hard work, money and guile.\\
Monsters represented most of the workload. Drawing a single sprite while facing the player was easy. Drawing it seven more times at increasing angles (45$^{\circ}$, 90$^{\circ}$, 135$^{\circ}$, 180$^{\circ}$, 225$^{\circ}$, 270$^{\circ}$, and 315$^{\circ}$) for each action was hell. To solve this problem they created a new process, leveraging both their artistic talent and the technological power of the NeXTDimension. First they drew it on paper, then they applied clay on a wooden manikin to bring it to life.\\
\par


\scaledimage{0.48}{sprites/doomguy_sketch.png}
\scaledimage{0.5}{sprites/doomguy_clay.png}\\


\begin{wrapfigure}[8]{r}{0.4\textwidth}
\centering
\scaledimage{0.4}{SonyHi8.png}
\end{wrapfigure}

\vspace{-4mm}
Once the character was carved they could change the pose at will. They only had to connect a Handycam Hi8 Sony video camera to the NextDimension. Placed on a spinner, the clay model was lit and digitized from eight viewpoints. It was a much faster and a more fun process.\\
\par
The output from the NeXTDimension was a 24-bit TrueColor image which had to be transformed to the \doom{} palette\footnote{See the \doom{} palette on page \pageref{doom_palette}.} of 256 colors via a tool called the "Fuzzy Pumper Palette Shop". To complete each sprite, artists performed manual coloring with Deluxe Paint.\\
\par
\fullimage{sprites/doomguy_fpps.png}\\
\par
The process was not without it own flaws since the clay dried and had a tendency to break instead of folding. Nevertheless, seven Doom characters were built as sculptures for \doom{} \& \doomii{}. The first models - the Doomguy, Baron of Hell and Cyberdemon -- were all sculpted by Adrian Carmack. The iconic Imp, Zombieman, and Sergeant were all mouse-drawn by Kevin Cloud.\\
\par
\trivia{Most models survived. Some are still in John Romero's possession while others are visible at id Software's headquarters. A few models managed to escape into the wild and are now highly priced by collectors.}

\cfullimage{sprites/adrian_hellknight.png}{Adrian Carmack sculpting the Hell Knight, working from his preliminary drawing.}
\par
\fullimage{sprites/hellknight_fpps.png}




\vspace{-4mm}
Using clay models was faster than drawing by hand but it was still not fast enough to produce the many monsters necessary. It also had limited capabilities in terms of textures since it was impossible to render specular material such as metal. There were also issues with intricate details which were way too fine to survive clay modeling. They needed better models, possibly stop-motion capable.\\
\par
Don Ivan Punchatz, who had been commissioned for the \doom{} package art and logo, mentioned he had a son doing exactly that. Greg had been successfully providing stop animation models for big Hollywood production such as \textit{A Nightmare on Elm Street 2}, \textit{RoboCop} and its sequels, and \textit{Coming to America}. \\
\par
Kevin Cloud got in touch with Greg Punchatz and the young artist was promptly commissioned for the Arch-Vile, Mancubus, Revenant and Spiderdemon.\\
\par

\fq{The spider creature was made out of parts I had literally just found at hardware and hobby stores, pieces of Tupperware and PVC pipes. The main body started out as a sculpture, then a plaster mold was pulled from that. Then we made the armature to fit that mold, and then foam latex was injected inside the mould and put into an oven.\\
\par
Mastermind's legs pretty much only just moved, and his arms moved, but his mouth didn't move. As we went along, the other maquettes become full ball and socket armatures, so they had a full range of motion. In some ways, these stop-motion maquettes are easier to get right than they would be in CG. You don't have to worry about how your skin is weighted on stop-motion model because it just sticks to the metal armature.}{Greg Punchatz, Interview by develop-online.net Feb 16, 2016}\\
\par

\fq{At one stage id offered me points on the backend to take \$500 off the price of one of the characters and I turned that down. It's a painful lesson. But to be part of something that has left a long-lasting impression on the world is kind of crazy -- people find out that I worked on Doom and it's like I played on the Beatles' White Album.}{Greg Punchatz, Interview by develop-online.net Feb 16, 2016}

\cfullimage{sprites/spiderdemon_model.png}{Notice the spinner, camera, and a virgin wooden manikin on the table}
\par
\fullimage{sprites/spiderdemon_model_fuzzy.png}





\subsection{Weapons}
The starting points for the weapons were mostly toys, digitized with NeXTDimension and heavily cleaned up via Deluxe Paint.\\
\par
 The shotgun was in fact the "TootsieToy Dakota" cap shotgun, manufactured by the Strombecker Corporation of America.\\
\par
\cfullimage{Doomshotgun.png}{TootsieToy Dakota. The thing had a rifle fire mode. (Courtesy of James Miller)}
\par
The chainsaw was a real, fully functional, McCulloch Eager Beaver. It was borrowed from Tom Hall's girlfriend.\\
\par
\cfullimage{chainsaw.png}{McCulloch Eager Beaver (Courtesy of James Miller)}
\par
Luckily it was only brought in after Romero locked himself up in his office. That could have been interesting.\\
\par
\trivia{The chainsaw worked well but it was leaking oil abundantly. They had to store it on top of a bowl on the ground.}\\
\par





\fullimage{props/chainsaw.png}\\
\par

The chaingun (photo below) was another Tootsietoy toy called the Ol' Painless. To the delight of many parents it was able to produce loud firing sounds when fitted with a 9V battery.

\par
\fullimage{props/chaingun.png}

\vspace{-5mm}
The fist was actually Kevin Cloud's hands wearing a knuckle duster. The Plasma rifle was based on the grenade launcher of a Rambo III M-60 toy set (upper left element in the box).
\par
\fullimage{props/rambo.png}\\
\par
The BFG 9000 was the RoarGun by Creatoy. It was photographed sideways, mirrored and inclined at an acute angle to give it more depth.\\
\par
\fullimage{props/bfg.png}
\pagebreak


\subsection{Skies}
\begin{wrapfigure}[7]{r}{0.4\textwidth}
\centering
\scaledimage{0.4}{skies/doom/SKY1.png}
\end{wrapfigure}
Since the player was to travel to several satellites of Mars during the game, matching SKY textures had to be produced.\\
\par In order to generate the "real" touch they wanted, Kevin and Adrian bought a set of 10 royalty free CD-ROMs called MediaClips. Each CD had a theme (Jets, Majestic Places, Props, Wild Places, Worldview) and the whopping 650MiB capacity allowed one hundred high-resolution (640x480) photos per CD.\\
\par 

\begin{wrapfigure}[7]{r}{0.4\textwidth}
\centering
\scaledimage{0.4}{skies/doom/SKY2.png}
\end{wrapfigure}

Since they did not have much time for Episode I which had to ship with the shareware in December, they simply cropped Yangshuo Cavern from China to the sky standard 256x128 resolution.  With more time for the other episodes they became more creative and composed images from numerous sources. The skylines of Phobos, Deimos and Hell ended up borrowing from places like China, Zion, and Hawaii. \\
\par

Because the sky repeats four times in the engine, the texture had to be patched in order to wrap at the edges. Notice how the right and left edges connect without any discontinuity.\\
\par
\trivia{Can you guess where the clouds in \doomii{} Episode 2 skies are from?}\\
\par
\fullimage{skies/doom2/RSKY2.png}





\begin{minipage}{\textwidth}
\fullimage{skies/doom/SKY2.png}
\par
\doom{}, Episode II. Made of Zion's Watchman rock formation and a red tainted sunset.\\
\par
\fullimage{e2composite.png}
\end{minipage}






\begin{minipage}{\textwidth}
\fullimage{skies/doom/SKY1.png}
\par
\doom{}, Episode I. Yangshuo Cavern in China. (All compositions are courtesy of James Miller).\\
\par
\fullimage{MAJEST3.png}
\end{minipage}
\par






\begin{minipage}{\textwidth}
\fullimage{skies/doom2/RSKY1.png}
\par
\doomii{}, Episode I. Hawaii beach at sunset.\\
\par
\fullimage{WILD4.png}
\end{minipage}


\begin{minipage}{\textwidth}
\fullimage{skies/doom2/RSKY2.png}
\par
\doomii{}, Episode II. Challenger rocket take-off used for burning city clouds.\\
\par
\fullimage{WORLD2.png}
\end{minipage}







