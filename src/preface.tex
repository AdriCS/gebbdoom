This is the second book in the \textit{Game Engine Black Book} series. It picks up right where the first one ended with the release of Wolfenstein 3D in May 1992. It carries on all the way down to December 1993 with the second breakthrought of the 90s in the world of PC gaming, \doom.\\
\par
 Like its predecessor, this volume attemps to describe in great details both the hardare and the software of the era. It opens a window back in time peeking over the engineering used to solve the various problems id Software had to face during the eleven months its took them to ship their new title.\\% which resulted in what is universally considered one of the best game of all times.\\
\par
It may seem odd to write a book about a game twenty-five years after its release. After all, who would be interested in a seemingly outdated technology dedicated to extinct hardware? It turns out there are many reasons people may want to learn about \doom and each would have warranted a book on their own.\\ 
\par
First, \doom has had such a profound and sustained impact that it is part of modern history. An unquestionable milestone, entertaining millions, shaping the industry, it steered entire careers. Because the source was available, programmers have learned to code with it. Because it was easy to modify, countless aspiring game makers first started by making levels or drawing assets for \doom. To this day, because it is such an icon, it has become the goto title for a hacker willing to demonstrate skills. From MacBook Touchbar, ATMs, CT scanner, watches and even fridges\footnote{\doom has been ported to absolutely everything with a screen and a CPU, there is even a website "itrunsdoom.tumblr.com" tracking it.} pretty much any piece of electronic has run \doom.\\
\par


There is also of course the sentimental value. \doom is one of these title which made an ever lasting impression upon first contact. Many of us were just in our teen years when this game came out and most are still able to remember im which circustances they first saw it running. The author can vividly remember going over to the neighbor's house (who owned a 486) and thinking it was the coolest thing he had ever seen (after Eric Cantona of course).\\
\par
\doom was a financial and critical success. During 1994 it received many awards, including \textit{Game of the Year} by both \textit{PC Gamer} and \textit{Computer Gaming World}, \textit{Award for Technical Excellence} from \textit{PC Magazine}, and \textit{Best Action Adventure Game award} from the \textit{Academy of Interactive Arts \& Sciences}. With more two millions unit sold and an estimated 20 millions shareware installation it generated close to \$100,000 on a daily basis. It reached such quality that players still buy and play it in 2017.\\
\par

Beyond the nostalgia, and this is the most important reason this book is relevant, the making of \doom is the ever repeating story of inventors, engineers and builders gathered around a common vision. There was no clear path between where the id Software team was and where they wanted to be. Only the certitude that nobody else had gone there before. They worked days and nights, slept on the floor\footnote{Or couch in the case of Dave Taylor.} to make their dream come true. Alike building pyramids or bringing men on the moon, \doom happening summarizes perfectly how achieving something one big goal assimilates to do a thousands small things right. This is the story of a bunch of dreamers who had the skills and the luck to take then all the way resulting in a perfect storm of game engine, artwork and design.\\
\par
 To tell this story, the author faced two seemingly orthogonal constraints. On one side the desire to have this book stand alone, without need for supplemental information or cross-reference to previous books. On the other side, avoid boring faithful readers with content described in the previous entry.\\
\par
Although content re-usage was kept to a minimum, ultimately, new readers were favored. It would have been a frustrating experience to constantly direct somewhere else for missing knowledge. The book would have felt incomplete. I hope readers with previous knowledge from \textit{Game Engine Black Book: Wolfenstein 3D} will forgive being bothered with VGA architecture, floating point internals, and i386 segmented memory. It is not a stratagem to increase the page count. There is much truth in "Less is more" and 300 pages would have been preferable to the 500 of this volume but the complexity of the engine code and love for details prevented it.\\
\par
-- Fabien Sanglard (fabiensanglard.net@gmail.com)