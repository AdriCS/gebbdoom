This is the second book in the \textit{Game Engine Black Book} series. It picks up right where the first one ended with the release of Wolfenstein 3D in May 1992. This volume describes the engineering history of a tile which was developed within just 11 months and would end up becoming one of the best game of all times: \doom.\\
\par
It may seem odd to write a book 25 years after a game was released. Who would read about a seemingly outdated technology dedicated to extinct hardware? It turns out there are many reasons. For one, \doom has had such a profound and sustained impact that it has become part of computing history. An unquestionable milestone, entertaining millions and triggering countless careers. Because the source code was officially released, many programmers have learned to code by looking under the hood. Because it was easy to modify, countless\footnote{Often now professional game designers.} started to design levels or generating new assets. Because it was easy to port to a new system, to this day it is often the first "tour de force" for a programmers\footnote{From MacBook TouchBar, fridges to ATM, \doom has been ported to absolutely everything with a screen and a CPU.} willing to demonstrate his skills.\\
\par
Yet, beyond the sentimental value, this is the story of a team of people who gathered around a common vision. They worked days and nights, sometimes sleeping on floors\footnote{Or couch in the case of Dave Taylor} to make their dream come true. This is the never ending story of the race of engineers and builders. From those who build pyramids, to those who brought men on the moon, the story of \doom describe well what it takes to achieve greatness. To do a thousand small things right in order to execute one perfect motion.\\
\par
In the case of \doom, the perfect storm happened with a fistful of people managing to build the best tools in order to produce an engine capable of powering breathtaking game design and assets. They released something of such great quality it is still being purchased 25 years later.\\
\par
 To tell this story, the author faced two seemingly orthogonal constraints. On one side the desire to have this book stand alone, without need for supplemental information or cross-reference to previous books. On the other side, avoid boring faithful readers with content described in previous entries.\\
\par
Although, content re-usage was kept to a minimum, ultimately, new readers were favored. It would have been a frustrating experience to constantly direct the readers somewhere for missing knowledge Even though it is part of a series, the book would have felt incomplete.\\
\par
I hope readers with previous knowledge from \textit{Game Engine Black Book: Wolfenstein 3D} will forgive being faced with VGA architecture description, floating point internals, and i386 segmented memory. It is in no way a cheap stratagem to increase the page count. Less is more and 300 pages would have been preferable but the complexity of the engine code and love for details prevented it.\\
\par
I hope you will enjoy reading it as much as I enjoy writing it.\\
\par
-- Fabien Sanglard (fabiensanglard.net@gmail.com)