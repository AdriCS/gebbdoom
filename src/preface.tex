This is the second book in the \textit{Game Engine Black Book} series. It picks up right where the first one ended with the release of Wolfenstein 3D in May 1992. It carries on all the way up to December 1993 with id Software's second breakthrough of the 90s, \doom{}.\\ %in the world of PC gaming
\par
 Like its predecessor, this volume attempts to describe in great details both the hardware and the software of the era. It opens a window back in time peeking over the engineering used to solve the various problems id Software encountered during the eleven months its took them to ship their next title.\\% which resulted in what is universally considered one of the best game of all times.\\
\par
It may seem odd to write a book about a game twenty-five years after its release. After all, who would be interested in a seemingly outdated technology dedicated to extinct hardware running obsolete operating systems? Given the success of the first \textit{Black Book}, it turns out many. Whether readers are into history, nostalgia, engineering, or even philosophy, its seems there is an edge for everyone.\\ 

\par
\doom{} has had such a profound and sustained impact that it has become part of modern history. It is an unquestionable milestone which entertained millions and catalyzed vocations. Because the source code was available, programmers have learned game engine architecture with it. Because it was easy to modify and the tools were available, countless aspiring game makers have had their first experience by designing or drawing assets. To this day, because it is such an icon, it is often the goto title for hackers willing to demonstrate their skills\footnote{Upon cracking BitFi wallet in Aug 2018, the hacker team demonstrated by running \doom{} on the device.}. From MacBook Touchbar, ATMs, CT scanner, watches, and even fridges, pretty much any piece of electronic has run \doom{} \footnote{There is a website "itrunsdoom{}.tumblr.com" tracking what \doom{} has been ported to.}.\\
\par

It was a financial and critical success which reshaped the PC gaming industry\footnote{And even killed the Amiga. Source: "Commodore: The Amiga Years" by Brian Bagnall.}. During 1994 it received several awards, including \textit{Game of the Year} by both \textit{PC Gamer} and \textit{Computer Gaming World}, \textit{Award for Technical Excellence} from \textit{PC Magazine}, and \textit{Best Action Adventure Game award} from the \textit{Academy of Interactive Arts \& Sciences}. With more than two millions unit sold and an estimated 20 millions shareware installation, at its highest the phenomenon generated close to \$100,000 per day. Before the term was overtaken by "First Person Shooter" people talked about "doom{}-clone" games.\\
\par

 \rawdrawing{Doom_clone_vs_first_person_shooter}
 \par
There is also tremendous sentimental value. \doom{} is one of these title which made an ever lasting impression upon first contact. Many of us were just in our teen years when this game came out and most are still able to remember in which circumstances they first saw it running. It is an exhilarating feeling to learn the internals of something once deemed "magical".\\
\par


Beyond the nostalgia, and this is the most important reason this book is relevant, the making of \doom{} is the ever repeating story of inventors, engineers and builders gathered around a common vision. There was no clear path between where id Software was and where they wanted to be. Only the certitude that nobody else had gone there before. They worked days and nights, slept on the floors, and waded across rivers to make their dream come true.\\
\par
 The making of \doom{} summarizes well how achieving a colossal task assimilates to do a thousands small things right. This is the story of a bunch of dreamers who combined skills, dedication, and good fortune, together resulting in a breathtaking combination of technology, artwork and design.\\
\par



% You may disagree with the values of these "old" things. Some people prefer to sail with the wind, rarely looking back. But even to them this book could turn out to be a useful engineering map someday.\\
% \par
 To narrate this wonderful adventure, the black book had to respond to two seemingly orthogonal constraints. On one side,  be able to stand alone without need for supplemental information or cross-references. On an other side, to avoid boring faithful readers with content already visited in the series. The middle ground was to allow people who had read about Wolfenstein 3D to get more out of this book but to not make it a necessity.\\
 \par
 Topics which would have been interesting to re-visit such as the architecture of the VGA, DOS's TSR, 386 Real-Mode, PC Speaker sound synthesis, PIC and PIT, DDA algorithms and a few others are mentioned but not extensively described since they were part of \textit{Game Engine Black Book: Wolfenstein 3D}. This trade-off allowed to reach the target which was a book around 400 pages long which can be handled with one hand while sipping on a cup of tea.\\
\par
A few liberties were taken with regard to code samples. Due to the restricted real-estate of the paper version, code sometimes had to be slightly modified to fit in. Other times, in order to introduce complexity progressively and not overwhelm the reader, portion of the code in functions were removed. Rest assured the semantic and spirit remains intact.\\
\par 
This book is the fruit of an exercise inspired by Nicolas Boileau who reportedly stated: "Whatever we well understand, we express clearly". It is also the volume I wish someone else had written so I could just have purchased it (I am quite lazy).\\
\par
I hope you will enjoy reading it!\\
\par
-- Fabien Sanglard (fabiensanglard.net@gmail.com)
