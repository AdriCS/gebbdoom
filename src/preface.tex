This is the second book in the \textit{Game Engine Black Book} series. It picks up right where the first one ended with the release of Wolfenstein 3D in May 1992. It carries on all the way down to December 1993 with the second breakthrough of the 90s in the world of PC gaming, \doom.\\
\par
 Like its predecessor, this volume attempts to describe in great details both the hardware and the software of the era. It opens a window back in time peeking over the engineering used to solve the various problems id Software encountered during the eleven months its took them to ship their new title.\\% which resulted in what is universally considered one of the best game of all times.\\
\par
It may seem odd to write a book about a game twenty-five years after its release. After all, who would be interested in a seemingly outdated technology dedicated to extinct hardware running obsolete operating systems? Given the success of the first volume, it turns out many would. Motivations may differ but whether they are into history, nostalgia, engineering, game industry pivots or even philosophy, its seems there is and edge for everyone.\\ 

\par
\doom has had such a profound and sustained impact that it has become part of modern history. It was an unquestionable milestone which entertained millions and catalyzed vocations. Because the source was available, hackers have learned to program with it. Because it was easy to modify, countless aspiring game developers had their first experience by drawings levels or drawing assets. To this day, because it is such an icon, it is the goto title for hackers willing to demonstrate their skills. From MacBook Touchbar, ATMs, CT scanner, watches and even fridges\footnote{There is even a website "itrunsdoom.tumblr.com" tracking what \doom has been ported to.} pretty much any piece of electronic has run \doom.\\
\par

It was a financial and critical success which reshaped the PC gaming industry. During 1994 it received many awards, including \textit{Game of the Year} by both \textit{PC Gamer} and \textit{Computer Gaming World}, \textit{Award for Technical Excellence} from \textit{PC Magazine}, and \textit{Best Action Adventure Game award} from the \textit{Academy of Interactive Arts \& Sciences}. With more than two millions unit sold and an estimated 20 millions shareware installation, at its highest it generated close to \$100,000 per day. Before the term was overtaken by "First Person Shooter" people talked about "doom-clone" games.\\
 \par
 \rawdrawing{Doom_clone_vs_first_person_shooter}
 \par
There is also tremendous sentimental value. \doom is one of these title which made an ever lasting impression upon first contact. Many of us were just in our teen years when this game came out and most are still able to remember in which circumstances they first saw it running. It is an exhilarating feeling to learn the internals of something once deemed "magical".\\
\par


Beyond the nostalgia, and this is the most important reason this book is relevant, the making of \doom is the ever repeating story of inventors, engineers and builders gathered around a common vision. There was no clear path between where the id Software team was and where they wanted to be. Only the certitude that nobody else had gone there before. They worked days and nights, slept on the floors to make their dream come true. Alike building pyramids or bringing man on the moon, \doom happening summarizes perfectly how achieving one big goal assimilates to do a thousands small things right. This is the story of a bunch of dreamers who had the skills, dedication, and good fortune to take then all the way resulting in a perfect storm of game engine, artwork and design.\\
\par



% You may disagree with the values of these "old" things. Some people prefer to sail with the wind, rarely looking back. But even to them this book could turn out to be a useful engineering map someday.\\
% \par
 To tell this story, the black book had to respond to two seemingly orthogonal constraints. On one side it had to stand alone without need for supplemental information or cross-references. On the other side, it had to avoid boring faithful readers with content already visited in the series. The middle ground was to allow people who had read about Wolfenstein 3D to get more out of this book but to not make it a necessity.\\
 \par
 Topics which would have been interesting to re-visit such as the architecture of the VGA, 386 Real-Mode, PC Speaker sound synthesis, PIC and PIT, DDA algorithms and a few others are mentioned but not extensively described since they were part of \textit{Game Engine Black Book: Wolfenstein 3D}. This trade-off allowed to reach the target which was a book less than 400 pages which can be handled with one hand while sipping tea.\\
\par
I hope you will enjoy it.\\
\par
-- Fabien Sanglard (fabiensanglard.net@gmail.com)