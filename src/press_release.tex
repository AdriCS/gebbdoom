\section{January 1993}
\label{label_pressrelease}
\hrule \par \bigskip
\fixme{Improve justification here. Maybe use monospace space algorithm?}
\begin{verbatim}
Id Software
1515 N. Town East Blvd. #138-297, Mesquite, TX  75150

FOR IMMEDIATE RELEASE
Contact:        Jay Wilbur
FAX:            1-214-686-9288
Email:          jay@idsoftware.com (NeXTMail O.K.)
Anonymous FTP:  ftp.uwp.edu (/pub/msdos/games/id)
CIS:            72600,1333

Id Software to Unleash DOOM on the PC

Revolutionary Programming and Advanced Design Make For Great
Gameplay

DALLAS, Texas, January 1, 1993-Heralding another technical
revolution in PC programming, Id Software's DOOM promises to
push back the boundaries of what was thought possible on a 386sx
or better computer.  The company plans to release DOOM for the
PC in the third quarter of 1993, with versions planned for 
Windows, Windows NT, and a version for the NeXTall to be
released later.

In DOOM, you play one of four off-duty soldiers suddenly thrown
into the middle of an interdimensional war!  Stationed at a
scientific research facility, your days are filled with tedium
and paperwork.  Today is a bit different.  Wave after wave of
demonic creatures are spreading through the base, killing or
possessing everyone in sight.  As you stand knee-deep in the
dead, your duty seems clear-you must eradicate the enemy and
find out where they're coming from.  When you find out the
truth, your sense of reality may be shattered!

The first episode of DOOM will be shareware.  When you register,
you'll receive the next two episodes, which feature a journey
into another dimension, filled to its hellish horizon with fire
and flesh.  Wage war against the infernal onslaught with machine
guns, missile launchers, and mysterious supernatural weapons. 
Decide the fate of two universes as you battle to survive! 
Succeed and you will be humanity's heroes; fail and you will
spell its doom.

The game takes up to four players through a futuristic world,
where they may cooperate or compete to beat the invading
creatures.  It boasts a much more active environment than Id's
previous effort, Wolfenstein 3-D, while retaining the
pulse-pounding action and excitement.  DOOM features a fantastic
fully texture-mapped environment, a host of technical tour de
forces to surprise the eyes, multiple player option, and smooth
gameplay on any 386 or better.

John Carmack, Id's Technical Director, is very excited about
DOOM: Wolfenstein is primitive compared to DOOM.  We're doing
DOOM the right way this time.   I've had some very good insights
and optimizations that will make the DOOM engine perform at a
great frame rate.  The game runs fine on a 386sx, and on a
486/33, we're talking 35 frames per second, fully texture-mapped
at normal detail, for a large area of the screen.  That's the
fastest texture-mapping around-period.

Texture mapping, for those not following the game magazines, is
a technique that allows the program to place fully-drawn art on
the walls of a 3-D maze.  Combined with other techniques,
texture mapping looked realistic enough in Wolfenstein 3-D that
people wrote Id complaining of motion sickness.  In DOOM, the
environment is going to look even more realistic.  Please make
the necessary preparations.

A Convenient DOOM Blurb

DOOM (Requires 386sx, VGA, 2 Meg) Id Software's DOOM is
real-time, three-dimensional, 256-color, fully texture-mapped,
multi-player battle from the safe shores of our universe into
the horrifying depths of the netherworld!    Choose one of four
characters and you're off to war with hideous hellish hulks bent
on chaos and death!  See your friends bite it!  Cause your
friends to bite it!  Bite it yourself!  And if you won't bite
it, there are plenty of demonic denizens to bite it for you!

DOOM-where the sanest place is behind a trigger.


An Overview of DOOM Features:

        Texture-Mapped Environment

DOOM offers the most realistic environment to date on the PC. 
Texture-mapping, the process of rendering fully-drawn art and
scanned textures on the walls, floors, and ceilings of an
environment, makes the world much more real, thus bringing the
player more into the game experience.  Others have attempted
this, but DOOM's texture mapping is fast, accurate, and
seamless.  Texture-mapping the floors and ceilings is a big
improvement over Wolfenstein.  With their new advanced graphic
development techniques, allowing game art to be generated five
times faster, Id brings new meaning to "state-of-the-art".

        Non-Orthogonal Walls

Wolfenstein's walls were always at ninety degrees to each other,
and were always eight feet thick.  DOOM's walls can be at any
angle, and be of any thickness.  Walls can have see-through
areas, change shape, and animate.  This allows more natural
construction of levels.  If you can draw it on paper, you can
see it in the game.

        Light Diminishing/Light Sourcing

Another touch adding realism is light diminishing.  With
distance, your surroundings become enshrouded in darkness.  This
makes areas seem huge and intensifies the experience.  Light
sourcing allows lamps and lights to illuminate hallways,
explosions to light up areas, and strobe lights to briefly
reveal things near them.  These two features will make the game
frighteningly real.

        Variable Height Floors and Ceilings

Floors and ceilings can be of any height, allowing for  stairs,
poles, altars, plus low hallways and high caves-allowing a great
variety for rooms and halls.

        Environment Animation and Morphing

Walls can move and transform in DOOM, which provides an
active-and sometimes actively hostile-environment.  Rooms can
close in on you, ceilings can plunge down to crush you, and so
on.  Nothing is for certain in DOOM.

To this Id has added the ability to have animated messages on
the walls, information terminals, access stations, and more. 
The environment can act on you, and you can act on the
environment.  If you shoot the walls, they get damaged, and stay
damaged.  Not only does this add realism, but provides a crude
method for marking your path, like violent bread crumbs.

        Palette Translation

Each creature and wall has its own palette which is translated
to the game's palette.  By changing palette colors, one can have
monsters of many colors, players with different weapons,
animating lights, infrared sensors that show monsters or hidden
exits, and many other effects, like indicating monster damage.

        Multiple Players

Up to four players can play over a local network, or two players
can play by modem or serial link.  You can see the other player
in the environment, and in certain situations you can switch to
their view.  This feature, added to the 3-D realism, makes DOOM
a very powerful cooperative game and its release a landmark
event in the software industry.

This is the first game to really exploit the power of LANs and
modems to their full potential.  In 1993, we fully expect to be
the number one cause of decreased productivity in businesses
around the world.

        Smooth, Seamless Gameplay

The environment in DOOM is frightening, but the player can be at
ease when playing.  Much effort has been spent on the
development end to provide the smoothest control on the user
end.  And the frame rate (the rate at which the screen is
updated) is high, so you move smoothly from room to room,
turning and acting as you wish, unhampered by the slow jerky
motion of most 3-D games.  On a 386sx, the game runs well, and
on a 486/33, the normal mode frame rate is faster than movies or
television.  This allows for the most important and enjoyable
aspect of gameplay-immersion.

        An Open Game

When our last hit, WOLFENSTEIN 3D was released the public
responded with an almost immediate  deluge of home-brewed
utilities; map editors, sound editors,  trainers, etc.  All
without any help on file formats or game layout from Id
Software.  DOOM will be release as an OPEN GAME.   We will
provide file formats and technical notes for anyone who wants
them.   People will be able to easily write and share anything
from their own map editors to  communications and network
drivers.

DOOM will be available in the third quarter of 1993. 
\end{verbatim}
\par \hrule
