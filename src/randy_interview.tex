\qaq{Can you give a little bit of background about yourself, how old you were when you developed \doom{} for SNES, etc.?}
\qaanor{DOOM/FX was published in 1995 and I was 25 y/o. I wrote it from San Diego, California but I'm Canadian, born and raised in Toronto, Ontario.  My first program was published when I was 13. Here is a list of what I have worked on:
	\begin{itemize}
      \item Bubbles -- A Centipede-style game for the Commodore 64.
      \item The 64 Emulator -- An Amiga program that emulated the Commodore 64.
      \item Dragon's Lair -- Another Amiga program that was the first full-screen full-color animated game on any home computer -- Dragon's Lair was also the first game which streamed data in real-time from six floppy discs during gameplay!
      \item DOOM -- A version of the popular "2-$\nicefrac{1}{2}$D" FPS for Super Nintendo running on an original game engine designed for the SuperFX RISC processor (aka "GSU2A") designed by Argonaut Software in the UK. DOOM for SNES was one of the very few titles which worked with most of the SNES hardware accessories including the mouse and light gun. It also supported the XBAND hardware modem and two players could complete head-to-head in one of the truly rare online games for SNES. I built a custom development system (Assembler, Linker, Debugger, etc.) which used a modified (ie. hacked apart) StarFox game cartridge to provide the CPU and communication to the Amiga-based toolchain by using a serial-based interface which plugged into both joystick ports on the SNES. Support for Nintendo's official hardware development system for the GSU2A was added later in the development because no games had been released at that point which could be "modified" for use (StarFox used the original GSU which ran at half the clock rate of the GSU2A and had less memory among other hardware changes and enhancements.)
      \item "Bleem!" was a PC-based emulator for Playstation games -- the program was entirely in x86. I reverse-engineered Playstation software and hardware, wrote a run-time optimizing recompiler for the game logic and added PC enhancements like higher resolution and 3D graphics card support.
      \item "Bleem!" for Dreamcast was a Playstation emulator written entirely in SH4 for the Sega Dreamcast. I was awarded four patents as a sole inventor for the technologies and techniques I developed for bleem! and bleem! for Dreamcast.
      \item Cyboid is a high-speed 3D FPS (first-person shooter) which features single- and two-player split-screen gaming and eight-player online multiplayer across a variety of play modes -- think "Quake" and that's pretty much it. The game features VR support, In-App purchasing, Achievements and Leaderboards, Advertising for multiple mediators and Multiplayer Internet gaming for up to eight players online -- all of which runs on multiple platforms from Google and Amazon. The game is written in ARM assembly, native C/C++ and Java and the entire package (code, data, graphics, sound and music) is eight megabytes (... smaller than a single graphics texture in games/apps these days!) The custom engines have all been highly optimized to support the widest range of devices possible, including lower-powered hardware (such as the Amazon Fire TV Stick) or older devices running Android API 17 "Jellybean".
     \end{itemize}
     }
 




\qaq{It seems the GSU-2 was not able to render fullscreen. I have read several theories online. Some mention a hardware limitation limiting the processor to 192 lines. Some mention bandwidth issues related to DMA (to read from the GSU RAM). Do you have more insight about this?}
\qaa{As you know, the Super NES ("SNES") was a character-based graphics architecture which used a "font" that defined each character's image and "map" which specified what character to display in each position on the screen.\\
\par
 In effect, the the "Super/FX" (aka GSU "Graphics Support Unit") enabled bitmap emulation on the SNES using a combination of fast, custom hardware and a carefully designed instruction set that was optimized for graphics processing.\\
\par
 The GSU was a truly incredible RISC custom chip designed by Jez San and Argonaut Software -- It is one of my favorite hardware architectures of all time for many reasons, in particular, the elegance and simplicity of the opcodes were both powerful and efficient and the entire design is very similar to the architecture of the ARM processor (which is one of my other favorite hardware systems!).\\
\par
 BTW, I agree 100000\% with Jez's comments about the GSU -- absolutely true!!!\\
\par
 The original GSU, used in StarFox, ran at 10.74 Mhz, the GSU/2A used in DOOM/FX ran at 21.48Mhz.\\
\par
 All standard ALU operations were supported (add, subtract, eor, etc.), plus a variety of fast multiplication operations, multiple memory load/store operations (based on whether you were accessing ROM or RAM), a "LOOP" command and the "PLOT" command.  Code could run from ROM, RAM or an on-board 512 byte cache (which DOOM/FX used extensively!)\\
\par
 What made the GSU really special was the large register set (16 general purpose registers compared to the 65816) plus the opcode prefixes "FROM/TO/WITH" which allowed you to specify a source register ("FROM") and destination register ("TO") or operate on the same register using "WITH" -- This type of operation is very standard these days, particularly on the ARM architecture, but back then it was both unique and powerful!\\
\par
 Bandwidth was never an issue for me -- here are two examples:
\begin{enumerate}
\item       The "PLOT" command writes to the emulated bitmap using "X" and "Y" coordinates and a "COLOUR" and can update single pixels at a time.  I wrote code which timed how fast the plot command executed, and discovered that since the underlying memory is still effectively a character-based array, each time you "plotted" to a new character "line" (8-pixels wide), the hardware fetched that character's memory (basically eight bytes that form the 256 color "plane") into an internal cache so that subsequent plots to the same character "line" executed much faster.


I used this knowledge in DOOM/FX by "pre-writing" multiple pixels the first time a "new line" was written -- in effect, instead of plotting a single pixel at a time, I wrote multiple pixels using the same color so the hardware quickly updated the internal "character line" cache and then overwrote the same pixels with the correct individual colors at high speed (typically 1 clock per pixel!)

 

\item       The GSU ran SO much faster than the SNES 65816, the vast majority of the game (close to 95\%) is all GSU code -- the 65816 is basically halted waiting for various interrupts to update memory, swap screens, read the joysticks, etc. -- but otherwise the 65816 is pretty much idle!
\end{enumerate}
 

The GSU had four different rendering heights (128, 160 and 192 pixels) plus an "OBJ" mode for sprites as well as three options for pixel "depth" (4, 16 or 256 colors).\\
\par
 DOOM/FX used the 192 pixel / 256 color mode.\\
\par
 Here's something most people don't know: DOOM/FX was the second-most expensive cartridge to manufacture because it included the GSU/2A, had the largest ROM (2 Megabytes) and RAM configuration available and a custom red plastic cartridge ... the only option missing was the battery backup!}















\qaq{I speculated that you used the Unofficial Doom Specs to extract all the assets, am I correct?}
\qaa{I used a variety of online resources to reverse-engineer the DOOM data formats -- without the tremendous amount of work from all those other people, DOOM/FX wouldn't have been possible in the amount of time!}

\qaq{Can you describe the Reality engine in its big lines. I assume you used the BSP but did you also create the equivalent of visplanes?}
\qaa{The Reality Engine is basically a highly-custom 2-$\nicefrac{1}{2}$D game engine that's almost all written in GSU code. All the code is designed to run in small blocks so they fit into the GSU's internal cache which runs at high-speed. The architecture is similar to DOOM in that it uses a 2D BSP, sectors, segments, etc. -- there are a number of optimizations that minimize the processing required to figure out what to draw, etc.}

\qaq{Did you get any help from id Software at all?}
\qaa{I demo'd the game to Sculptured Software when I had a fully operational prototype running -- graphics, texturing, movement, enemies, etc. -- it was obviously "DOOM" to anyone looking at what I had running on the hardware.\\
\par
Id Software was shown the game a few weeks later and even though it was an early version, there wasn't much left for them to help with since it was much further along than a typical prototype or "proof of concept."\\
\par
Of course, there was still a lot of work to do before the game could be published ... for example: sound, music, the rest of the levels, testing, etc. -- and it was with the help of multiple people at Sculptured Software that the game was finally done.}

\qaq{How long did it take you to build the toolchain, then how long to code the game? }
\qaa{IIRC, the game was developed in roughly eight months, give or take. There were multiple toolchains used to create the game, graphics, sound and data...\\
\par
The code was developed using a custom development system I wrote that included an assembler, linker, source-level debugger, etc. -- I had written multiple projects with the development system (NES, SNES among others) and it was built over a number of years.  Adding support for the SuperFX only took a couple weeks, though. I also wrote a number of tools and utilities to extract and convert the original assets from the PC game into the optimized formats for the Reality Engine. Sound and Music used a toolchain from Sculptured Software so all I had to do was convert the extracted assets into the appropriate formats for processing.}

\qaq{How did you get in touch with the publisher? Did you try several publishers before Sculptured Software accepted it?}
\qaa{Sculptured Software was the only publisher I considered -- I worked at Sculptured for a few years and knew many people there -- all of whom were awesome!}

\qaq{Was it difficult to comply with Nintendo "no violence, no blood" policy at the time?}
\qaa{ Nintendo was incredibly easy to work with -- they had very few requirements, minimal changes and it was an easy submission process in general.}

\qaq{Anything you want to mention, looking back at this game/part of your life? (Some people lament that programming is more complex nowadays but i feel it was harder back then, just to even get the programming manuals).}
\qaa{  IMO, programming these days is very different than the "old days...\\
\par
 It was so much easier to write something for a single piece of hardware like the SNES -- you could spend time focusing on the game and improving performance, etc.\\
\par
The same was true of the Commodore 64, Amiga and Sega Dreamcast -- each of which had unique technical aspects and advances which provided opportunities to make something stand out in ways that is much harder to do these days...\\
\par
In comparison, writing an Android app is a huge amount of work for many reasons, many of which are out of the developers' control...\\
\par
 There's a lot of hardware out there (with many differences in what is/isn't supported as far as graphics, sound, input, etc.), there are multiple versions of the OS (each of which have their own "quirks", limitations, differences and changes) and although it's truly wonderful that Google provides comprehensive components like Firebase and Google Play, the frequent updates, changes and differences require a huge amount of time and overhead just to keep your app "current"\\
These days it's fairly painless and even easy to create a simple "app" that works on a bunch of Android devices ... but writing something which takes advantage of the unique aspects of as many different devices as possible takes a lot of time, resources and effort -- and I think that's why there are so many similar games, apps and titles out there -- sure, there's uniqueness in many of them, but far less than there used to be -- we don't see things like "Lode Runner", "Archon" or "Sword of Sodan" anymore.\\
\par
A friend recently recommended a FPS that was over 1Gig to download! ... and then the game required a constant internet connection for updates, resources and even to play the game... we've sure come a long way from Dragon's Lair on the Amiga which was an 8K program and 5 Megs of data.}



\b{More about how the GSU-2 worked:}\\
\qaa{Basically the GSU generates a bitmap into it's own RAM that is then transferred to the SNES PPU for display. I think there wasn't sufficient clock cycles and RAM to generate a full screen worth of data and transfer it to the PPU in time to avoid tearing/glitching, etc. The GSU RAM can be accessed by only one device at a time (either the GSU or the SNES), but not both ... so you have to generate data into the GSU RAM, then transfer it to the SNES-side PPU ("Picture  Processing Unit") so it can be displayed on screen. During the transfer, the SNES-side has access to the RAM and the GSU does not, so you have to wait until the transfer is complete before you can use the GSU to generate more data. The GSU can still do many other things without accessing the internal RAM, though -- and DOOM/FX used that feature extensively. BTW, the same is true of the ROM -- only one device can access the ROM at a time -- either the GSU or the SNES.}

\b{Anecdotes from DOOM SNES development:}
\qaanor{
\begin{enumerate}
\item   At one time a developer at Sculptured had modified a bunch of the basic levels -- added some really nice changes like object placement, etc. -- and after showing the new levels to id Software, they   basically wanted the game to be as true to the original as possible... so we ended up reverting all the changes and keeping the original levels with as few modifications as possible.

        One thing that id Software let me keep was the auto-rotating overhead map.

\item     The engine had a number of configurable options -- many of which weren't used in the final game ... some examples:
\begin{enumerate}
        \item      The walls could be drawn single-pixel and not doubled, but there weren't enough clock cycles for a reasonable framerate
        \item      Floors could also be drawn -- as single or double pixels -- but there wasn't enough ROM to store the textures and it was much faster to use colours/dithering instead.
\end{enumerate}
\item    The timing and logic to generate the display was fairly complicated... as you'll see:

        I basically generate the display in "thirds" due to the limited amount of RAM available in the GSU and PPU.

        The "PPU" is the SNES-side "Pixel Processing Unit" -- it's basically the portion of the hardware which fetches memory and displays it on screen.

        Each "third" is basically generated in three "steps":
        \begin{enumerate}
                \item Calculate all the necessary data/etc.
                \item Generate the graphic output into GSU RAM
                \item Transfer the GSU RAM to PPU.
        \end{enumerate}
        There isn't enough PPU memory to store two complete game frames (one for the current game frame and a second for the next game frame) ...

        To solve this, I divided the SNES-side PPU memory into five "portions", three of which are required to display a single frame (obviously) since the PPU has to be able to access that data to show it on        screen.

        The remaining two "portions" are where I store the first and second "thirds" of the next game frame... but what about the last third?

        The final third of the next game frame is more complicated because it will be transferred overtop of one of the "active" thirds which the PPU is still using to show the current game frame!

        When the GSU has finished generating the last third, I wait for a raster interrupt that ensures the PPU is NOT showing the area which is used by the "common" third and transfer the last portion overtop before the raster hits that area of the screen.

        After the final transfer is complete, I reconfigure which of the five portions is used by the PPU to display the new game frame and start the whole process again, generating the next game frame into the      two "portions" that are now available and updating the final third in the same way as before. 

        Of course, if the game was running at 60fps, I could just generate everything on-the-fly and wouldn't need all the complicated code!

        Sound was also a bit tricky, for similar reasons ... but not nearly as complex.  There wasn't enough RAM for all the sound effects, so for example, the weapon sound is transferred dynamically when you        switch weapons -- there's enough time to transfer the sound data for the weapon while the "weapon change" animation is running, but there's only one weapon sound in the SPU (sound processing unit)    at a time.

\item     Most of the graphics (textures, objects, etc.) are stored compressed internally and decompressed while drawing.

        I used a simple encoding which reduced the memory requirements and since the GSU was fast enough, the tradeoff between processing cycles and storage was a good one.
\end{enumerate}

        }
