\vspace{-10pt}
\section{Network}
The early 90s predated the democratization of the Internet and Wifi by a decade. Connecting computers together was difficult and expensive. Even if you had the means, bandwidth and latency were abysmal. Most of the time, playing with friends meant getting all computers into the same room. Playing from the comfort of your room was extremely uncommon. Amusingly, an unconnected computer is now deemed useless. Communication with other machines is something natural and the bare minimum for a machine to be useful.\\
\par
 But back in the early 90s, to pack your 50 pounds machine (including the CRT) on your bike, making it to a friend's place alive, plug the cables, start \doom{} and finally see your character move on the other computer screen was an indescribable feeling. To witness machines actually communicating felt unreal and almost magical.\\
\par 
To achieve the impossible, players had three technology available: NullModem cable, Modem, and LAN via Network cards.









\subsection{Null Modem Cable}
The cheapest way and what most people used was the \$20 cable known as "Null Modem" which was directly plugged in each PCs COM port. The cable offered no modulation at all (hence the name). For obvious reasons, only two players could participate.\\
\par 
\cscaledimage{0.5}{nullmodemcable.png}{NullModem cable}
\par
 Two players may sounds lame by today standards but back then it was so new and cool that it felt like the most amazing thing in the world.







\subsection{BNC 10Base2 LAN (Local Area Network)}
To play with more than one opponent was substantially more difficult. Besides the relatively easy financial burden to pay for the equipment, you had to overcome the much more difficult task to convince a parent to let four teenagers come to their house where they would scream all night. The famous saying: "Shame on me if you fool me once, shame on you if you fool me twice" is rumored to have originated from betrayed mothers and fathers whom had been \textit{doomed} all night.\\
\par



\begin{wrapfigure}[7]{r}{0.25\textwidth}
\centering
\scaledimage{.25}{BNC_T-piece.png}
\end{wrapfigure}

Leaving creative ways to ask for forgiveness aside, on the technical side a player had to plug a 10Base2 network card via the ISA bus. \\
\par The card had a BNC connector upon which was to be plugged a T-shaped connector. Each PC node was connected to up to two other nodes via a 10Base2 coaxial cables. There was no central point in this type of networking, all machines involved in the networked formed a chain. At both ends the chain had to be plugged a signal terminator in order to prevent an RF signal from being reflected back from each end, causing interference, or power loss.


\begin{wrapfigure}[11]{r}{0.5\textwidth}
\centering
\scaledimage{.5}{BNC_connector_50_ohm_male}
\end{wrapfigure}
The coaxial cable were bulky and so were the connectors. Connecting an end to a T-Shape connector was fully achieved with a cool quarter turn of the coupling nut.\\
\par
 Once physically connected, there was no configuration required (IPX is a network level protocol like Ethernet). The Network card MAC address were enough to run the IPX protocol which most games used.\\
 \par
\drawing{10base5bncConnector}{10-Base-2 BNS based network.}
\par
In figure \ref{10base5bncConnector} the four elements of a 1994 LAN. \circled{1} the T-Shaped connector connecting two \circled{2} coaxial cable, forming a link. Each ends of the chain must be closed with two load terminators \circled{3}. The network card connects to both the PC via its ISA bus extension slots (\circled{4}) and the LAN T-base slots.\\
\par
\trivia{Adding new machine on the network meant either unplugging one of the T-shaped connector or unplugging a chain terminator. In both cases, the central bus was broken and all other machines lost connectivity. Everybody remember this one friend who was always late to the LAN and forcing everybody to disconnect so he could join. The bandwidth was shared and the theoretical 10Mb/s was often closer to 5Mb/s. This does not account for friends who wanted to exchange 30MiB song in \cw{.wav} format (there was no mp3 at the time).}\\
\par
\trivia{Really really fancy people could use 10baseT network which requires a "hub" central device resulting in a star shaped network.}









\subsection{Modem}
The most fortunate players were able to afford the luxury of networking from home. That was very expensive since they not only had to pay for a modem and the service of an Internet provider monthly but they also had to pay for the time spent online.\\
\par 
Before broadband, modem used phone landlines to connect to the Internet provider. Which means nobody could use the telephone while the connection was active. Anybody picking up the phone in the house created enough disturbance to kill the connection.\\
\par
AOL (America OnLine) offered a package of five hours for \$9.95 with each extra hours billed \$3.50 per unit. A player averaging 2h/day could be billed $9.95 + 55 * 3.5 = \$202 $ for a month\footnote{Adjusted to inflation: \$352 in 2018.}. Add the price of a fast 9600 bauds model for \$399\footnote{Adjusted to inflation: \$696 in 2018.} and you get a good idea of how much it cost for be able to post text messages on a BBS (Bulletin board system).\\
\par
\cscaledimage{0.9}{robotic28-8.png}{US Robotic 28.8 bauds modem. The top of the line in 1994.}

 While establishing the initial handshake the modem speaker was kept open. An attuned ear could easily recognize the different phase where V.X bis transaction, speed negotiation, echo canceller disabling, and modulation modes selection together making the unforgeable melody of a deathmatch in the making.\\
 \par 
\cfullimage{spectrogram2.png}{18 seconds spectrogram of a V.34 handshake\protect\footnotemark }
\par
  \footnotetext{Source: "The sound of the dialup, pictured" by Oona R\"{a}is\"{a}nen.}
 
 \begin{figure}[H]
\centering  
\begin{tabularx}{\textwidth}{ R{0.1} L{1.9} }
  \toprule
  \textbf{Stage} &  \textbf{Description} \\
  \toprule 
   
   1 & Modem goes off hook.\\
   2 & Telephone exchange sends a dial tone.\\
   3 & DTMF: Model dials 1-(570)-234-0001 a number in Pennsylvania, USA.\\
   4 & Answering modem initiates a V.8 bis transaction.\\
   5 & Answering modem asks caller for a list of its capabilities.\\
   6 & Caller responds to V8 bis initiation, agrees to list its capabilities and request to escape from telephony into information transfer mode..\\
   7 & FSK Data @ 300 bps: I'm capable of full V8. I can transmit ACK. My country is US and I was made by Net2phone Inc.\\
   8 & FSK Data @ 300 bps: Why don't we use V8 then.\\
   9 & Ok, mode acknowledged. Terminating V.8 bis transaction.\\
   \toprule 
   A & Answering modem disable echo suppressors and cancellers in PSTN.\\
   B & FSK Data @ 300 bps: Repeated 6x Here are my modulation modes: V.34, V.32, v.23 duplex ...\\
   C & FSK Data @ 300 bps: Repeated 3x: I can do any of those. \\
   D & Both modems send a wide-spectrum probing signal in both directions to do measurements on the line.\\
   E & Both modems go to scrambled data \\
   \toprule
\end{tabularx}
\caption{}
\end{figure}
\par




All along the 90s the bandwidth kept on steadily improving. Upon \doom{} release most modem were capable of 28.8 kbit/s. Those who downloaded the shareware version in December 1993 had to wait 20 minutes to collect the (2,166,955 bytes) of the zip archive.\\
\par

 \begin{figure}[H]
\centering  
\begin{tabularx}{\textwidth}{ L{0.2} L{0.3} L{0.5}}
  \toprule
  \textbf{Year} & \textbf{Version} & \textbf{Bandwidth} \\
  \toprule 
   
    1990 & V.32 & 9.6 kbit/s \\
    1991 & V.32bis &  14.4 kbit/s \\
    1994 & V.34 & 28.8 kbit/s \\
    1995 & V.34 & 33.6 kbit/s \\
    1996 & V.90 & 56.0/33.6 kbit/s\\
    1999 & V.92 & 56.0/48.0 kbit/s\\
   
   \toprule
\end{tabularx}
\caption{Modem speeds over the 90s decade.\protect\footnotemark}
\end{figure}
\footnotetext{Bits rate increased at the expense of latency. A 9600 baud modem played \doom{} better than the default configurations of 56kbit modems. Quake needed more bandwidth than \doom{}'s controller replication, so it became a different tradeoff.}


\par
On top of the V.XX hardware communication layer, modems were driven using Hayes commands. Notice how command \cw{ATDT} was translated to DTMF in the previous spectrogram.\\% in figure \ref{spectrogram2.png} on page \pageref{spectrogram2.png}.\\

\par
 \begin{figure}[H]
\centering  
\begin{tabularx}{\textwidth}{ L{0.2} L{0.15} L{0.65}}
  \toprule
  \textbf{Modem A} & \textbf{Modem B} & \textbf{Comments} \\
  \toprule 
   
    ATDT15551234 &	&	Modem A issues a dial command: AT-Get the modem's ATtention; D-Dial; T-Touch-Tone; 15551234-Call this number\\
    \toprule 
      & RING	& Modem A begins dialing. Modem B's phone-line rings, and the modem reports the fact.\\
      \toprule 
    & ATA	& Modem B issues answer command.\\
    \toprule 
    CONNECT	& CONNECT	& The modems connect, and both modems report "connect"..\\
    abcdef	& abcdef	& When the modems are connected, any characters typed at either side will appear on the other side.\\
    \toprule 
    & +++	& Modem B issues the modem escape command.\\
    \toprule 
     OK &	& The modem acknowledges it.\\
    \toprule 
    & ATH	& Modem B issues a hang up command.\\
    \toprule 
    NO CARRIER &	OK	& Both modems report that the connection has ended. Modem B responds "OK" as the expected result of the command; modem A says NO CARRIER to report that the remote side interrupted the connection.\\
   \toprule
\end{tabularx}
\caption{AT layer dialog between caller and callee.}
\end{figure}
\par
\trivia{The fragility of these connection led to humorous way to end a message. People would finish a forum post with "Hey! Wait! Don't pick up the ph\{\#`\$\{\%\&`+'\%NO CARRIER".}
