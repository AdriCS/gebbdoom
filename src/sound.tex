\section{Sound System}
Thanks to its aggresive marketing, superior technology, and cheaper cards, Creative Labs dominated the market of sound cards. In order to hope surviving the market, any new cards had to label itself "Creative Labs compatible". Up to that point, sound cards had combined FM synthetized used for music and a DSP to playback digitized sounds in 44Khz,8bits per sample, stereo quality.\\
\par
 The early 90s would be the theatre of the last wave of innovation for PC audio gaming and saw the extinction of a previously key manufacturer named AdLib. For \doom, two cards constituted the top of the line. The Sound Blaster 16 by Creative Labs and the Gravis Ultrasound by Advanced Gravis Computer Technology Ltd\footnote{An other Canadian company!}.\\
\par
\subsection{Sound Blaster 16}
 In June 1992, with the release of the Sound Blaster 16, Creative Labs solved the problem of audio for games on PC forever with a card capable of CD quality playback, 44Khz, 16-bit sample on stereo.\\
\par
\fullimage{SoundBlaster_16_ASP_CT1740.png}
\par
Solving the audio problem turned out to be a problem itself for Creative Labs. Innovation for audio professional were not enough to drive high volumes. A few attempts to innovate for games with ASP and EAX technologies were made but there were of little appeal to consumers ears. As audio chips became cheaper to make and with the technical requirements stagnating, manufacturers started to provide audio capability built-in the motherboard.\\
\par
For a time the only devices able to provide an extra IDE connectors, audio cards survived by being bundled with CD-ROMs. That was not enough to save them. Within ten years the market of audio cards dissapeared.\\
\par
\drawing{sb16}{}
\par









\subsection{Gravis UltraSound}
Gravis Computer Technology originally built what was universally accepted as the best PC joypad. Strong with the cashflow they decided to enter the sound card market with an audatious and innovative card. The Gravis Ultrasound (nicknamed GUS) was released in 1992.\\
\par
The card offered Sound Blaster 16 compatible music and digitized sound playback. However the way it achieved it was line no other company had done before. On a SoundBlaster 16, music was synthetized with a FM chip which did its best to eumlate instruments via sin waves. Gravis offered an "all digitized" road where even music could be played with digitized.\\
\par
\fullimage{Gravis_UltraSound_PnP_Pro_V1.png}
\par
Components list here\\
here\\
and here\\
\drawing{gravis}{}
\par
wavetable synthesis, with 256 kB of sample RAM (upgradeable to 1MB). It could play up to 32 channels of hardware audio at 19 kHz or 14 channels at 44.1 kHz. This made it quite popular with PC-based music composers and DEMO SCENE\\
GF1 chip\\
Came with 12 MiB of sample (mostly percussion instruments).\\
You really have to hear both FM synthesis and Wavetable version of "At Doom's Gate", guitar is amazing.\\
Bad timing: RAM Expensive.\\