\section{Sound System}
PC were equipped with a "PC speaker", a device able to produce monotonic and annoying "beeps". The intent was to help diagnose system health at startup (one short beep meant the system was okay). But serious gamers always invested in a sound card. Thanks to its aggressive marketing, superior technology, and cheaper cards, Creative Labs dominated the market. In order to survive, any newcomer had to label itself "SoundBlaster-compatible". The unofficial standard meant OPL2-based FM synthesizer capability for music and a DSP able to play back digitized sounds at 22Khz, 8-bit per sample in stereo.\\
\par
 The early 90s were the theater of the last wave of innovation for gaming audio and saw the extinction of a previously key manufacturer named AdLib\footnote{Which ironically had established the OPL2 chipset necessity.}. Two cards nonetheless managed to bring something new to the table. They were the Sound Blaster 16 by Creative Labs and the Gravis Ultrasound by Advanced Gravis Computer Technology Ltd.\\

\vspace{-2mm}
\subsection{Sound Blaster 16}
 In June 1992, with the release of the Sound Blaster 16, Creative Labs solved the problem of PC audio for gaming forever with a card capable of CD quality playback -- 44Khz 16-bit stereo samples. It was an instant hit and immensely successful with customers.\\
\par
\fullimage{SoundBlaster_16_ASP_CT1740.png}
\par
Solving the audio problem turned out to be a problem of its own for Creative Labs' business. Even though they subsequently managed to release new products, those were mainly of interest to audio professionals. A few attempts to innovate with ASP and EAX technologies were made but consumers remained deaf to the melody of these improvements. As audio chips became cheaper and with technical requirements stagnating, manufacturers started to provide audio capability built in on motherboards.\\
\par
For a short time the extra Panasonic/Matsushita connectors permitting connection of a CD-ROM allowed sound cards to survive, in bundle products. But that was not enough to save them. Within ten years the market for sound cards disappeared.\\
\par
\rawdrawing{sb16}
\par
Above, a Sound Blaster 16 model CT1740 from 1994. \circled{1} Panasonic/Matsushita connector (for CD-ROM), \circled{2} C1741 DSP Chip, \circled{3} C1748 ASP chip, \circled{4} CT1746B Bus Interface, \circled{5}, 46.61512 Mhz Timer, \circled{6} CT1745A Mixer, \circled{7} Expansion Interface for wavetable-capable WaveBlaster daughtercard, \circled{8} mic-in, line-in, volume wheel, line-out, joystick port.







\subsection{Gravis UltraSound}
Gravis Computer Technology originally built what was universally accepted as the best PC joypad, the Gravis PC GamePad. With strong cash flow they decided to enter the sound card market with an audacious and innovative card. The Gravis Ultrasound (nicknamed GUS) was released in 1992.\\
\par
The GUS offered Sound Blaster 16-compatible music and digitized sound playback. On top of that the card had a capability like no other on the market. It was able to play back music not with FM synthesis but with digitized instrument samples. The technology, named "wavetable synthesis", achieved an audio quality far superior to its competitors.


\fullimage{Gravis_UltraSound_PnP_Pro_V1.png}\\
\par
The Gravis UltraSound Pro, \circled{1} 2 SIMM slots allowing up to 8 MiB RAM, \circled{2} IDE Connector and \circled{3} ATAPI connector (both for CD-ROM), \circled{4} IW78C21M1 chip (1 MiB Flash ROM), \circled{5} HM514260ALJ7 70ns DRAM, \circled{6} Main CPU InterWave AM78C201KC and \circled{7} from top to bottom: mic-in, line-in, line-out, joystick port.\\
\rawdrawing{gravis}

The concept was aggressive and so was the hardware that came out of the Gravis factories. The red resin they used made their card unmistakably recognizable.\\
\par
The cost of this technology was twofold. First, the card needed audio samples. This problem was "solved" via a Gravis driver that installed more than 12 MiB of sound samples\footnote{That was enormous at the time, when the full version of \doom{} was 12 MiB as a matter of comparison.}. Second, the card had to be able to access the sample at runtime, meaning it had to have its own RAM. Since samples take up more space than sine equations, the original GUS shipped with 256 KiB, upgradeable to 1 MiB.\\
\par
It rapidly developed a cult following among demo-makers who loved the high music quality it could achieve. For the gamer market however, things were more complicated. The GUS's GF1 main chip had difficulties emulating the OPL2 and setup was complicated (a mediocre TSR emulator had to be loaded manually by the user). The GUS also suffered from an unfortunate release date concurrent to the RAM shortage of 1993/1994. Players were reluctant to fork over the \$169 it cost, being \$40 more than a Sound Blaster 16. Initially selling well, sales slowed around 1995 and it was discontinued in 1996.\\
\par
 id Software was one of the few companies to support the Gravis UltraSound. \doom{} included a mapping file that translated MIDI instrument IDs to Gravis \cw{.PAT} instrument files\footnote{It also controls which samples get loaded into RAM at various card configurations.}. Listening to the electric guitars and drums of "At Doom's Gate" from \doom{}'s OST makes the SoundBlaster version pale in comparison. But all success stories must have the right timing and sadly the GUS was ahead of its time.\vspace{-5pt}

 \subsection{Roland}
 It is impossible to close the audio section without mentioning Roland's hardware. Established in 1972, Roland Corporation not only manufactured equipment for audio playback, it also provided the best hardware to author and record music. The real breakthrough for DOS gaming was the Roland SC-55 (a.k.a SoundCanvas) released in 1991. Not only was it the very first General MIDI standard device (which defined 128 instruments that every device following the standard could adopt), it synthesized music using Roland's proprietary combination of prerecorded samples and subtractive synthesis which was far superior to Yamaha's OPL.\\
 \par
 \fullimage{roland_cr55.png}\\
 \par

Roland's equipment was built entirely around the MIDI protocol which was carried via cables employing a special 5-pin circular connector.\\
\par
 The precursor of the SC-55, the MT-32 synthesizer, could be connected to a PC via an MPU-401 ISA MIDI adapter card. There was also a combo LAPC-I card which combined both the adapter and an MT-32 successor, the CM-32L, inside a single ISA card.\\
\par 
\cfullimage{roland_lapc1.png}{Roland LAPC-I}
\par
Roland also released the SCC-1 which combined the SC-55 and an MPU-401 onto a single ISA card.\\
\par 
\cfullimage{roland_scc1.png}{Roland SCC-1}
\par
\pagebreak
For recording, an artist connected a musical keyboard to a "note recording program" called a MIDI sequencer. Once captured on the computer, the MIDI-based music could be tweaked and edited like any media.\\

\par
For playback, things were a tiny bit more complicated. When Sierra On-Line pioneered support for Roland sound cards in 1988, the games leveraged the hardware to play beautiful music. The audio effects were either done via the PC Speaker or later using General MIDI stock audio effects\footnote{Another World in 1991 used the stock audio effects.}. As games became increasingly elaborate with PCM digitized effects (which Roland cards could not play), gamers faced a dilemma where the best music needed a Roland but the best sound effects needed either a SoundBlaster or a GUS. The expensive (\$499 in 1991) solution was to buy both cards and mix the streams externally.\\
\par
\rawdrawing{ultimate_audio}

