\section{Sound System}
Thanks to its aggressive marketing, superior technology, and cheaper cards, Creative Labs dominated the market of sound cards. In order to hope surviving the market, any newcomer had to label itself "Creative Labs compatible". The unofficial standard meant OPL2 based FM synthesize capability for music and a DSP able to playback digitized sounds at 44Khz, 8-bit per sample in stereo.\\
\par
 The early 90s were the theater of the last wave of innovation for PC audio gaming and saw the extinction of a previously key manufacturer named AdLib\footnote{Which ironically had established the OPL2. chipset necessity.}. Two cards nonetheless managed to bring something new and were favorably met by customers. They were the Sound Blaster 16 by Creative Labs and the Gravis Ultrasound by Advanced Gravis Computer Technology Ltd\footnote{An other Canadian company!}.\\
\par
\subsection{Sound Blaster 16}
 In June 1992, with the release of the Sound Blaster 16, Creative Labs solved the problem of audio for games on PC forever with a card capable of CD quality playback, 44Khz, 16-bit sample on stereo. It was an instant hit and immensely successful on the market.\\
\par
\fullimage{SoundBlaster_16_ASP_CT1740.png}
\par
Solving the audio problem turned out to be a problem in itself for Creative Labs business. Even though they subsequently managed to release new products, those mainly interested audio professional. A few attempts to innovate for games with ASP, and EAX technologies were made but consumers remained deaf to the melody of these improvements. As audio chips became cheaper to make and with the technical requirements stagnating, manufacturers started to provide audio capability built-in the motherboard.\\
\par
For a short time the extra IDE connectors allowing connection of a CD-ROMs allowed them to survive in bundle offers. But that was not enough to save them. Within ten years the market of audio cards disappeared.\\
\par
\rawdrawing{sb16}
\par
Above, a Sound Blaster 16 model CT1740 from 1994. \circled{1} IDE connector (for CD-ROM), \circled{2} C1741 DSP Chip, \circled{3} C1748 ASP chip, \circled{4} CT1746B Bus Interface, \circled{5}, 46.61512 Mhz Timer, \circled{6} CT1745A Mixer, \circled{7} MIDI daughtercards Interface, and \circled{8} from top to bottom: mic-in, line-in, volume controller, line-out, joystick port.







\subsection{Gravis UltraSound}
Gravis Computer Technology originally built what was universally accepted as the best PC joypad, the Gravis PC GamePad. Strong with the cash flow they decided to enter the sound card market with an audacious and innovative card. The Gravis Ultrasound (nicknamed GUS) was released in 1992.\\
\par
The GUS offered Sound Blaster 16 compatible music and digitized sound playback. On top of that the card had a capability like no others one the market. It was able to play music, not with FM synthesize but with digitized instrument samples. The technology named "Wavetable Synthesis" achieved an audio quality far superior to its competitors.


\fullimage{Gravis_UltraSound_PnP_Pro_V1.png}\\
\par
The Gravis UltraSound Pro, \circled{1} 2 SIMM slot allowing up to 8 MiB RAM, \circled{2} IDE Connector and \circled{3} ATAPI connector (both for CD-ROM), \circled{4} IW78C21M1 chip (1 MiB Flash ROM), \circled{5} HM514260ALJ7 70ns DRAM, \circled{6} Main CPU InterWave AM78C201KC and \circled{7} from top to bottom: mic-in, line-in, line-out, joystick port.\\
\rawdrawing{gravis}
\par
The concept was aggressive and so were the hardware that came out of Gravis factories. The red resin used made their card unmistakably recognizable.\\
\par
The cost of this technology was twofold. First the card needed audio samples. This problem was "solved" via Gravis driver which installed more than 12 MiB of sound samples\footnote{That was enormous at the time, the full version of \doom was 12 MiB as a matter of comparison.}. Second, the card had to be able to access the sample at runtime which mean its had to have its own RAM. Since samples are much more voluminous than sin equations, the original GUS shipped with 256 KiB and could be boosted to 1 MiB.\\
\par
It rapidly developed a cult following among demo-makers who loved the high music quality it could achieve. On the lambda customer side however, things were more complicated. The GF1 CPU had difficulties emulating the OPL2 and setup was complicated (a huge TSR emulator had to be loaded manually by the user). The GUS also suffered of an unfortunate release date concurrent to the RAM shortage of 1993/1994. Players were reluctant to fork the \$169 it cost which was \$40 more than a Sound Blaster 16. With sales plummeting, Gravis almost went bankrupt.\\
\par
 \doom was one of the few title to support the Gravis Ultra Sound. It provided a \cw{.PAT} file which was a simple mapping of 256 values to translate midi ID instruments to Gravis ID instruments. Listening to the electric guitars and drums of "At Doom's Gate" from Doom soundtrack on a GUS is quite an experience which make the FM version pale in comparison.