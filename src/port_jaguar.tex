\begin{wrapfigure}[5]{r}{0.35\textwidth}{
\centering \scaledimage{0.35}{jaguar_logo.png}}
\end{wrapfigure}
Development of the machine started in 1990 when Atari commissioned Cambridge based Flare Technology to design not only one but two new game system simultaneously. A fourth generation, 32-bit, system called Panther and an audacious 64-bit system called Jaguar\footnote{Atari named all its console after big cats. Besides the Jaguar and Panther, its handheld game system from 1989 was called "Lynx".} were to be developed.\\
\par Three years later, with the Jaguar project ahead of schedule, Atari decided to abandon the Panther and released the 64-bit Jaguar on November 1993.\\
\par
\fullimage{consoles/Jaguar.png}
\par
Martin Brennan, Ben Cheese, and John Mathieson from Flare Technology made opinionated design decisions. Besides the 18 buttons joysticks, the machine had no less than five processors to juggle with.\\
\par
On the audio side, a 32-bit 27MHz RISC nicknamed "Jerry". On the video side, no less than three processors were all contained in a 32-bit 27MHz RISC chip nicknamed "Tom" with a GPU, a Blitter, and an Object processor. To orchestrate everything 2 MiB of RAM and a 16/32-bit 13Mhz 68000. To connect them all, to the delight of marketing, a 64-bit data bus.\\
\par
The bold ad "Do the Math!" and its featured 64-bit claim triggered mostly suspicion from potential customers.  No matter how John Mathieson attempted to explain it in interviews the machine felt like a misleading attempt from an already suspicious Atari marketing department. How Atari could have managed to manufacture something four times better than its Nintendo and Sega's 16-bit Super Nintendo and Genesis was not clear.
\drawing{jaguar_arch}{Architecture of the Jaguar. Notice the uneven buses.}
\par
Cautious purchasers on one side were met by incredulous game developers on the other side. The five processor architecture was powerful but highly unusual for people used to deal with one processor on the 8-bit Nintendo Entertainment System or Sega Master System. Programming the Jaguar was an art that few took the time to learn.\\
\par
The low volume of games available at launch prevented the formation of a critical mass of customers. Low sales made developers less likely to invest in the Jaguar which in turn impacted sales. Over it's three-year lifetime, Atari sold about 100,000 units. 


\fullimage{jagua_motherboard.png}\\
\par
Inside the machine: \circled{1} TOM, \circled{2} 2MiB RAM, \circled{3} JERRY, \circled{4} Motorola 68000, \circled{5} Operating System ROM, \circled{6} Cartridge slot, \circled{7} Joystick ports, \circled{8} AC Adapter Jack, \circled{9} DSP Port, \circled{A} Monitor (composite, component, and S-Video) Port, and \circled{B} Channel switch, TV port.\\
\par
\fq{Jaguar has a 64-bit memory interface to get a high bandwidth out of cheap DRAM. ... Where the system needs to be 64 bit then it is 64 bit, so the Object Processor, which takes data from DRAM and builds the display is 64 bit; and the blitter, which does all the 3D rendering, screen clearing, and pixel shuffling, is 64 bit. Where the system does not need to be 64 bit, it isn't. There is no point in a 64-bit address space in a games console! 3D calculations and audio processing do not generally use 64-bit numbers, so there would be no advantage to 64-bit processors for this.}{John Mathieson}



\rawdrawing{jaguar_motherboard}
\par
John Mathieson granted many interviews giving more insight on the type of constraints a hardware engineer has to deal with. From cost related pressure to plain edit from Atari to use CPU from Motorola. Grass on the hardware side is not greener than on the software side.\\
\par
\fq{Atari were keen to use a 68K family device, and we looked closely at various members.  We did actually build a couple of 68030 versions of the early beta developers systems, and for a while were going to use a 68020. However, this turned out too expensive.  We also considered the possibility of no [Motorola 680x0 chip] at all.  I always felt it was important to have some normal processor, to give developers a warm feeling when they start. The 68K is inexpensive and does that job well.}{John Mathieson}







\subsection{Programming The Jaguar}
To unleash the beast meant to have all five processors work in parallel\footnote{Source: "Jaguar Technical Reference Manual: Tom \& Jerry".}. It was complicated in theory and it was complicated in practice.\\
\par
\fq{The 68000 may be the CPU in
   the sense that it's the center of operation, and boot-straps the machine,
   and starts everything else going; however, it is not the center of Jaguar's
   power. ... The 68000 is like a manager who does no real work, but tells
   everybody else what to do.\\
   \par
     I maintain that it's only there to read the joysticks.}{John Mathieson}

\subsubsection{Theory}
\par
The Motorola 68000 is used as a manager. It deals with the outside world and manages resources for the other processors. It is at the highest control level and has complete control over the system.\\
\par
The Object Processor is connected to the TV and is in charge of generating display lines. It reads an object list usually made of pixels (which can overlap). It performs all the functions of a traditional sprite engine. Its 16-bit per pixel CRY (Cyan-Red-Yellow) color model was unconventional for consoles at that time. One byte is an (\cw{X,Y}) coordinate in a sRGB cube flattened in a square which gives a color. The other byte give the brightness for a total of 65536 colors.\\
\par
\fullimage{CRY.png}\\
\par

\fullimage{CRY_persp.png}
\par
The Graphic Processor is the powerhouse with a fast instruction throughput, a powerful ALU with a
parallel multiplier, a barrel-shifter, and a divide unit, in addition to the normal arithmetic functions. Its internal 4KiB of SRAM is meant to store not only data to operate on but also its local program instructions.\\
\par
The Blitter performs fast RAM block move and fill operations. It can generate strips of pixels for Gouraud shaded Z-buffered polygons and is also capable of skipping pixels (based on Z-testing). It is capable of rotating bit-maps, line-drawing, character-painting, and a range of other effects. It is in charge of loading the SRAM for Tom \& Jerry with local data and instructions\\
\par
Jerry the DSP is the twin brother of the Graphic Processor. Its bigger local SRAM (8 KiB) and its smaller 32-bit connection to the 64-bit main bus make it a perfect candidate for generating music and sound effects. However if the programmer decided so it could perform any other task, even graphic ones. The versatility is such that the Jad-Link to connect two Jaguar together is plugged in the "DSP port" on the back of the console and Jerry is in charge of networking.\\
\par
All RAM (even SRAM in the RISC CPUs) is addressable by any components thanks to a flexible memory controller. At any point during execution of their program, any processor could become the DMA bus master. Despite its status as governor, the 68000 has the lowest level of priority (were its DMA master request was to conflict with an other processor) while the DSP is king in order to avoid extremely unpleasant audio glitches.
\par



\subsubsection{Practice}
The hardware had several bugs, especially at the Memory Controller level, making multitasking hard to rely on and even harder to debug.







It may not be obvious at first sight but the Motorola 68000 and Tom/Jerry had different architectures and different instruction sets. The intended workflow was to program the Motorola in C while the GPU/DSP RISC path was more convoluted. The programmer had to first learn the new instruction set, then writing assembly code, use the provided assembler to generate machine code and finally write a full pipeline to store and later deliver the machine code to Tom \& Jerry local program SRAM.\\
\par











\subsection{Doom On Jaguar}
John Carmack took an early interest in the Jaguar and did the port himself with Dave Taylor taking care of the multi-player code. It was not his first contact with the machine.\\
\par
\fq{I converted Wolfenstein 3D on a whim. I was thinking about how the Jag's hardware could be applied to games other than Doom, and Wolfenstein seemed a pretty good utilization. I started programming one afternoon and 15 CDs later, when the other guys were coming in the next morning, I had a functional port of the SNES Wolfenstein code running. We sent it to Atari, and they gave us the go-ahead to stall Doom for a little while and get Wolfenstein out read quick.}{John Carmack for EDGE Magazine, June 1994}\\
\par
For \doom, the duo had something up and running two weeks after they signed on the port even thought it was running at a wretched rate\footnote{Source: EDGE Magazine, June 1994.}.\\
\par
\trivia{To ease the task of generating RISC instructions for Tom \& Jerry, John Carmack wrote his own \cw{lcc} compiler backend. Output was optimized further by hand.\footnote{It was not the last time id Software would use the excellent \cw{lcc}. in 1999, it was used for Quake 3 to generate the VMs bytecode.}}\\
\par
To run on a machine with 50\% less RAM was difficult. Cyberdemon and Spiderdemon were cut. Sprites and textures resolution was lowered. Maps were heavily modified to use fewer textures, have fewer segments, and create fewer visplanes. Take a look at E1M1 in Figure \ref{doom_jaguar3.png} and compare it with the PC version (on page \pageref{mashed_potatoes1.png}). See how  the floor blue texture is gone and there is only one step instead of two.\\
\par
The 3D renderer had to be rewritten to work in small chunk of asm to run on the RISCs. Nine overlays were swapped in and out of the SRAM at runtime based on engine "phases".\\
\par
\cfullimage{doom_jaguar3.png}{}
\par
\vspace{-12pt}
The source code of \doom{} for Jaguar was released in May, 2003 by Carl Forhan of Songbird Productions. Peeking inside reveals the details of how memory footprint was reduced. Visplanes storage size for example, was reduce from 128 to 64.\\
\par
\ccode{jaguar_visplanes.c}
\par
A big decision was to cut music playback during game. This was due to the lack of performance of Tom which was not able to handle the game engine on its own. To resolve this problem\footnote{The game ended up managing a solid 20 framerate per second.}, Jerry (the so called DSP) was used to run collision detections. Thankfully, there was still enough juice in Jerry to play sound effects and the engine managed a solid 20 fps.\\
\par
The 3D rendering was done at a resolution of 160x180. The vertical lines were doubled to reach 320x180 with 60 pixels at the bottom for the status bar, bringing the overall resolution to 320x240. In many occurrences the graphic result was better than PC. With its tailor-made 16-bit CRY mode, the Jaguar really had 65,536 colors with no color banding in sight.




\fullimage{doom_jaguar2.png}
\par
Further inspection shows the Jaguar code also contains a few artifacts from the Sega X32 version (\cw{MARS} was the code name of the Sega Genesis X32). We can also learn there that development was done on NextSTEP which was called "Simulator" mode.\\
\par
\ccode{jaguar_src.c}
\par

\fq{The memory, bus, blitter and video processor were 64 bits wide, but the
processors (68k and two custom risc processors) were 32 bit.\\
\par
The blitter could do basic texture mapping of horizontal and vertical spans,
but because there wasn't any caching involved, every pixel caused two
ram page misses and only used 1/4 of the 64-bit bus. Two 64-bit buffers
would have easily trippled texture mapping performance. Unfortunate.\\
\par
It could make better use of the 64-bit bus with Z buffered, shaded triangles,
but that didn't make for compelling games.\\
\par
It offered a useful color space option that allowed you to do lighting effects
based on a single channel, instead of RGB.\\
\par
The video compositing engine was the most innovative part of the console.
All of the characters in Wolf3D were done with just the back end
scalar instead of blitting. Still, the experience with the limitations and
hard failure cases of that gave me good ammunition to rail against Microsoft's
(thankfully aborted) talisman project.\\
\par
The little risc engined were decent processors. I was surprised that they
didn't use off the shelf designs, but they basically worked ok. They had
some design hazards (write after write) that didn't get fixed, but the only
thing truly wrong with them was that they had scratchpad memory instead
of caches, and couldn't execute code from main memory. I had
to chunk the DOOM renderer into nine sequentially loaded overlays to
get it working (with hindsight, I would have done it differently in about
three...).\\
\par
The 68k was slow. This was the primary problem of the system. You options
were either taking it easy, running everything on the 68k, and going
slow, or sweating over lots of overlayed parallel asm chunks to make
something go fast on the risc processors.\\
\par
That is why playstation kicked so much ass for development - it was programmed
like a single serial processor with a single fast accelerator.
If the jaguar had dumped the 68k and offered a dynamic cache on the
risc processors and had a tiny bit of buffering on the blitter, it could have put up a reasonable fight against sony.} {John Carmack}





