\begin{wrapfigure}[3]{r}{0.4\textwidth}{
\centering \scaledimage{0.4}{snes_logo.png}}
\end{wrapfigure}
The Super Nintendo Entertainment System was released in 1990 in Japan and the following years in USA and Europe. It was the 16-bit replacement to the 8-bit NES. In Japan, the Super Fami-Com ("FAMIly COMputer") was an instant success with the initial shipment of 300,000 units sold out within hours. The frenzy was such that the government requested Nintendo to release its future systems on weekends to avoid further disturbances.\\
\par
Nintendo had established a merciless system to ensure quality of its games. Publisher were only allowed five games per year. To make sure this rule was enforced, only Nintendo was allowed to produce cartridges. Publisher had to buy them from Nintendo. To make sure everybody played by the rules (and also protect games from being copied) the SNES looked for a CIC lockout chip before a game was allowed to start. It was a powerful mechanism which was only cracked as the SNES had reached its end of life.\\
\par
During its nine years lifespan\footnote{The Super Nintendo was discontinued in 1999.} 721 games were published among them several critical and commercial success such as Super Mario World, Zelda III, Mario Kart, Z-Zero, Super Metroid, or Donkey Kong Country. Having sold close to 50 million units it is arguably one of the best console of all times\footnote{Source: "The SNES is the greatest console of all time" by Don Reisinger} both in terms of sales and catalog.\\
\par
\cfullimage{consoles/SNES.png}{The Super Famicom (a.k.a SNES, Super Nintendo) by Nintendo.}
\par
From a technical standing point, the SNES excelled at 2D. Its 16-bit 65C816 3.58 MHz CPU had 128 KiB RAM available. It piloted a PPU Picture Processing Unit with 64 KiB of RAM to manipulate large sprites using up to 256 colors at a resolution of 256x240. On the audio side a powerful combo made of an 8-bit Sony SPC700, and a 16-bit DSP with 64 KiB of dedicated SRAM.\\
\par

Despite it impressive 2D sprite engine and especially its "Mode 7" capability, the machine struggled with computationally intensive operations such as 3D calculations. Nintendo was vividly award that 3D would be the next big thing in gaming but was struggling to make it happen. As fate would have it, a small UK firm would hold the solution to the problem.\\

\subsubsection{Argonaut Games}
Back in 1982, Jez San was a lone game developer working exclusively on C64, Atari ST, and Amiga computers. To sell his creations he needed a company. Seeing a similarity between his name (J.San) and the mythological story of Jason and the Argonauts, he named it Argonaut Games, Plc.\\
\par
His venture did not remain a single man project very long. By 1990 he had gathered talents in London offices and developed an interest in Nintendo's 1989 handset, the Game Boy. The team had managed two feats most deemed impossible. They had a 3D wire-frame engine and they had cracked CIC protection to install in on the Game Boy.\\
\par
\fq{They had the Nintendo logo drop down from the top of the screen, and when it hit the middle the boot loader would check to see if it was in the right place.\\
\par
The game would only start if the word was correctly in place in the ROM. If anyone wanted to produce a game without Nintendo's permission, they would be claiming to use the word 'Nintendo' without a licensed trademark, and therefore Nintendo would be in a position to sue them for trademark infringement. We figured out that with just a resistor and capacitor - around 1 cent's worth of components - we could find out how to beat the protection. The system read the word 'Nintendo' twice - once to print it on the screen at the boot up, and a second time to check if it was correct before starting the game cartridge. That was a fatal mistake, because the first time they read 'Nintendo' we got it to return 'Argonaut', so that was what dropped down the screen. On the second check, our resistor and capacitor powered up so the correct word 'Nintendo' was in there, and the game booted up perfectly.}{Jez San\protect\footnotemark}

\footnotetext{Source: Jez San interview with Damien McFerran for "Born slippy: the making of Star Fox" article.}







At CES '90, his demo of the engine on a hacked cartridge to Nintendo booth was relied all the way up to Kyoto headquarters. Unknown to Jez at the time, his timing could not have been better. Back in Japan, Nintendo was working on the Super Famicom titles which were to showcase its superior technology upon launch. Super Mario World was in its infancy but the flight simulator, "Pilot Wings", was a bit more advanced.\\ 
\par
\begin{wrapfigure}[14]{r}{0.5\textwidth}{
\centering \scaledimage{0.5}{pilotwings.png}}
\end{wrapfigure}
The SNES PPU's Mode 7 capable of rotating and projecting huge sprites was cleverly used to simulate Pilot Wings terrain. The planes however were still 2D flat hand drawn sprites. That bugged producer Shigeru Miyamoto immensely since it prevented the camera from rotating around the planes.\\
\par
At the time, it was not in Nintendo's habit to deal with outsides or even foreigners. However this time they made an exception and flew Jez and Dylan Cuthbert who did the 3D work to their headquarters in Kyoto.\\
\par
 The young pair\footnote{Jez was 23 and Dylan 18.} of two met with all Nintendo VPs: Miyamoto, Gunpei Yokoi, Takehiro Izushi, Yasuhiro Minagawa, Genyo Takeda. They were shown everything, from the secret SNES to the secret Mario/Pilot Wings. And then they were asked if there was a way to draw the planes as full faced polygon objects.\\
\par
\fq{I told them that this is as good as it's going to get unless they let us design some hardware to make the SNES better at 3D. Amazingly, even though I had never done any hardware before, they said YES, and gave me a million bucks to make it happen.}{Jez San}.\\
\par

Boldly promised a "10x" performances increase by Jez, Nintendo embraced the offer to get special hardware designed for their game. "Pilot Wings" would ship with sprite planes to be able to be released at the same time as the Super Famicom but the "Super FX" chip as it would be marketed later was to be used for an other project Nintendo had in sleeves.\\
\par
The name was "StarFox".

\subsubsection{StarFox}
The agreement was such that Nintendo would make all game design decision while financing Argonaut Games so it would produce not only the hardware but also the 3D engine for the title. Jez San lost no time hiring or contracting the best UK talents he knew.\\
\par
For the hardware he contacted Flare Technology (the same people who designed the Atari Jaguar). Ben Cheese, Rob Macaulay, and James Hakewill project was codenamed Mathematical Argonaut Rotation I/O, or "MARIO". What they ended up designing was so powerful they jokingly labeled the Super NES "just a box to hold the chip". Since there was no way to modify the console, the chip was soldered on each new game cartridge which increased MSRP significantly.\\
\par
\fq{We designed the Super FX chip in a way no one had designed hardware before - we built the software first, and designed our own instruction set to run our software as optimally as possible. No one did it that way around! Instead of designing a 3D chip, we actually designed a full RISC microprocessor that had math and pixel rendering functions, and the rest was run in software. It was the world's first Graphics Processing Unit, and we have the patents to prove it.}{Jez San}\\
\par
For the engine, Carl Graham and Pete Warnes worked in London headquarters while Dylan Cuthbert, Krister Wombell, Giles Goddard (plus later Colin Reed) permanently relocated to Kyoto in Nintendo's offices to work in close collaboration with Miyamoto's team.\\ 
\par
The project resulted in a critical, commercial, and engineering success. StarFox shipped on February 21, 1993 and went on to sale four million copies worldwide\\
 \par

The rest of this idyllic story between the two companies is bitter. A sequel to their megahit, Star Fox 2 was completed by Argonauts and set to release in 1996 when Nintendo canceled it abruptly, fearing the impact over the launch of the Nintendo64. Argonauts Games was unpleased and relationships with Nintendo soured. Nintendo subsequently poached Dylan, Giles and Krister. Dylan Cuthbert would have joined too but a non-compet clause in his contract prevented him to. He quit his position at Argonauts to join Sony Computer to work on the Playstation.\\
\par
The divorce was finalized when Nintendo refused to let Argonauts use Yoshi for a platform game they were planing for the PS1 \footnote{They ended up replacing Yoshi with a crocodile in "Croc: Legend of the Gobbos".}. Nintendo later released Mario 64 with mechanism seemingly inspired off "Croc" ... and even beat it to market by around a year.





\fullimage{snes_cartridge.png}
\par
The GSU\footnote{Graphics Support Unit} as its technical name would be\footnote{I agree. These are a lot of names for a chip.} had a simple design based on a 16-bit RISC processor running at 10.74 with a 512 byte i-cache. It had its own instruction set to excel at math function and its own framebuffer to excel at pixel plotting. Its mode of operation was to render to the frame buffer where the data would be periodically transfered to the SNES RAM via DMA. It was reportedly capable of rendering 76,458 Polygons/s.\\

\par
The first generation of GSU-1 powered five games: Dirt Racer, Dirt Trax FX, Star Fox, Stunt Race FX, and Vortex.\\
\par
 The second generation, GSU-2, was the same processor running at 21.4 Mhz and extra pins soldered to the bus to increase the size of supported ROM and framebuffer. It was used in three games Doom, Super Mario World 2: Yoshi's Island, and Winter Gold\footnote{Star Fox 2 was finished but canceled at the last minute to not amped the imminent release of the Nintendo 64.}.\\
\par
To open a \doom game cartridge reveals all components previously discussed. \circled{1} The 16-bit GSU-2, \circled{2} 512 KiB framebuffer where the GSU writes, \circled{3} 2MiB ROM where code and assets are stored, \circled{4} Hex inverter, and \circled{5} Copy protection CIC chip.







\rawdrawing{snes_board}
\fq{The 'ten times' figure was a complete over-promise on my part. We didn't really know if that was even possible.\\ 
\par
But it allowed us to over-promise and yet also over-deliver. Instead of achieving just 10x the 3D graphics performance, we actually made things about 40x times faster. In some areas - like 3D math - it was more like a 100x faster. It was not only capable of 3D math and vector graphics, but it was also able to do sprite rotation and scaling - something that Nintendo really wanted for their own games, like Super Mario World 2: Yoshi's Island.}{}\\
\par

\trivia{Some passionate fans have managed to collected all 791 games of the SNES catalog. Seeing them on a shelve is impressive. You can usually spot \doom cartridge from 20 feet away. Only three games were ever allowed to not be made of the standard gray. Two were red: "DooM" and "Maximum Carnage" while "Killer Instinct" was black.}
\pagebreak


\drawing{snes_cartridge}{SNES 721 games library. Zelda stands apart. Because Zelda stands apart.}
\par
\rawdrawing{snes_cartridge2}
















\subsection{Doom On Super Nintendo}

Doom on SNES happened thanks to the genius and determination of a single man: Randy Linden. The man had an admiration for the game and decided to port it to a mass market machine so more players could enjoy it. Randy never had access to the source code or the assets from neither the PC nor the console version. He started from nothing.\\
\par
To retrieve the assets, he was able to leverage the "Unofficial Doom Specs" by Matthew Fell which explained the \cw{.wad} lumps layout in details. The sprites, textures, music, sound effects and maps were extracted from \cw{DOOM.WAD}. The engine was an entirely different story.\\
\par

\fq{DOOM was a truly ground-breaking title and I wanted to make it possible for gamers without a PC to play the game, too. DOOM on the Super Nintendo was another one of those programming challenges that I knew could be accomplished.\\
\par

I started the project independently and demo'd it to Sculptured Software when I had a fully operational prototype running. A bunch of people at Sculptured helped complete the game so it could be released in time for the holidays.\\
\par
The development was challenging for a few reasons, notably there were no development systems for the SuperFX chip at the time. I wrote a complete set of tools -- assembler, linker and debugger -- before I could even start on the game itself.\\
\par
The development hardware was a hacked-up StarFox cartridge (because it included the SuperFX chip) and a modified pair of game controllers that were plugged into both SNES ports and connected to the Amiga's parallel port. A serial protocol was used to communicate between the two for downloading code, setting breakpoints, inspecting memory, etc.\\
\par
I wish there could have been more levels but unfortunately the game used the largest capacity ROM available and filled it almost completely. I vaguely recall there were roughly 16 bytes free, so there wasn't any more space available anyway! However, I did manage to include support for the SuperScope, Mouse and XBand modem! -- Yes, you could actually play against someone online!}
{Randy Linden (Interview with gamingreinvented.com)}\\
\par
\trivia{Randy Linden later worked on an even more impressive reverse engineering magic. In 1999 he was co-author of a commercial PlayStation emulator called "bleem!". A gutsy move since the console was still being produced and marketed by Sony at the time.}\\
\par


% \begin{wrapfigure}[11]{r}{0.25\textwidth}
% \centering
% \scaledimage{0.25}{superfx.png}
% \end{wrapfigure}
What is remarkable with this version is how Randy decided to cut different corners than the other console ports for his engine.\\
\par
\cfullimage{consoles/snes/snes_e1m1.png}{}
\par
The starting screen of E1M1 (figure \ref{consoles/snes/snes_e1m1.png}) is a good summary. Despite having only 600 KiB RAM, see how the blue flooring was not removed (althought this is a plain color and not a texture now). Likewise regarding the geometry where all steps are visible (compare with the PC version page \pageref{mashed_potatoes1.png} and Jaguar version page \pageref{doom_jaguar3.png}).\\
\par The Reality engine as Randy name it, was able to deal with PC map geometry but must have had issues with fillrate or texture sampling because ceiling and floor textures were removed altogether.













\cfullimage{consoles/snes/wide.png}{E1M1 exterior toxic pond}
\par
In the screenshot above we can see the window is actually not fullscreen. This was not an issue exclusive to \doom since StarFox, StarFox2 and all games using a SuperFX had to reduce the active area size. It was likely due to the SNES limited bandwidth which did not permit a DMA transfer for a fullscreen\footnote{Although some like \cw{anthrofox.org} theorized that the SuperFX cannot render more than 192 lines.}.\\
\par
Out of the native 256 x 224 resolution, only 216x176 were actually drawn and only 216x144 for the 3D canvas (32 pixels tall for the status bar). With vertical lines duplicated, Reality Engine was actually rendering at 108x144. Even at this low resolution, the average framerate was around 10fps which was a remarkable achievement. The "low" framerate was not enough to discourage players from enjoying \doom, according to Randy Linden the game sold very well.







\cfullimage{consoles/snes/enemy.png}{E1M2 had all its stairs steps.}
\par
Amazingly, Reality was able to implement diminished lighting for the walls, as seen in figure \ref{consoles/snes/enemy.png} (walls only, flats are always solid colors).\\
\par
 On the list of features sacrificed to the holy RAM, sprite resolution was lowered significantly to the point they are sometimes hard to recognize (which contrast with the player weapon rendered at a higher resolution), enemy poses are all remove except for the one facing the player, monster infighting was also removed, there is no sound propagation (monsters are only awoken by visual contact), most SFX were cut and all monsters sound like imps.
\\
\par
Nintendo forbid blood\footnote{Wolfenstein 3D enemies had to be replaced with zombies and dogs by rats.} but a lot remained. There are no blood splashes and corpse don't explode but enemies are bloodies as they die. Pieces of flesh lying around levels.	


   % Super FX2:
   % Released 1993\\
   % GSU: 23 MHz\\
   % VRAM: 32 KiB\\
   
   % SNES:\\
   % Released 1990\\
   % CPU: 5A22 3.58 MHz\\
   % Audio: SPC700\\
