\section{Video Manager}
Before continuing our trip through the code and looking at the rendering in  \cw{D\_Display}, let's take a few pages to study the graphics stack where each frame is stored and manipulated. The video system is located in the core. It maintains and exposes two data structures: a set of five framebuffers, and one dirtybox (used as a "dirty rectanges"). All write operations update the framebuffers and the box.\\
\par
\cccode{kernel_renderer.c}{Five framebuffers.}
\par
\par
\cccode{kernel_renderer_dbox.c}{The dirtybox.}
\par
\vspace{-10pt}
The video system implementation must provide four functions to the core. \cw{I\_InitGraphics} is to be told when to initialize itself. \cw{I\_UpdateNoBlit} is to be called when a portion of the framebuffer has been modified. \cw{I\_FinishUpdate} is called when the framebuffer is fully composed and should be presented to the screen. \cw{I\_WaitVBL} blocks and returns on next V-Sync.\\ 
 \begin{figure}[H]
\centering  
\begin{tabularx}{\textwidth}{ L{0.27} | L{0.73} }
  \specialrule{1pt}{0pt}{0pt}
  \textbf{Method} & \textbf{DOS Implementation} \\
  \specialrule{1pt}{0pt}{0pt}
\cw{I\_InitGraphics} & Set VGA in Mode-Y (320x200 256 colors stretched to 4:3).\\
\cw{I\_UpdateNoBlit} & Send updated area of the screen to the VGA hardware.\\
\cw{I\_FinishUpdate} & Flip buffer (update CRTC start scan address).\\
\cw{I\_WaitVBL} & Wait for V-Sync (Used to wait before updating palette).\\
   \specialrule{1pt}{0pt}{0pt}
\end{tabularx}
%\caption{Video system interface}
\end{figure}
\par
To host the framebuffer in the core and not in the video system itself was an audacious trade-off: it was a performance hit since data had to be copied twice before reaching the screen. But it tremendously improved portability since the framebuffer had been abstracted. It also opened the door to reading back from the core framebuffer. This capability allowed new effects such as the "predator" transparency seen with the "spectre" demon.\\
\par
 % The performance hit was not too back on VLB systems. If you refer back to the architecture on page \pageref{vlbarchitecture}, you will see that it allowed transfer of one framebuffer to the video system while the next frame was being generated in the cachelines. This parallel pipeline allowed the framerate to be 90\% of what it would have been with direct access to the VGA.\\
 \trivia{Because \doom{} renders a full screen on every frame, there is no need to "clear" the framebuffer like Wolfenstein 3D did. If we were to interrupt the engine while rendering a frame to peek inside framebuffer \#0, it would look like a composition of the previous frame and the unfinished new one.}\\
\par
\vspace{5pt}
\drawing{renderer_full_pipeline}{}
\pagebreak
%\par
%\trivia{Notice how the VGA framebuffer are not exactyl 320x200=0x3E80 in size but 0x4000 instead. This is done because updating the CRTC start address can be don atomically.}
\par
The life of a frame during the 3D phase is as follows:
\begin{enumerate}
\item All write operations are done in framebuffer \#0. Each time the engine updates this buffer, it also updates the \cw{dirtybox} to mark which area has been touched.
\item After a major write sequence, the engine calls \cw{I\_UpdateNoBlit} which triggers the video system to read framebuffer \#0, optimizing for the smallest data transfer possible via the \cw{dirtybox}. On DOS the VGA subsystem copies content to the VGA hardware in one of its three VRAM framebuffers. Each VGA buffer has its own dirtybox \cw{updatebox} to speed up blitting but this remains an expensive operation.

\item Once a frame has been completed, the engine calls \cw{I\_FinishUpdate}  which signals the video system that framebuffer \#0 will no longer be updated. On DOS, the implementation of \cw{I\_FinishUpdate} simply updates the start address of the CRTC which is a lightweight operation.
\end{enumerate}
\par
\ccode{I_FinishUpdate.c}\\
%\par
%On \NeXT system, the video implementation is very different. Calls to \cw{I\_UpdateNoBlit} are ignored. Only when \cw{I\_FinishUpdate} is called the framebuffer \#0 is copied and sent to Display PostScript subsystem. The \NeXT video system is detailed in the Annex on page \pageref{label_next_video_system}.\\
\par
The four other framebuffers in the core are used intermittently for temporary storage.
\begin{itemize} 
\item Framebuffer \#1 is used when taking a screenshot but also to store the background when the 3D view is not full screen (\cw{r\_FillBackScreen}).

\item Framebuffers \#2 and \#3 are used during the wipe animation to store the start and end screens while compositing into framebuffer \#1 (see detailed wipe system in the appendix on page \pageref{label_melt}). 
\item Framebuffer \#4 is smaller and only stores a virgin status bar used when content cannot be delta updated and a full redraw is needed. 
\end{itemize}
\par



The content of function \cw{I\_UpdateBox} to transfer data from RAM to VRAM may look surprising. After all the hardware discussion about 32-bit CPUs and the 32-bit VL-Bus it is uncanny to see the transfer loop do it 16 bits at a time (\cw{short *dest;}). It turns out this was a deliberate choice.\\
\par
\fq{Our artwork was done in 8x8 blocks, or "ebes" as Tom called them.  A 32-bit loop would have needed more code to handle widths that were an odd number of ebes.  It might have been a speedup, but the bus and video card would have to have handled full 32-bit writes, and I don't recall that being common back then.  Many VL-Bus cards were still the same basic chipset used on the ISA cards, and often still 16 bit, which meant that a 32-bit write just took 2x as long.}{John Carmack}\\
\par
\ccode{I_UpdateBox.c}
\par
