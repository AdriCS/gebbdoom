\section{Public Relations}
In an era before Facebook, Instagram, and Twitter, before social media and before a democratized Internet, studio had few ways to get directely in touch with their potential customers. Most of the time, newspapers and magazines relied information which was infrequent, slow, inaccurate and most of time left the reader with more questions than answers.\\
\par
id Software quickly noticed the Unix system that came with their NeXT hardware features a tool called \cw{finger}. Entirely text based, \cw{finger} allowed to remotely explore an UnIX system. A \cw{finger} daemon listened for incoming connections on TCP port 79 ready to response to requests. Running \cw{finger} on a domain name returned a directory of all the accounts registered on that machine.\\
\par
\tcode{finger_idsoftware}
\par
\trivia{When it was invented, the term "finger" had, in the 1970s, a connotation of "is a snitch": this made "finger" a good reminder/mnemonic to the semantic of the UNIX finger command (a client in the protocol context).}\\
\par

Each employee at id Software could create a \cw{.plan} text file located in the home directory of their UNIX account. Anybody blessed with a modem and an Internet connection could consult the content of each \cw{.plan}. This was done by prepending an email address before the domain name. This was the equivalent of what is known today as a blog and allowed direct one way communication with users.\\
\par
When this system was first started, few people were aware of it and even fewer had means to finger id. The system did not have an update feature (to keep \cw{.plan} short, users had a tendency to simply delete the entire \cw{.plan} and write a new one) so most of the early \cw{.plan} have now been lost to history. id Software fans started saving them around 1996. The oldest available on Internet is one of John Carmack's date Feb 18, 1996. The date indicates he was working on Quake at the time.\\
\par
\tcode{finger_johnc.txt}


