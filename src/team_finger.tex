\section{Public Relations}
In an era before Facebook, Instagram, and Twitter, before social media and before a democratized Internet, studios had few ways to get directly in touch with their potential customers. Most of the time, newspapers and magazines relayed information which was infrequent, slow, inaccurate and most of time left the reader with more questions than answers.


id Software quickly noticed the Unix system that came with their NeXT hardware features a tool called \cw{finger}. Entirely text based, \cw{finger} allowed to remotely explore an UNIX system. A \cw{fingerd} daemon listened for incoming connections on TCP port 79, ready to response to requests. Running \cw{finger} on a domain name returned a directory of all the accounts registered on that machine.\\
\par
\tcode{finger_idsoftware}
\par
\trivia{When it was invented, the term "finger" had, in the 70s, a connotation of "is a snitch". This made "finger" a good reminder/mnemonic to the semantic of the UNIX finger command (a client in the protocol context).}\\
\par

Each employee at id Software could create a \cw{.plan} text file located in the home directory of their \NeXT workstation. Anybody blessed with a modem and an Internet connection could consult the content of each \cw{.plan}. This was done by prepending an email address before the domain name. The result was a direct one way connection from developer to consumers and the closest equivalent of what is known today as a blog.\\
\par
When this system was first started few people were aware of it and even fewer had means to finger id. It had no update capability nor notifications. Some like John Carmack updated their \cw{.plan} daily. Originally a bullet points of bug fixes, the \cw{.plan} morphed into blog like content like John Carmack's famous "OpenGL vs Direct 3D"\footnote{See page \pageref{openglvsdirectd}}. The oldest plan known to have been preserved was authored during the development of Quake on Feb 18, 1996.

\tcode{finger_johnc.txt}


