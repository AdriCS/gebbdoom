\section{Waiting for the Dots}
\label{dots_explained}
For anybody who played \doom{} on a PC in 1994, the most frustrating part was waiting for the game to load. One step in particular, the mysterious \cw{R\_Init}, seemed to take forever\footnote{In fact, it took anywhere from 15 to 30 seconds depending on the HDD speed.}. An improvised progress bar made of dots informed the player that the game was loading and to just wait and be patient. Millions of hours were spent watching these tidy dots progress to the right. Probably a few more were spent trying to guess what \cw{R\_Init} actually did in the background.\\
\par
\fakedosoutput{dots.txt}\\
\par
With access to the source code it is possible to modify the engine to output a "label" matching the current phase performed instead of dots. It turns out there are eleven "phases".\\
\par
\fakedosoutput{dots_explained.txt}\\
\par

Phase \cw{OOOOO} corresponds to \cw{R\_InitTextures}. Something that was not mentioned in the 3D renderer section of the book (for the sake of simplicity) is that textures are made of patches. In order to save space, textures reference all patches via an identifier. This is particularly powerful in the case of textures made of a repeating pattern. This phase is where all textures attributes such as dimension and patches definition are loaded to RAM (however the texels remain in the \cw{.WAD}). Because of the volume of data and the amount of \cw{Z\_Malloc} memory allocation, it is the most expensive phase.\\
\par
The first step of this phase is to open a lump named \cw{PNAMES} that contains all the patch names. From the order they appear, a mapping of patch name to ID is established.\\
The second step (which accounts for the bulk of the processing time) is to open lumps \cw{TEXTURE1} and \cw{TEXTURE2}. These contain all the texture entries. Each entry features a name and list of patch IDs, along with patch coordinate and offsets.\\
\par
Phase \cw{1} is just a marker showing when \cw{R\_InitTextures} returns.\\
\par

Phase \cw{2} corresponds to \cw{R\_InitFlats}. It looks for lumps \cw{F\_START} and \cw{F\_END} which are the markers surrounding flat textures. Only the number of flats is retrieved so a proper \cw{malloc} of the flat array can be performed.\\
\par

Phase \cw{333333333333} is similar to phase \cw{0} but this time it involves sprite lumps. Function \cw{R\_InitSpriteLumps} looks up lumps \cw{S\_START} and \cw{S\_END} to find the width and horizontal offset of all sprites in the WAD and saves that data to a sprite array. In the process it prints a dot every 64 sprites. This was the second slowest phase in \cw{R\_Init} (after \cw{R\_InitTextures}) since it had to perform a lot of I/O.\\
\par

Phase five (\cw{4}) is a simple marker to show the end of \cw{R\_InitSpriteLumps}.\\
\par

Phase six (\cw{5}) is also a marker to show the end of \cw{R\_InitData} which encompasses phases \cw{OOOOO}, \cw{1}, \cw{2}, \cw{333333333333}, and \cw{4}.\\
\par

Phase \cw{6} matches function \cw{R\_InitPointToAngle} -- when used -- to build the tangent lookup table. This is now an empty function since it is pre-calculated and stored in \cw{tables.c}\\
\par

Phase \cw{7} matches function \cw{R\_InitTables} which used to build lookup tables \cw{finetangent} and \cw{finesine}. Like the tangent lookup table, these are now pre-calculated in \cw{tables.c} and baked into the executable.\\
\par

Phase \cw{8} matches function \cw{R\_InitPlanes} and does nothing. What a waste of a dot.\\
\par

Phase \cw{9} matches function \cw{R\_InitLightTables} and initializes the zlight table used to implement the lightmaps.\\
\par

Phase \cw{A} matches function \cw{R\_InitSkyMap} which initializes the static \cw{skyflatnum}.\\
% \par
% As you will have guessed, the "dot thermometer" was not very accurate. It started fast with next to instant \cw{OOOOO12}, slowed down to a crawl for \cw{333333333333} which accessed a lot of data, and then sped up again at the end with \cw{456789A} when calling mostly empty functions.\\
\par
\section{Reload Hack}
While developing the game, the long startup time was a problem. Even if a designer made no change to the geometry of the map (and therefore did not need to run \cw{doombsp}) it still took 30 seconds before he could see the result. To avoid breaking the creative flow, the engine was gifted with a "reload hack".\\
\par
By prefixing the path to a WAD with a tilde (\textasciitilde), the engine was set up to reload the WAD on every level start.\\
\par
\fakedosoutput{reload_hack.txt}
\par
This allowed artists and designers to see the results of their work almost instantly.
