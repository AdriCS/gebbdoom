In 1992, id Software was a the top of the PC gaming industry. Wolfenstein 3D had established the First
Person Shooter genre and sales of the sequel "Spear of Destiny" were skyrocketing. The 3D game engine and the associated 
tools they had taken years to develop were no match for the competition. They had an efficient game production pipeline and the talent to use it well. Nobody even came close to challenge them.\\
\par
For their next title, Doom, they could have kept on using the same technology. But id core values were based on novelty and innovation. They had already written a sequel. The Right Thing to do (and the most risky\footnote{Things You Should Never Do (Netscape 6 development), by Joel Spolsky}) was to throw everything away and start their next game from scratch.\\
\par
It was a crazy bet but also a necessity.\\
\par
\fq{Because of the nature of Moore's law, anything that an extremely clever graphics programmer can do at one point can be replicated by a merely competent programmer some number of years later.}{John Carmack}. \\
\par
Even a summary assessment of the consumer hardware landscape showed that the state of things had evolved significantly since their last hit. Intel's latest CPU, the 486 announced in 1989, was finally becoming affordable. More and more customers now declined to go for an "old", twice as slow, Intel 386. Frustrated with the bus bottleneck, manufactured had teamed together to come up with a new standard. PCs often came equipped with a bus four times faster than legacy ISA called VESA Local Bus (VLB). Price of RAM had dropped significantly and the once standard 2 MiB of RAM had doubled to 4 MiB. The audio ecosystem had become even more fragmented with many SoundBlaster "compatible" audio cards on the market and also new players such as Gravis.\\
 \par 
 Not only the hardware had evolved, the software was also different. Better compilers such as Watcom, allowed faster code to be generated. Time for hand-coded assembly was slowly becoming a thing of the past\footnote{Intel would bring that trend back with its superscaler processor, the Pentium and make Quake development ASM intensive but this is an other story altogether}. DOS Extenders allowed to break the machine free from 16 bit programming.\\
 \par
 On the developer side things had evolved too. Powerful workstation were now available and affordable. One manufacturer in particular, founded by Steve Jobs after his departure from Apple, combined strong hardware with efficient development tools. The NeXT machines and their UNIX based operating system looked like solid productivity boosters.\\
 \par
 In this whirlwind of options and possibilities, it would have been easy to go in the wrong direction. Yet, id software seems to have made all the right choices. Thirty years after its release, Doom is still re-released and still considered one of the best game of all time.\\
\par
 How did they manage to start from nothing and make one of the best game of all time within a year? (2 millions sales, 20 milliions sharesware install base). This is the question this book attempts to answer! In order to fully understand the code, one must understand the hardware it runs on. his is why this book first describes the IBM PC and \NeXT workstations architectures. The third chapter introduces id software team and the many tools they wrote. Finally the last chapter dives into DOOM game engine and explore all the secret tricks which helped the game become legend.\\
\par

\begin{figure}[H]
\centering
\scaledimage{1}{ad.png}
\caption{Doom advertisement, circa 1994}
\end{figure}