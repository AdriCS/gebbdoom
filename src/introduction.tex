In 1992, id Software was a the top of the PC gaming industry. Wolfenstein 3D had established the First
Person Shooter genre and sales of the sequel "Spear of Destiny" were skyrocketing. The 3D game engine and the associated 
tools they had taken years to develop were no match for the competition. They had an efficient game production pipeline and the talent to use it well with gorgeous levels and assets. Nobody even came close to challenge them. But for how long? They could have kept milking their technology but the evolution of hardware would have \textit{doomed} them.\\
\par

\par
\fq{Because of the nature of Moore's law, anything that an extremely clever graphics programmer can do at one point can be replicated by a merely competent programmer some number of years later.}{John Carmack}. \\
\par


id Software core value was innovation. They had already written a sequel and it was time to move on. The Right Thing to do (and the most risky\footnote{Things You Should Never Do (Netscape 6 development), by Joel Spolsky}) was to throw away everything they had worked to hard to build and start their next game from a blank sheet. Assets, level, game engine. Everyting.\\
\par
Even a summary assessment of the consumer hardware landscape showed that the state of market had evolved significantly since their last hit. Intel's latest CPU, the 486 announced in 1989, was finally becoming affordable and offered twice the processing power. More and more customers now declined to go for an "old", twice as slow, Intel 386. Frustrated with the bus bottleneck, manufactured had teamed together to come up with a new standard. PCs often came equipped with a bus four times faster than legacy ISA called VESA Local Bus (VLB). Price of RAM had dropped significantly and the once standard 2 MiB of RAM had doubled to 4 MiB. The audio ecosystem had become even more fragmented with many SoundBlaster "compatible" audio cards on the market and also new players such as Gravis.\\
 \par 
 Not only the hardware had evolved, the software was also different. Better compilers such as Watcom, allowed faster code to be generated. Time for hand-coded assembly was slowly becoming a thing of the past\footnote{Intel would bring that trend back with its superscaler processor, the Pentium and make Quake development ASM intensive but this is an other story altogether.}. DOS Extenders allowed to break the machine free from 16 bit programming.\\
 \par
 On the developer side things had evolved too. Powerful workstation were now available and affordable\footnote{Kind of...}. One company in particular, founded by Steve Jobs after his departure from Apple, combined strong hardware with efficient development tools. \NeXT machines and their UNIX based operating system were overlooked yet solid productivity boosters.\\
 \par
 In this whirlwind of options and possibilities, it would have been easy to go in the wrong direction. Yet, id software seems to have made all the right choices. Twenty five years after its release, Doom is still re-released and considered one of the best game of all time.\\
\par
 How did they manage to start from nothing and make one of the best game of all time in just eleven months \footnote{2 millions sales, 20 milliions sharesware install base}?. This is the question this book will attempts to answer.\\
 \par
  In order to do so, in the first two chapters, we will take a close look at the hardware. Not only the IBM PC on which \doom ran but also the \NeXT machines which id Software elected as the cornerstone of its production pipeline.\\
  \par 
  Chapter three takes a close look at the team and the tools they wrote to bridge the hardware and the software.\\
  \par
  And with all these capabilities and restrictions in mind the fourth chapter we will be dedicated to a deep dive in the software which hopefully will help to appreciate why things are designed the way they are.\\
\par
Enjoy.

\begin{figure}[H]
\centering
\scaledimage{1}{doom-shotgun-hallway.png}
%\caption{Fast, Immersive, Dark and Violent.}
\end{figure}
\par
\begin{figure}[H]
\centering
\scaledimage{1}{actionpacked.png}
%\caption{Fast, Immersive, Dark and Violent.}
\end{figure}
