In May of 1992, id Software was among the top of the PC gaming industry. Wolfenstein 3D had established the First
Person Shooter genre and sales of the sequel "Spear of Destiny" were skyrocketing\footnote{By mid-1994 150,000 shareware copies were registered and id sold another 150,000 retail copies as Spear of Destiny; the company estimated that one million shareware copies were distributed worldwide.}. The game engine and the associated 
tools which had taken years to develop were no match for the competition. They had an efficient game production pipeline and the talent to use it well with gorgeous levels and assets. Nobody even came close to challenge them...but for how long? They could have kept milking their technology but the evolution of hardware would have \textit{doomed} them.\\
\par

\par
\fq{Because of the nature of Moore's law, anything that an extremely clever graphics programmer can do at one point can be replicated by a merely competent programmer some number of years later.}{John Carmack}. \\
\par


Competitors were coming with their own games. Notwithstanding others, ever since it foundation around the technological breakthrough named "Adaptive tile refresh", id Software's core value had been innovation. They had already written a sequel to Wolf3D and it was time to move on. The Right Thing to do (and the most risky\footnote{Things You Should Never Do (Netscape 6 development), by Joel Spolsky.}) was to throw away everything they had worked so hard to build and start their next game from a blank sheet. Assets, levels, tools, and game engine, everything would be new and innovative.\\
\par
What would be the hardware target? Even a summary assessment of the consumer landscape showed that the state of the market had evolved significantly since their last hit.
\begin{itemize}
\item Intel's latest CPU, the 486 announced in 1989, was finally becoming affordable and offered twice the processing power. More and more customers now declined to go for an "old", twice as slow, Intel 386. 
\item The advent of Microsoft Windows 3.1 and its hungry GUI had triggered hardware manufacturers to offer more powerful graphic adapters. Rendition still had to be done in software but chipsets were faster and had more capacity.
\item Frustrated with the bus bottleneck, vendors had teamed together to come up with a new standard. PCs often came equipped with a bus four times faster than legacy ISA called VESA Local Bus (VLB/VL-Bus). 
\item Price of RAM was dropping significantly. The once standard 2 MiB of RAM was now forecasted to be 4 MiB. 
\item The audio ecosystem had become even more fragmented with many SoundBlaster "compatible" audio cards clone on the market and also new innovative technology such as Gravis Ultrasound.\\
\end{itemize}
 \par 
 Not only the hardware had evolved, the software was also different. Better compilers such as Watcom, allowed faster code to be generated. There was less need for time-consuming assembly which was slowly becoming a thing of the past\footnote{Intel would bring that trend back with its super-scalar processor, the Pentium and make Quake development ASM intensive but this is an other story altogether.}. DOS Extenders allowed to break the machine free from 16 bit programming and its infamous limited 1 MiB address space.\\
 \par
 On the developer side, new options had appeared. Powerful workstation were now available and affordable\footnote{..to professionals!}. One company in particular, founded by Steve Jobs after his departure from Apple, combined strong hardware with efficient development tools. \NeXT produced impressive machines running on they UNIX based OS called NeXTSTEP.\\% NextStep were overlooked yet solid productivity boosters.\\
 \par
 In this whirlwind of novelties, it would have been easy to go in the wrong direction. Yet in retrospect, id software seems to have made all the right choices. How did they manage to start from nothing and make one of the best game of all time in just eleven months? This is the question this book will attempts to answer.\\
 \par
 To do so, the first two chapters take a close look at the hardware of the time. Not only the IBM PC on which \doom ran but also the \NeXT machines which id Software elected as the foundation of its production pipeline. The third chapter focuses on the team and the tools they wrote to bridge the hardware and the software. With all these capabilities and restrictions in mind the last chapters are a deep dive in the game engine which hopefully will help to appreciate why things are designed the way they are.\\
\par
So load your shotgun, pack a few extra medkits and let's dive!\\
\par
% \vspace{2cm}
\fullimage{doom-shotgun-hallway.png}
\centering "Doom means two things: demons and shotguns!" -- John Carmack

% \fullimage{actionpacked.png}
%\caption{Fast, Immersive, Dark and Violent.}
