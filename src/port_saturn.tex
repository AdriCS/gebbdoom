\begin{wrapfigure}[10]{r}{0.30\textwidth}{
\centering \scaledimage{0.30}{saturn_logo.png}}
\end{wrapfigure}
Development of the Sega Saturn started in June 1992\footnote{Source: "Console Wars: Sega, Nintendo, and the Battle That Defined a Generation".} as a replacement for the insanely popular yet aging Genesis. At this point in time, the Genesis has sold more than 30 millions units and had a "cool" image among the 15-25 range. Am image built with many good games and massive TV advertising campaigns. It was a colossal yet mandatory undertaking to at least match its predecessor.\\
\par
After two years of hard work, Sega announced the Saturn during Tokyo Toy Show in June 1994. Unknown to them, it would trigger the collapse of Sega International sales and image.\\
\par
During its development, Sega worked in partnership with Hitachi to develop a new CPU tailored to its needs. The join venture resulted in the "SuperH RISC Engine" (a.k.a SH-2) at the end of 1993 which Sega used in dual configuration as foundation for the Saturn.\\
\par
 On the graphic side, one video display processor (VDP) was to do most of the job. However reports of the PlayStation capabilities prompted Sega to add a second VDP to improve the system's 2D performance and texture-mapping.\\
\par
\fullimage{consoles/Saturn.png}\\

\par
Sega managed to release its console before the dreaded PSX and sales in Japan were initially promising with games such as Daytona USA and especially Virtua Fighter being well received. The success had a lot to do with Virtua Fighter which was more popular than Sonic, Mario, and Tetris\footnote{Source: 2006 poll for Top 100 games.}.\\
\par
Preceding Sony came at a great price and came across as rushed. During E3 1995 in Los Angeles, Sega CEO Tom Kalinske surprise announced that the Saturn would be available the very same day. Even supplier did not know about it and the console was out of stock rapidly.\\  
\par
An other consequence of the rush was that only six games were available upon launch. Panzer Dragoon which could have made for a perfect tech flagship missed its deadline\footnote{Source: "The Making Of... Panzer Dragoon Saga".}. The way to do 3D was similar to the 3DO were programmers had to deal with quads that they distorted in screen space to (poorly) fake perspective. It was not a matter of not trying hard, the hardware was complex.\\
\par
\fq{Currently we only use the Master SH2, the slave SH2 will be used when we get around to figuring out how.}{Mick West 1995 Saturn development journal}\\
\par
Worldwide, the the platform was mildly received. And then the beast was unleashed.\\
\par
 Two weeks after release, the PSX came out with Ridge Racer and took the world by storm\footnote{The PlayStation outsold the Saturn by a factor of three.}. Not only the Saturn was more expensive (\$399) than the PSX (\$299), games such as Daytona USA which used to look good now had blatant issues when put side by side with Ridge Racer. The lower framerate, polygon pop-up and letter-boxed presentation begged for mercy. To add more pressure, in June 1996 the market welcomed an other competitor with the Nintendo 64.\\
\par
Ironically Sega's next console, the DreamCast addressed all issues. It was easy to program, ran on Windows and was very well liked by developers. Many good games were released but unfortunately damage had been done. With the 32X and then with the Saturn, Sega had been bleeding money for years. The copy protection was hacked early on. Electronic Arts refused to release its popular E.A sports game on the platform. With the release of the XBox and the PS2, Sega found itself technically outgunned. In late 2001, on the verge of bankruptcy, Sega decided to withdrew from the hardware business and focus on producing games on its successful "Virtua" line of products.








\cfullimage{Sega-Saturn-Motherboard.png}{Sega Saturn motherboard}
\par
Opening a Sega Saturn and taking a look at the motherboard reveals no less than twenty chips.\\
\par
\circled{1} 32-bit 28.6 MHz SH-2, 
\circled{2} 32-bit 28.6 MHz SH-2, 
\circled{3} VDP2, 
\circled{4} The YMF292, aka SCSP (Saturn Custom Sound Processor), 
\circled{5} SCU DSP Math coprocessor @ 14.31818 MH, 
\circled{6} BIOS, 
\circled{7} SMPC (System Management \& Peripheral Control), 
\circled{8} Motorola 68CE00, 
\circled{9} 32 KiB Battery-backed SRAM, 
\circled{A} 4 MiB RAM (2MiB RAM + 1.5MiB VRAM + 540KiB Audio RAM), 
\circled{B} VDP1, 
\circled{C} Hitachi CD-ROM I/O data controller, 
\circled{D} 32-bit 20 Mhz SH1  microcontroller with 64k internal ROM, 
\circled{E} Two controllers connectors, 
\circled{F} A/V OUT socket,  
\circled{G} Sega Communication socket,  
\circled{H} Cart slot (RAM extender requested by "X-Men vs Street Fighter"), 
\circled{I} CD-ROM connector.






\rawdrawing{saturn_motherboard}
Despite its issues and ill timed release, it is a bitter feeling to see what happened to the Saturn and seeing it considered a failure. Over its four years of live, the platform managed to host amazing technical and entertaining games such as Radiant Silvergun, Grandia, Sega Rally Championship, Virtua Fighter 2, Panzer Dragoon Saga, Guardian Heroes, NiGHTS into Dreams, Panzer Dragoon II Zwei and Virtua Cop. \doom would not part of the previous list.\\
\par
\fq{After years of waiting, doom finally arrives on Saturn. Unfortunately it is a breath-takingly bad conversion of a classic game.}{Sega Saturn Magazine \#16, February 1997}


\subsection{Programming the Saturn}
Programming the Saturn was a difficult task. The programmer's manual is broken down in no less than eight voluminous booklet that took several reads before building a mental image of the flow of data. The diagram in figure \ref{arch_saturn} is represents well the daunting effort.\\
\par
Main programming was done via the SH-2 processors connected to 1.5MiB of shared RAM. One SH-2 was deemed the master and the other the slave. In the common configuration, the slave was intended to be used as a helper for parallelizable tasks. Communication between chip (to indicate what to execute) had to be done via a tedious interrupts system. To deal with static and global variables the programmer had to deal with mutex and semaphores which were uncommon concept at that time. Because they were on the same bus, one had to wait for the other if both system needed to access either RAM of something on the system. Attempt to minimize the issue were made via a shared unified 4KiB cache which had little impact on the bottleneck.\\
\par
Audio was done via the SMSP (Saturn Custom Sound Processor) which piloted a Sound processor (a Motorola 68000). The SCSP was to be configured to produce sound mixes in the dedicated 512 KiB of RAM where they were picked up by the Sound Processor. The combination of these made it a powerful system able to synthesize, play PCM, and perform 3D effects/distortion. The chip also polled controls inputs from the player and stored them in internal registers to be polled by the SH-2s.\\
\par
Graphic programming was done via two chips called VDP1 and VDP2. At the beginning of a frame, the VDP1 plots parts of the active framebuffer on the basis of commands and bitmaps in the VRAM. Once done, the active framebuffer is labeled "passive framebuffer". The VDP2 take the passive framebuffer, adds background based scroll panes and send it to the TV. Note that the two chips work in parallel. While the VDP1 works in the next frame, the VDP2 finishes it.\\
\par
Access to the CD-ROM was done via a driver piloting the SH-1 processor. The double speed unit could read at 150 KiB/s but average access time was 300 ms. To compensate, the SH-1 stored data to a 512KiB buffered. Based on the abysmal access time, programmers were instructed to request data well in advance.\\ 
\par
To control all these components and transfer data between system, a seventh chip called the SCU (System Control Unit) acted as DMA controller, DSP and bus controller. The DSP was able to perform matrix transformations and writhe the result directly in the VDP RAM.\\
\par


\trivia{Reading the programmer manual in details reveals that each components can somehow interact with each others RAM. This created a spaghetti hardware machine which made debugging very difficult.}
\pagebreak












\drawing{arch_saturn}{Saturn Block Diagram, "Introduction to Saturn Game Development", April 1994 }
\label{arch_saturn}












\subsection{Doom on Saturn}
Port to the Saturn was done by Rage Software on a very short schedule. Like all other ports, it is based on the Jaguar assets. The console had the same technical shortcomings when it came to texture sampling as remembers Jim Bagley\footnote{RetroGamer \#134.}.\\
\par
\fq{When I started the project, I had to do a demo for id Software to approve. I started by extracting all the levels and audio and textures from the WAD files and made my own Saturn version of this, then got an early version of the renderer working using the 3D hardware. The got sent off and a couple days later I for a call from John Carmack, who stipulated that under no circumstances could I use the 3D hardware to draw the screen. I had to use the processor like the PC. Thankfully I enjoy challenges, so it turned out to be a really enjoyable project, using both SH2s to render the display like the PC did it, using the 68000 to orchestrate them both.\\
\par
However, it kneecapped the game and the speed-framerate suffered greatly.}{Jim Bagley for RetroGamer \#134}\\
\par
Years later, in 2014, Carmack had reconsidered.\\
\par
\fq{I hated affine texture swim and integral quad verts, but in hindsight, I probably should have let experiment.}{John Carmack}\\
\par
The hardware accelerated 60 fps capable engine was tossed. Due to time constraints, Jim did not have the time to change the renderer to work with one pixel wide triangles like the PlayStation. Upon shipping, the game managed a framerate which could go up to 20 but most of the time dropped in the single digit at a full screen resolution of 281x235. To compensate for the low framerate, designers of the game slowed down all movements which enraged the playing community.\\
\par
\trivia{An other effect of the rushed schedule was a pretty big bug with the audio system which made all sound effects to be panned left. Players were forced to play in mono to hear from both speakers.\footnote{Digital Foundry: "Every Console Port Tested and Analysed!".}}







\fullimage{doom_saturn1.png}\\
\par
In the screenshot above, notice how E1M1 is the same at the other version (all based on the initial work for the Jaguar) . The status bar also welcomed a makeover.\\
\par
To take advantage of the VDP parallelism, it it claimed\footnote{Digital Foundry: "Every Console Port Tested and Analysed!".} the VDP1 is used to render sprites and walls while the VDP2 is used for the status bar and also renders visplanes which are treated as background.\\
\par 
Translucency was done in a peculiar way. It is unknown if that was a design decision, a hardware limitation or simply a time constraints but translucency is not done via alpha bending like the Playstation did.\\
\par
 A more software friendly technique known as dithering is used to represent the Specters and glass surfaces.
% \fq{The Saturn had to be programmed straight to the metal. It used quads and messy 3D maths.}{David Shea, Alien Trilogy developer}







\fullimage{doom_saturn4.png}
\par
\fullimage{doom_saturn41.png}


\vspace{-10pt}
\fullimage{doom_saturn2.png}
\par
\fullimage{doom_saturn21.png}





To this day, most of the information about this port was retrieved via reverse engineering.\\
\par
\fq{The main resource wad on the Saturn version is called JIMSDOOM.WAD, obviously named after Jim Bagley, one of the lead programmers for this port. The wad is a 1:1 copy of PSXDOOM.WAD , and even includes resources that only pertain to the PSX port}{Mattfrie1 on \cw{doomworld.com} thread "Dissecting Sega Saturn Doom".}\\
\par
Despite heroic efforts from the team, time constraints crushed the final quality. The game was universally disliked and labeled as the second worse port of doom (right after the 3DO).\\
\par

