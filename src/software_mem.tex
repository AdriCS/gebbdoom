\section{Memory Manager}
Like all game engine of the era, \doom did not trust stock \cw{malloc}, even the one provided by the excellent Watcom's \cw{libc}. Because it could lead to memory fragmentation, a standard allocator would have jeopardized the stability of the engine. It was also a good opportunity to add debugging tools.\\
\par
The original design called for a zone memory manager. However the concept was abandonned in favor of a tag system allowing memory block allocated to be marked and recognized for a certain purpose.\\
\par
\ccode{memory_zone.c}
\par
\cw{PU\_LEVEL} for example allows \doom to free all memory between levels without any tracking. The engine simply ask the zone memory manage to free all blocks tagged \cw{PU\_LEVEL}. An other powerful tag is \cw{PU\_CACHE} which allows the memory system to free a block by itself if it cannot complete an allocation request. \\
\par
  
\par
\trivia{Unsurprisingly the zone allocator is not entierely common to both DOS and \NeXT. Most of the code can be found in \cw{z\_zone.h} and \cw{z\_zone.c} but the RAM system is need to provide \cw{I\_ZoneBase}.\\
\par \tcode{zone_undefined.txt}}
\par
\trivia{Dave Taylor had his own TAG: \cw{PU\_DAVE}, which was used for "anything else Dave wants static". Unfortunately this tag was never used.}