\section{Network}

\fq{I still remember the day that multiplayer started just barely working in Doom. I had two DOS boxes set up in my office—in addition to my NeXT workstation—to test multiplayer. The IPX networking was forwarding user input between the systems, but there was no error recovery, so it was very fragile. Still, I could spawn two marines in a test level, and they could look at each other.\\
\par
I was strafing back and forth on one system and looking over my shoulder at the other computer, watching the marine sprite slide side to side in front of the other player's pistol. I let it coast down, centered on the screen, and turned to the other computer. "Bang!" "Urgh!" Twitch. Shuffle. Big smile. :-) "Bang!" "Bang!" "Bang!" "Bang!" There was a consistency failure before the first frag was truly logged, but it was blindingly obvious that this was going to be awesome.}{John Carmack, kotaku.com "Memories Of Doom"}

\par

\fq{From that point up to the first public release, Doom used IPX broadcast packets to communicate between the players. This seemed like a good efficiency to me—a four player game just involved four broadcast packets each frame. My knowledge of networking was limited to the couple of books I had read, and my naive understanding was that big networks were broken up into little segments connected by routers, and broadcast packets were contained to the little segments. I figured I would eventually extend things to allow playing across routers, but I could ignore the issue for the time being.}{John Carmack, kotaku.com "Memories Of Doom"}

\fq{
What I didn't realize was that there were some entire campuses that were built up out of bridged IPX networks, and a broadcast packet could be forwarded across many bridges until it had been seen by every single computer on the campus. At those sites, every person playing LAN Doom had an impact on every computer on the network, as each broadcast packet had to be examined to see if the local computer wanted it. A few dozen Doom players could cripple a network with a few thousand endpoints.\\
\par
The day after release, I was awoken by a phone call. I blearily answered it and got chewed out by a network administrator who had found my phone number just to yell at me for my game breaking his entire network. I quickly changed the network protocol to only use broadcast packets for game discovery, and send all-to-all directed packets for gameplay (resulting in 3x the total number of packets for a four player game), but a lot of admins still had to add Doom-specific rules to their bridges (as well as stern warnings that nobody should play the game) to deal with the problems of the original release.}{John Carmack, kotaku.com "Memories Of Doom"}

\trivia{The "official" (as stated in the IPXSETUP source code) IPX network socket for Doom is 869C hex (34460 decimal). This was apparently registered with Novell, as it appears in the list of well-known IPX sockets published by Novell.}

DeathManager!

\fullimage{DeathManagerv1-2.png}
\trivia{DeathManager is a 16-bit program that appears to have been written in Turbo C - as many of the other Doom utility programs were. Unlike the other utilities, the source code to DeathManager has never been released.}
\par

The networking code was completely rewritten in v1.2. The networking component was stripped out of the main executable and into separate IPXSETUP.EXE and SERSETUP.EXE drivers. Using networked PCs, it is possible to extend the game's field of view over three monitors. The player needs to set up a network game on the center PC, then the left and right PCs must launch with the -left and -right parameters. 