
\section{2D Renderers (Drawers)}
All "Drawers" in \doom{} were the work of  Dave Taylor. The self proclaimed "spackle coder"\footnote{Source: "Dave Taylor Interview" by blankmaninc.com.} was hired three months before the game shipped but he managed to produce many systems. The code style and naming conventions are very different from John Carmack.
\begin{itemize}
\item Status Bar (\cw{ST\_Drawer()})
\item Automap (\cw{AM\_Drawer})
\item Intermission (\cw{WI\_Drawer()})
\item HUD (\cw{HU\_Drawer()})
\item Menus (\cw{M\_Drawer()})
\item Transition screens (\cw{wipe\_StartScreen()})
\end{itemize} 
\par




\subsection{Intermission}
The intermission screen is simple and fully contained in \cw{wi\_stuff.c}. It start by loading a virgin background map from the wad archive to RAM and placing it in FrameBuffer \#1.\\
\par
\fullimage{intermissions/WIMAP0.png}\\
\par
Every time a screen refresh is needed, the intermission code does a \cw{memcpy} (called "slam" in the code) from FrameBuffer \#1 to Framebuffer \#0, then draw sprites and text on top of it.\\
\par
\fullimage{intermissions/stats.png}\\

Once the intermission is finished, method \cw{WI\_unloadData} does not free the RAM, it just mark all elements it needs as PU\_CACHE.\\

%\fullimage{intermissions/Intermission.png}\\



\par

\subsection{Status Bar}
The design of the status bar is similar to the Intermission Drawer. Contained in \cw{st\_lib.c} and \cw{st\_stuff.c}, the code calls its elements "widgets'. There are seven widgets, from left to right, ready weapon ammo, health percentage, arms, faces, armor percentage, keyboxes, and all four kind ammo count.\\
\par
\fullimage{stbar.png}
\par
%\trivia{Head shows where damage came from.}\\

\par
\fullimage{faces.png}\\
\par
Most of the time, the face widget plays the "normal" animation where the marine's eye looks left and right. Notice how there is a set of state for each five bracket of life percentage. Perhaps because of the stress incurred during combat, many players never knew this widget showed the origin (left/right) of damage incurred. "Evil" is showed when picking up a weapon. "Kill" is obviously displayed when opponents are slayed.\\
\par
Even to player who spent a lot of time on \doom, the "Ouch" face is likely to be something they have never seen. It was intended to show up when the player suffers an enormous amount of damage (more than 20 hit points), enough to move two brackets down. A bug in the code prevented this from happening:\\
\par
\ccode{ouch_face.c}\\
\par
The test is the opposite of what it should have been. The ST module shows the ouch face when 20 life is gained which almost never happens. The test should be reversed.\\
\par
\ccode{ouch_face_fixed.c}\\
\par
With regards to interaction with the video system, the status bar is similar to the intermission module. Upon startup, it draws a virgin status bar in FrameBuffer \#4. When the status bar needs a refresh, it does a \cw{memcpy} from FrameBuffer \#4 to FrameBuffer \#0 and draws all the widgets on top of it.\\
\par
\fullimage{stbar_virgin.png}


\subsection{Menus}

\begin{wrapfigure}[5]{r}{0.1\textwidth}
\centering
\scaledimage{0.1}{M_SKULL1.png}
\end{wrapfigure}
There are ten menus and they are all hard-coded in \cw{m\_menu.c} and \cw{m\_misc.c}. The design is simple with \cw{menu\_t} containing a list of \cw{menuitem\_t}. A \cw{name} field for each menuitem is used to retrieve the appropriate sprite.\\
\par

\ccode{menu_item.c}

\par
A \cw{menu\_t*} is consumed by the menu drawing routine and a skull is drawn on the left of the currently selected \cw{menuitem\_t}.\\
\par
\begin{minipage}{0.55\textwidth}
\ccode{main_menu.c}
\end{minipage}
\begin{minipage}{0.45\textwidth}
\centering
\scaledimage{0.9}{menu.png}
\end{minipage}
\par
\ccode{M_DrawMainMenu.c}
\par
In the previous code sample describing the main menu, notice how the strings \cw{M\_NGAME}, \cw{M\_OPTION}, \cw{M\_DOOM} are all names of a lump found in the wad archive.\\
\par
%\trivia{The slider to select the screen size, mouse sensitivity, sfx volume, and music volume are called "Thermostats".}\\
%\par








\subsection{HUD (Head Up Display)}
In its early instances, \doom{} featured a HeadUp Display mimicking doomguy helmet.\\
\par
\fullimage{alpha_hud.png}
\par
Over time the design changed and the HUD shrank to be reduced to lines of text. The small code is contained in \cw{hu\_lib.c} and \cw{hu\_stuff.c}.\\
\par
\fullimage{new_hud.png}\\
\par
%\trivia{Despite being a small market, french characters receveid special attention in this module.}



\subsection{Automap}
The automap is a small and simple component contained in \cw{am\_map.c}. As the players discovers the level, the map keeps track of lines which have been seen. Red lines indicate solid walls. Yellow lines indicate changes in ceiling height (e.g. doors). Brown lines indicate changes in floor height.\\
\par
Sadly, it is not possible to play the game in this mode (we all tried to, let's not kid ourselves) since visited lines marking is done by the 3D renderer which is disabled when the automap is active.\\
\par
\fullimage{automap.png}
\par
\trivia{Automap almost featured an easter egg. Files \cw{am\_oids.h/c} were to allow player to play a remake of Asteroids. Unfortunately the easter egg was left unfinished.}\\
\par
\fq{I can't recall whose idea it was, but it was probably mine.  I was taken by the vector art style of the automap, so Asteroids would have been a good fit, but Doom was behind, and the pace of development at id was incredibly fast.}{Dave Taylor}\\
\par
%\trivia{The cheatcode \cw{iddt} allowing to see everything in the automap is named after Dave Taylor.}






\subsection{Wipe}
\label{label_melt}
Also known as the "Screen melt", wipe is used to transition between sections of the game.

\par
\scaledimage{.9}{melt/melt0.png}

 \vspace{10pt}
\scaledimage{.9}{melt/melt1.png}
\pagebreak

\scaledimage{.9}{melt/melt2.png}

 \vspace{10pt}
\scaledimage{.9}{melt/melt3.png}
\pagebreak

\par
"Wipe" takes whatever is in FrameBuffer \#2 and progressively transforms it into what is stored in FrameBuffer \#3, writing the output in FrameBuffer \#0.\\% (FrameBuffer \#1 is left available in case a screenshot request was received in the middle of a wipe.).\\
\par
The first step is to reorganize the two sources of data from row major to column major. This is done so the read operations on vertical strip play nicely with the 486 cachelines.\\
\par
\fullimage{face_rawmajorized.png}
\par
After that, a random sequence of 160 numbers is generated in an array \cw{y} where each value are within 16 units of their two neighbors. These are used to form the top "wave".\\
\par The animation is made so columns of pixels from the source are falling down, letting the destination show behind, like if the source image had been wiped of the screen. During the animation, columns two pixels wide and 200 pixels tall are copies repeatedly from src and dest to Framebuffer \#0. All columns falls at the same speed but they are offset by values in \cw{y}, hence producing the wipe illusion.
%The code for the wipe animation is in \cw{dutils.c}, all method start with \cw{wipe\_*}. Also known as \\




