\section{RAM}
With the price of RAM dropping, game developers now could count on 4 MiB. This increase should have been a good news, resulting in video games with richer worlds shipping with better assets, more characters and bigger maps. Due to the infamous way memory had to be managed that meant more complexity to handle and more headache.\\
\par
The fault fell a little bit on Intel and a lot on Microsoft. In 1981, IBM released its first PC, the 5150,  which was built around an Intel 8088. The The CPU was limited to 16-bit registers, but Intel wanted it to be able to access a 20-bit address space. To reconcile both elements Intel designers came up with an abomination called segmented addressing where two 16-bit registers were combined to form a 20-bit address.\\
\par

\drawing{register_combination_20_bits_address}{}

\par
Pointer manipulation was error prone since different segment/offset combinations could point to the same RAM location. There were also issues related to pointer arithmetic where once the offset wrapped around, the segment was not automatically updated.\\
\par
With the release of the 386 and later 486, things worsened. The bus address width was increased to 24 bits which once again was bigger than the 20-bit segmentation trick was capable off. The solution was to resort to memory managers such as \cw{EMM386.EXE} and \cw{HIMEM.SYS}\footnote{16-bit programming and memory manager were covered in \textit{Game Engine Black Book: Wolfenstein 3D}.} which both provided means to work with non-addressable RAM located beyond the 1 MiB barrier.\\

There would have been a simpler solution. Intel allowed its CPUs to function in two modes. The backward compatible real mode which made the CPU behave like a very fast 8088 and the protected mode which unleashed all the power of the CPU. In protected mode, 32-bit registers were bit enough to address all RAM on board (this is known as flat addressing).\\
\par
It would have worked out if the operating system had been able to run in protected mode. However, in the name of backward compatibility, Microsoft's DOS could only handle real mode which effectively locked developers to 16-bit programming.\\ 
\par




\par
Among the growing growing pain and frustration of DOS, some people saw an opportunity.\newpage 

While there were many products that could address this, two companies in particular stood out with a winning combination: Watcom International Corporation would produce a C compiler, and Rational Systems would produce a "DOS extender", that together allowed programs to run in protected mode while still having access to 16-bit DOS functions.\\
\par


\subsection{DOS/4GW Extender}
Under DOS the normal way to perform a "system call" is to use a software interrupt instruction with parameter \cw{21h}. In C programming, this was abstracted away by header \cw{DOS.H} which performed all the lower level work behind the scenes.\\
\par
\drawing{realmode_app_limk}{}
To allow the app to run in one mode and the OS to run in an other, the two worlds had to be bridged. A middle layer called "DOS Extender" able to run in both mode inserted itself between the program and the operating system.\\

\par
\rawdrawing{realmode_app_limk_extender}
Upon startup, the DOS Extender would place hooks in the OS software Interrupt Vector Table and place its own routines there. From an application standing point everything was transparent, the developer had no code to change. To perform a system calls not hooked by the extender (e.g: \cw{int 33h} to read mouse inputs), the extender offered a special interface called DPMI on interrupt \cw{31h} which took care of translating 32-bit register request to 16-bit so IVT routines would understand them.\\
\par
\trivia{DMPI (DOS Protected Mode Interface) was originally created to allow Windows 3.0 to run 32-bit application and be compatible with a joint operating system project with IBM called OS/2.}\\
\par
When the extender intercepted an operating system call, it had to do a lot of work:
\begin{enumerate}
\item Perform all translation needed (e.g: a 32-bit address had to be expressed as a 16-bit offset with a 16-bit segment).
\item Switch the CPU to real mode.
\item Forward the call to DOS.
\item Retrieve the results and convert 16-bit register values back to 32-bit.
\item Switch the CPU back to protected mode.
\end{enumerate} 

The performance sensitive operations were switching between real mode and protected mode. Originally this was a problem on 286 CPUs since Intel never imagined a program may want to switch back to real Mode from protected Mode. Various tricks had to be used\footnote{These are detailed in \textit{Game Engine Black Book: Wolfenstein 3D}.} among them faking a keyboard Ctrl-Alt-Del reboot to reset the CPU without actually rebooting.\\
\par
On the other side, switching from real Mode to protected Mode is simple. Setting the Control Register from bit 0 to 1 takes six instructions.\\
\par
\acode{switch_to_protected_mode.asm}
\par

\doom{} used DOS/4GW extender by Rational Systems. Its presence could briefly be seen on startup. Executing \cw{DOOM.EXE} triggered DOS to load the tiny extender. Once loaded, DOS/4GW switched the CPU into Protected Mode, loaded \doom's code into memory and branched to the \cw{main} function.\\
\par
\fakedosoutput{dos4gw.txt}





