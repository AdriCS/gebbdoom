Doom was ported to Windows for Microsoft by Gabe Newell. TOOD: Interview him ?!
\section{Jaguar (1994)}
\section{Sega 32X (1994)}
\section{Super Nintendo (1995)}
Super FX 2 chip.

\fq{DOOM was a truly ground-breaking title and I wanted to make it possible for gamers without a PC to play the game, too. DOOM on the Super Nintendo was another one of those programming challenges that I knew could be accomplished.\\
\par

I started the project independently and demo’d it to Sculptured Software when I had a fully operational prototype running. A bunch of people at Sculptured helped complete the game so it could be released in time for the holidays.\\
\par
The development was challenging for a few reasons, notably there were no development systems for the SuperFX chip at the time. I wrote a complete set of tools — assembler, linker and debugger — before I could even start on the game itself.\\
\par
The development hardware was a hacked-up StarFox cartridge (because it included the SuperFX chip) and a modified pair of game controllers that were plugged into both SNES ports and connected to the Amiga’s parallel port. A serial protocol was used to communicate between the two for downloading code, setting breakpoints, inspecting memory, etc.\\
\par
I wish there could have been more levels but unfortunately the game used the largest capacity ROM available and filled it almost completely. I vaguely recall there were roughly 16 bytes free, so there wasn’t any more space available anyway! However, I did manage to include support for the SuperScope, Mouse and XBand modem! … Yes, you could actually play against someone online!}
{Randy Linden (Interview with gamingreinvented.com)}
\par
Almost no censorship from Nintendo except for removing red blood.

\section{Playstation 1 (1995)}
How PSX renders doom: https://www.youtube.com/watch?time\_continue=58\&v=MS9MGSm4gng\\
\par
\section{3DO (1996)}
\fq{this was the product of ten intense weeks of work due to the fact that I was misled about the state of the port when I was offered the project. I was told that there was a version in existance with new levels, weapons and features and it only needed "polishing" and optimization to hit the market. After numerous requests for this version, I found out that there was no such thing and that Art Data Interactive was under the false impression that all anyone needed to do to port a game from one platform to another was just to compile the code and adding weapons was as simple as dropping in the art.\\
\par
Uh... No...\\
\par
My friends at 3DO were begging for DOOM to be on their platform and with christmas 1995 coming soon (I took this job in August of 1995, with a mid October golden master date), I literally lived in my office, only taking breaks to take a nap and got this port completed.\\
\par
Shortcuts made...\\
\par
I had no time to port the music driver, so I had a band that Art Data hired to redo the music so all I needed to do is call a streaming audio function to play the music. This turned out to be an excellent call because while the graphics were lackluster, the music got rave reviews.\\
\par
3DO's operating system was designed around running an app and purging, there was numerous bugs caused by memory leaks. So when I wanted to load the Logicware and id software logos on startup, the 3DO leaked the memory so to solve that, I created two apps, one to draw the 3do logo and the other to show the logicware logo. After they executed, they were purged from memory and the main game could run without loss of memory.\\
\par
There was a Electronic Arts logo movie in the data, because there was a time that EA was going to be distributing the game, however the deal fell through.\\
\par
The vertical walls were drawn with strips using the cell engine. However, the cell engine can't handle 3D perspective so the floors and ceilings were drawn with software rendering. I simply ran out of time to translate the code to use the cell engine because the implementation I had caused texture tearing.\\
\par
I had to write my own string.h ANSI C library because the one 3DO supplied with their compiler had bugs! string.h??? How can you screw that up!?!?! They did! I spent a day writing all of the functions I needed in ARM 6 assembly.}{Rebecca Ann Heineman}
\section{Game Boy Advance (2001)}
\section{Saturn}
https://www.doomworld.com/forum/topic/86671-dissecting-sega-saturn-doom/\\
\par
JIMSDOOM.WAD, obviously named after Jim Bagley.\\
The wad is a 1:1 copy of PSXDOOM.WAD\\
John Carmack shut down their original plan to use a hardware-accelerated renderer\\
\fq{I hated affine texture swim and integral quad verts, but in hindsight, I probably should have let experiment.}{John Carmack}
See author comments: https://www.doomworld.com/forum/topic/86671-dissecting-sega-saturn-doom/?page=2\#comment-1580057\\
\fullimage{affine_texture_mapping/post_far.png}\\
\fullimage{affine_texture_mapping/post_near.png}\\
\par
\fullimage{affine_texture_mapping/door.png}\\
\par
\fullimage{affine_texture_mapping/hallway.png}\\
\par
\begin{minipage}{\textwidth}
\scaledimage{0.4}{affine_texture_mapping/tex_perp_pc_repeat_nearest_replace.png}
\scaledimage{0.4}{affine_texture_mapping/tex_perp_no_pc_repeat_nearest_replace.png}\\
\par

\scaledimage{0.4}{affine_texture_mapping/tex_pc_repeat_nearest_replace.png}
\scaledimage{0.4}{affine_texture_mapping/tex_no_pc_repeat_nearest_replace.png}\\
\par

\scaledimage{0.4}{affine_texture_mapping/tex_pc_repeat_nearest_replace_sharp.png}
\scaledimage{0.4}{affine_texture_mapping/tex_no_pc_repeat_nearest_replace.png}\\
\par
\end{minipage}
\par
Explanation:\\
\par
\fullimage{affine_texture_mapping/affine2.png}
There are solutions to this:\\
\par
\fullimage{affine_texture_mapping/subdivision_sample.png}
