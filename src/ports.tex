The success of the PC version and its mind-numbing sales figures made it an extremely desirable title for any game console publisher. From 1994 to 1997, \doom{} was ported to the six major systems of the era with varying degrees of success.\\
\par
% \begin{itemize}
%    \item 1994: Atari Jaguar and the Sega 32X.
%    \item 1995: Super Nintendo and Sony Playstation.
%    \item 1996: 3DO.
%    \item 1997: Sega Saturn.
% \end{itemize}
% INSER ARRAY HERE: Console, Year, Authors, Avg framerate, price, CPUS, RAM.
% \bigskip

% \begin{figure}[H]
% \centering  
% \begin{tabularx}{\textwidth}{ L{1}  R{1} R{0.5} R{0.5} R{0.5} R{0.5} }
%   \toprule
%    \textbf{Year} & \textbf{System} & \textbf{\#CPUs} & \textbf{RAM} & \textbf{Resolution} & \textbf{Price} \\
%   \toprule 
%             1994 & Atari Jaguar.   &  5              & 2 MiB.       &  320x240            & BLA\\
%             1994 & Sega 32X        &  3              & ISA-8        &  8                  & BLA\\
%             1995 & Super Nintendo  &                 & ISA-16       & 13                  & BLA\\
%             1995 & Sony Playstation&                 & VLB          & 24                  & BLA\\
%             1996 & 3DO             &                 & VLB          & 24                  & BLA\\
%             1997 & Sega Saturn     &                 & VLB          & 24                  & BLA\\
%    \toprule
%  \end{tabularx}
% \caption{Ports of \doom to video game console from 1995 to 1997}
% \end{figure}

At the time, the four year "console war" was raging, with consoles ranked by generation according to their "bitness". The third generation, 8-bit NES and Sega Master System had long since disappeared. The fourth, 16-bit generation, consisting of Nintendo's SNES, Sega's Genesis and the Turbografx-16 was reaching end of life. The fifth, 32-bit generation, was starting to appear, featuring Sony's Playstation and Sega's Saturn, with marketers trying to brand some systems as 64-bit, including the Nintendo 64 and Atari Jaguar\footnote{After this, consumers either tired of "bits" or realized the silliness of the whole nomenclature. Bitness was soon forgotten.}. In retrospect, it was a rich period, prone to hardware innovation which contrasts compared to 2018's uniform world of Sony vs Microsoft where systems barely differ by more than their name.\\
\par
The architecture of the \doom{} engine, based on a core with system-specific subsystems, may suggest it would have been an easy task to port it to consoles. This intuition could not be further from the truth. From design trade-offs due to restricted resources, to crazy schedules, all versions have a unique and rich story to tell.\\
\par
Technically, the common problem to solve was to deal with smaller memory requirements. The PC minimum requirement was to have 4 MiB installed on the machine. It was of course not possible to ask customers to add more RAM to their console. Developers sometimes had to deal with as few as 512 KiB of RAM which was eight times less than the original version.\\
\par
The second technical difficulty was in dealing with exotic hardware. The PC was designed around a single "big iron" CPU while consoles were made of a constellation of processors.


\section{Jaguar (1994)}
\subfile{port_jaguar} 

\section{Sega 32X (1994)}
\subfile{port_x32} 

\section{Super Nintendo (1995)}
\subfile{port_snes}

\section{Playstation 1 (1995)}
\subfile{port_psx}

\section{3DO (1996)}
\subfile{port_3do}

\section{Saturn (1997)}
\subfile{port_saturn}

