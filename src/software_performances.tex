\vspace{-40pt}
\section{Performance} \label{performances}
There is a convenient and portable way to assess performance. Thanks to the fixed tic duration architecture, a demo can be recorded and played back exactly. With frame skipping disabled, a \cw{-timedemo} will produce the exact same sequence of frames to render regardless of the machine's power. Only wall time will vary.\\
\par
The gold standard of \doom{} benchmarking was created by Anton Ertl around 1994. It consists of playing back DEMO3 from the unregistered version of \doom{} shareware v1.9.\\
\par
\fakedosoutput{timedemo.txt}
\par
Over the last twenty five years, Anton has gathered metrics for hundreds of configurations\footnote{Source: https://www.complang.tuwien.ac.at/misc/doombench.html} of the famous game session recorded by John Romero. The machines tested range from Amiga 1200s up to Core i5s. Because no frames are skipped, the playback duration varies. Upon completion two values are displayed.\\
\par
\fakedosoutput{timedemo_result.txt}
\par

The first value, \cw{gametics}, is the number of game tics rendered. For \cw{demo3} this is always equal to 2134 since it comes from the recorded lump. The second value, \cw{realtics}, represents the wall time in tics that it took to render every frame.\\
\par
 The average frame per second is obtained with the following formula:\\
 \par 
\begin{equation*}
\resizebox{.4 \textwidth}{!} 
{
 $ fps = \frac{gametics}{realtics} * 35 $ 
 }
 \end{equation*}\\

 
In the previous example\footnote{Benchmark machine was a miniPC Unisys CWD 4001 (486DX2-66/CirrusLogic-GD5424).}, the game ran at an average of $\frac{2134}{4268}*35 = 17.5$ fps. \\
\par
This mechanism allows running benchmarks across varying configurations. Since machines from 1994 have become difficult to come by these days, a generous collector named Foone Turing kindly volunteered his impressive fleet of machines. The results of the archaeological-benchmarking session are visible in figure \ref{bnechmarsks}.\\
\par
 The results of the session demonstrate that none of the hardware of the time was able to max out the game. Remember that beyond 35fps there would be no visual improvement since the game logic is hard-coded to run at 35Hz. A machine able to render 70fps video/audio would render the same game frame twice.\\
\par
\begin{figure}[H]
\centering  
\begin{tabularx}{\textwidth}{ L{1}  R{0.3} R{2.4} R{1} R{0.3} }
  \toprule
   \textbf{CPU} & \textbf{Frequency} & \textbf{Graphic card} & \textbf{Bus} & \textbf{fps}\\
  \toprule 
  386DX & 33 & Tseng Labs ET3000    & ISA-8  &  4\\
  386DX\protect\footnotemark & 33 & Cirrus Logic CL-GD5420 & ISA-16 &  7\\
  \toprule 
  486SX & 33 & Tseng Labs ET3000                & ISA-8  &  7\\
  486SX & 33 & Cirrus Logic CL-GD5420           & ISA-16 & 11\\ 
  486SX & 33 & Diamond Stealth (Tseng ET4000)   & VLB    & 15\\
  \toprule 
  486DX2 & 66 & Tseng Labs ET3000               & ISA-8  &  8\\
  486DX2 & 66 & Cirrus Logic CL-GD5420          & ISA-16 & 13\\
  486DX2 & 66 & Diamond Stealth (Tseng ET4000)  & VLB    & 24\\
   \toprule
 \end{tabularx}
\caption{Benchmark with out-of-the-box \doom{} shareware.}
\label{bnechmarsks}
\end{figure}
\footnotetext{For reference, this configuration was able to run Wolfenstein 3D at 20 fps.}
\par
Using out of the box settings for the game, a top of the line machine could barely reach 25 fps\footnote{It would not be until the Pentium when PCI buses came out that \doom{} could be maxed out at 35 fps.}. Notice the importance of the bus, which yields a 3x performance hit/boost. At equal frequencies a 486 provides twice the framerate of a 386.\\
\par






\subsection{Profiling}
Even without special tools, it is possible to gain insight into which parts of the engine are responsible for CPU cycle consumption, thanks to built-in command line parameters.\\
\par
Parameter \cw{-nodraw} skips rendering altogether (but does blit).\\
\par
\par
\fakedosoutput{timedemo_result_nodraw.txt}
\par
Without drawing, the game's framerate improved from 17fps to 878fps.\\
\par
Parameter \cw{-noblit} renders to RAM but doesn't transfer the content to VRAM which is a good way to assess the impact of the bus speed. Because of optimizations described later this parameter was only available on non-DOS versions like on NeXTSTEP.\\
\par
\tcode{timedemo_result_noblit.txt}


\vspace{-10mm}
\subsection{Profiling With A Profiler}
An excellent way of visualizing performance is with a flame graph. These are built by running a program, repeatedly interrupting it, and unwinding the stack starting from the Program Counter to generate a backtrace. This is repeated hundreds of times while playing back a demo. After completion, all backtraces are collected and merged together.\\
\par
 This produces a tree where the width represents 100\% of the time and each level is a function call. It provides a visual breakdown of where the machine spends time during a frame\footnote{This is a wall time-based flamegraph but there are many other kinds, like CPU-cycle based for example.}. On NeXTSTEP, thanks to the multiprocessing capability of the OS it is easy to use \cw{gdb} from a second terminal and gather backtraces. The result is as follows.\\
\par

\drawing{flamegraph}{Flame graph of \cw{./doom -timedemo demo1} on NeXTstation TurboColor}
\par
Obviously, \cw{D\_DoomLoop} accounts for 100\% of the time, with \cw{D\_Display} overwhelmingly dominating\footnote{But keep in mind the NeXTSTEP port did not implement the audio system.}. Gameplay (\cw{TryRunTics}) (isolated column on the very right) is barely visible. In a flame graph, bottlenecks are identified by "mesas" which are high flat plateaus with no children. With that clue, notice the high cost of blitting from framebuffer \#0 to the screen (\cw{Update16}), the horizontal drawing routine (\cw{R\_DrawSpan}) rendering visplanes and the less obvious BSP traversal resulting in vertical drawing routines (\cw{R\_RenderBSPNode}) to render walls/sprites.




On DOS, generating a flame graph is more difficult since it is a single-threaded operating system. However, with a little bit of instrumentation it is possible to get something similar. The following flame graph was generated by instrumenting ten functions.\\
\par
\vspace{4mm}
\drawing{dos_flamegraph}{Flame graph of \doom{} running on DOS}
\par
The DOS flame graph shows that I.A. ran in \cw{G\_Ticker} consumes little CPU time. The audio routine \cw{S\_UpdateSounds} since even less taxing since it is barely visible in the tiny red span. This makes sense since the only work it does is to retrieve sound effects and music data from the \cw{.wad} and place it into RAM for the audio card to use.\\
\par
 Something weirder to notice is that, despite being instrumented, \cw{I\_FinishUpdate} is not visible at all whereas it was a huge part of the loop on \NeXTns{}. How can the DOS version transfer a full framebuffer across the bus so fast? As it turns out, the DOS version benefited from heavy optimization, and as a result, did not have to blit at the end of a frame.\\
\par

\subsection{DOS Optimizations}
Given that it was meant to be the money maker, the DOS version received special care. During optimization sessions, John Carmack identified three areas for improvement. One was in the math functions and two had to do with the 3D renderer.\\
 \par
 \fq{An exercise that I try to do every once in a while is to "step a frame" in the game, starting at some major point like common->Frame(), game->Frame(), or renderer->EndFrame(), and step into every function to try and walk the complete code coverage. This usually gets rather depressing long before you get to the end of the frame. Awareness of all the code that is actually executing is important, and it is too easy to have very large blocks of code that you just always skip over while debugging, even though they have performance and stability implications.}{John Carmack}
\par




\subsubsection{Math Optimizations}
Fixed-point operations are performed everywhere. The function \cw{FixedMul} is found 124 times in the source code and is called over a thousand times per frame. Along with \cw{FixedDiv2}, it was optimized with inline assembly\footnote{There is next to no assembly in \doom. Around 1994, John Carmack even declared that "The days of assembly are numbered". This was without counting on Intel's Pentium, the super-scalar architecture of which would require extra care by Michael Abrash for Quake.}. The C version "\cw{return ((long long) a * (long long) b) >> 16}" was a function call in Watcom resulting in close to 30 instructions but the assembly version only uses two.\\
\par
\acode{FixedMul.asm}\\


\vspace{-2mm}
\subsubsection{Direct Framebuffer Access}
A more substantial optimization had to do with layering. On DOS the rules separating the core and the video system were bent. Renderers like the status bar and the automap still function as originally designed but 3D drawing functions such as \cw{R\_DrawColumn} and \cw{R\_DrawSpan} are given direct access to the VGA banks. By bypassing framebuffer \#0, one read and one write per pixel (plus bus transfer) are avoided. Since the menu renderer has to be able to draw on top of everything, it was also granted direct VGA VRAM access.\\

\vspace{-10pt}
\subsubsection{Assembly Renderer Optimization}
Not only were \cw{R\_DrawColumn} and \cw{R\_DrawSpan} given direct access to the VGA bank, they were hand-optimized with gorgeous assembly using all the tricks in the book. Taking a look at \cw{R\_DrawColumn} from \cw{planar.asm} is revealing. The function uses self-modifying code (see reserved value \cw{12345678h} meant to contain the scaling factor, which is patched). We can also see the loop was unrolled to process two pixels at at time (and therefore avoid emptying the i486 prefetch queue because of the \cw{jnz} instruction). Notice the cool trick where register \cw{eax} is reused three times, once as a pointer to the texture source, then as \cw{al} for texel storage (\cw{al}), then as \cw{al} for translated texel (lightmapped) storage again. Overall this method yields an amortized 7 instructions per pixel for drawing columns which is remarkable.\\
\par
Altogether these three optimizations improved performance by a substantial 15\%.






\acode{R_DrawColumn.asm}

\vspace{-1.25cm}
\section{Performance Tuning}
Despite all the care and optimization, most gamers could not get more than 10 fps with the game out of the box. To help reach a decent framerate, two tuning mechanisms helped to reduce the number of pixels written.
\begin{enumerate}
\item High detail/low detail toggle.
\item Adjust the size of the 3D canvas.
\end{enumerate}
\par
\trivia{These tradeoffs were not specific to the PC. All console ports -- from the "weak" Super Nintendo to the "strong" PlayStation -- used a combination of these two settings.}\\
\par



\section{High/Low detail mode}
The first tuning option was to lower the horizontal resolution using column doubling. In low resolution mode the engine only renders one out of very two columns but it writes them to the framebuffer twice. Given how the 3D renderer dominates runtime, this resulted in a tremendous performance improvement since fewer pixel values are generated but they also don't have to transit the bus.  A performance gain confirmed as the following benchmark shows.\\
\par
 \vspace{0.5cm}
\begin{figure}[H]
\centering  
\begin{tabularx}{\textwidth}{ C{1} C{1} C{1} C{1} } 
  \toprule
  \textbf{High detail resolution} & \textbf{High detail FPS} &  \textbf{Low detail resolution}  & \textbf{Low detail FPS}\\
  \toprule 
320x200  &  19   & 160x200  &       29 \\
320x168  &  20   & 160x168  &       30 \\
288x144  &  23   & 144x144  &       32 \\
256x128  &  25   & 128x128  &       35 \\
224x112  &  28   & 112x112  &       38 \\
192x096  &  31   & 096x096  &       41 \\
160x080  &  35   & 080x080  &       45 \\
128x064  &  40   & 064x064  &       49 \\
096x048  &  45   & 048x048  &       54 \\
  \toprule
\end{tabularx}
\caption{\doom{} performance in low and high detail mode.}
\end{figure}
The benchmark above was conducted on a machine which would have been deemed "top of the line" in 1994, a Unisys CWD 4001 featuring a 486DX2-66 CPU with a Cirrus Logic VLB graphics card. Using low detail mode instead of high detail yields a variable 20\%-50\% performance improvement which brought \doom{} up to a playable framerate on "weaker" PCs running on Intel 386 CPUs.


\vspace{-6pt}
\fullimage{high_res.png}\\
\par
Above, the high resolution is 320x168. Below, in low resolution, it is dropped to 160x168 with odd columns duplicated. The resolution drop is particularly noticeable on the non-magnified sergeant and the door but only because diminished lighting and CRT scaling are disabled for demonstration purposes. In practice the difference was more subtle.\\
\par
With direct access to the VGA banks, "low detail mode" is a completely free optimization without the need to write the same pixel column twice, since the VGA mask is set up to write the same pixel to two banks simultaneously.\\
\par
\vspace{-3pt}
\fullimage{low_res.png} 

\vspace{-30pt}
\section{3D Canvas size adjustment}
Another option to improve the frame rate was to lower the size of the 3D canvas. The player had access to a sliding bar allowing eight sizes to be selected. The slider was a multiplier affecting the variable \cw{numblocks} to produce a value in the range [3,11]. Value 11 was special and hard-coded to be recognized as "full-screen" 320x200.\\
\par
\vspace{4mm}
\ccode{R_ExecuteSetViewSize.c}
\par
It is unknown if anybody had the misfortune and courage to play the game in mode 3. This would have been an achievement in itself.\\
\par
\drawing{lod}{The nine 3D canvas configurations.}
\par
\trivia{id shipped \doom{} by default in "high detail" with canvas size \cw{9}.}

\begin{figure}[H]
\centering  
\begin{tabularx}{\textwidth}{ R{1}  R{1} R{1} R{1}  R{1.4}  R{0.6} }
  \toprule
  \textbf{numblocks} & \textbf{Width} & \textbf{Height} & \textbf{\# Pixels} & \textbf{\# Pixels \%} & \textbf{fps}\\
  \toprule 

 0xB & 320 & 200 & 64,000 & 100 & 19 \\
 0xA & 320 & 168 & 53,760 &  84 & 20 \\
 0x9 & 288 & 144 & 41,472 &  64 & 23 \\
 0x8 & 256 & 128 & 32,768 &  52 & 25 \\
0x7 & 224 & 112 & 25,088 &  39 & 28 \\
 0x6 & 192 & 96 & 18,432 &  28 & 31  \\
 0x5 & 160 & 80 & 12,800 &  20 & 35  \\
 0x4 & 128 & 64 &  8,192 &  12 & 40  \\
 0x3 &  96 & 48 &  4,608 &   7 & 45  \\
   \toprule
\end{tabularx}
\caption{Benchmark of performance gain vs 3D canvas size.\protect\footnotemark}
\end{figure}

\footnotetext{Benchmarked on a miniPC Unisys CWD 4001 (486DX2-66/CirrusLogic-GD5424, 8MiB RAM).}
\par
Gains are not as substantial as the detail level. Reducing the 3D canvas size by 50\% yields only a 31\% improvement yet the visible area is so small it is almost not worth it.\\

\par
{
\setlength{\belowcaptionskip}{-10pt}
\cfullimage{shipped_config}{\doom{} out-of-the-box configuration: Mode 9, High detail}
}

