\section{Performances} \label{performances}
There is a convenient and portable way to assess performances. Thanks to the fixed tic duration architecture, a demo can be recorded and replayed with exactitude. With frame skipping disabled, a \cw{timedemo} will produce the exact same sequence of frames to render regardless of a machine's power. Only wall time will vary.\\
\par
The gold standard of \doom~benchmarking was created by Anton Ertl around 1994. It consist in playing the default timedemo from the unregistered version of \doom~shareware v1.9.\\
\par
\fakedosoutput{timedemo.txt}
\par
Over the last twenty five years, Anton has gathered metrics for hundreds of configuration\footnote{Source: https://www.complang.tuwien.ac.at/misc/doombench.html} for the famous game session recorded by John Romero. The machines tested range from Amiga 1200 up to Core i5. Because no frames are skipped, the playback duration varies. Upon completion two values are displayed.\\
\par
\fakedosoutput{timedemo_result.txt}
\par

The first value, \cw{gameticks}, is the number of game tic rendered. For \cw{demo3} it is always equal to 2134 since it comes from the record lump. The second value, \cw{realticks}, represents the walltime in ticks it took to render all each frames.\\
\par
 The average frame per second is obtained with the following formula:\\
 \par 
\begin{equation*}
\resizebox{.4 \textwidth}{!} 
{
 $ fps = \frac{gameticks}{realticks} * 35 $ 
 }
 \end{equation*}\\

 \par
In the previous example\footnote{Benchmark machine was a miniPC Unisys CWD 4001 (486DX2-66/CirrusLogic-GD5424).}, the game ran at an average of $\frac{2134}{4268}*35 = 17.5$ fps. \\
\par
This mechanism allowed to run benchmark across varying configurations. Since machines from 1994 have become difficult to come around these days, a generous collector named Foone Turing kindly volunteered his impressive fleet of machines. The results of the archeolog-benchmarking session are visible in figure \ref{bnechmarsks}.\\
\par
 The result of the benchmarking session demonstrated that none of the hardware of the time was able to max out the game. Remember that beyond 35fps there would be no visual improvement since the game logic is hardcoded at 35Hz. A machine able to render 70fps video/audio would render the same game frame twice.\\
\par
\begin{figure}[H]
\centering  
\begin{tabularx}{\textwidth}{ L{1}  R{0.3} R{2.4} R{1} R{0.3} }
  \toprule
   \textbf{CPU} & \textbf{Frequency} & \textbf{Graphic card} & \textbf{Bus} & \textbf{fps}\\
  \toprule 
  386DX & 33 & Tseng Labs ET3000    & ISA-8  &  4\\
  386DX\protect\footnotemark & 33 & Cirrus Logic CL-G05420 & ISA-16 &  7\\
  \toprule 
  486SX & 33 & Tseng Labs ET3000                & ISA-8  &  7\\
  486SX & 33 & Cirrus Logic CL-G05420           & ISA-16 & 11\\ 
  486SX & 33 & Diamond Stealth (Tseng ET4000)   & VLB    & 15\\
  \toprule 
  486DX2 & 66 & Tseng Labs ET3000               & ISA-8  &  8\\
  486DX2 & 66 & Cirrus Logic CL-G05420          & ISA-16 & 13\\
  486DX2 & 66 & Diamond Stealth (Tseng ET4000)  & VLB    & 24\\
   \toprule
 \end{tabularx}
\caption{Benchmark with out-of-the-box Doom shareware.}
\label{bnechmarsks}
\end{figure}
\footnotetext{For reference, this configuration was able to run Wolfenstein 3D at 20 fps.}
\par
Using out of the box settings for the game, a top of the line machine could barely reach 25 fps\footnote{It would not be until Pentium with PCI buses came out that \doom~could be maxed out at 35 fps.}. Notice the importance of the bus, which yields a 3x performance hit/boost. At equal frequency a 486 provides twice the framerate of a 386.\\
\par






\subsection{Profiling}
Even without special tools, it is possible to gain insight on which part of the engine are responsible for CPU cycles consumptions thanks to built-in command line parameters.\\
\par
Parameter \cw{-nodraw} skips rendering all together (but does blit).\\
\par
\par
\fakedosoutput{timedemo_result_nodraw.txt}
\par
Without drawing, the game framerate improved from 17fps to 878fps.\\
\par
Parameter \cw{-noblit} renders to RAM but doesn't transfer the content to VRAM which is a good way to assess the impact of the bus speed. Because of optimization described later this parameter was only available on non-DOS versions like on a NextStation Turbo.\\
\par
\tcode{timedemo_result_noblit.txt}


\vspace{-10mm}
\subsection{Profiling With A Profiler}
An excellent class of performance visualizers are flamegraphs. They are built by first running a program and repeatedly interrupt it, unwinding the stack starting from the Program Counter to generate a backtrace. This is repeated hundreds of time over the course of the demo. When it finished , all backtraces are merged together.\\
\par
 This produces a tree where the width represents 100\% of the time and each level is a function call. It provides a  visual breakdown of were the machine spends time during a frame\footnote{This is a walltime-based flamegraph but there are many other kinds, like CPU-cycles based for example.}. On NeXTStep, thank to the multiprocess capability of the OS is easy to use \cw{gdb} from a second terminal and gather backtraces. The result looks as follow.\\
\par

\drawing{flamegraph}{Flamegraph of \cw{./doom -timedemo demo1} on NeXTStation Turbo Color}
\par
Obviously, \cw{D\_DoomLoop} accounts for 100\% of the time, with \cw{D\_Display} overwhelmingly dominating\footnote{But keep in mind NeXTStep port did not implement the audio system.}. Gameplay (\cw{TryRunTics}) (isolated column on the very right) is barely visible. Notice part of the graph with no children with the high cost of bliting from framebuffer \#0 to the screen (\cw{Update16}), the vertical drawing routine (\cw{R\_DrawColumn}) and the horizontal drawing routine (\cw{R\_DrawSpan}).


On DOS generating a flamegraph is slightly more difficult since it is a mono-threaded operating system. However with a little bit of instrumentation it is possible to get something similar. Instrumenting ten functions, the result is flatter but we can still gather some precious information as seen in the following graph.\\
\par
\vspace{4mm}
\drawing{dos_flamegraph}{}
\par
The result is not exactly the same as the NeXT. The sound routine in charge of updating the audio data (\cw{S\_UpdateSounds}) is barely visible in red. It makes sense since the only work to do is to retrieve sfx and music data from the \cw{.wad} and place it in the RAM for the audio card to use.\\
\par
 Something weirder is to notice that \cw{I\_FinishUpdate} seems instantaneous. How can the DOS version transport a full framebuffer accross the bus? The answer is off course: It doesn't! This is because the DOS version was optimized.\\
\par


\subsection{DOS Optimizations}
Given that it was meant to be the money maker, the DOS version received special care. During optimization sessions, John Carmack identified three area of improvement. One was in the math functions and two had to do with the 3D renderer.\\
 \par
 \fq{An exercise that I try to do every once in a while is to "step a frame" in the game, starting at some major point like common->Frame(), game->Frame(), or renderer->EndFrame(), and step into every function to try and walk the complete code coverage. This usually gets rather depressing long before you get to the end of the frame. Awareness of all the code that is actually executing is important, and it is too easy to have very large blocks of code that you just always skip over while debugging, even though they have performance and stability implications.}{John Carmack}
\par




\subsubsection{Math Optimizations}
Fixed point operations are performed everywhere. Function \cw{FixedMul} is found 124 times in the source code and is called a thousand times per frame. Along with \cw{FixedDiv2}, it was optimized with inline assembly\footnote{There is next to no assembly in \doom. Around 1994, John Carmack even declared that "The days of assembly are counted". That was without counting on Intel's Pentium which super-scalar architecture would require extra care by Michael Abrash for Quake.}. The C version "\cw{return ((long long) a * (long long) b) >> 16}" was a function call on Watcom resulting in close to 30 instructions but the assembly version only uses two.\\
\par
\acode{FixedMul.asm}\\


\vspace{-2mm}
\subsubsection{Direct Framebuffer Access Optimization}
A more substantial optimization had to do with the layering. On DOS the rules separating the core and the video system were bent. Renderers like the status bar and the auto-map still function as originally designed but the 3D drawing such as \cw{R\_DrawColumn} and \cw{R\_DrawSpan} are given direct access to the VGA banks. By bypassing framebuffer \#0, one read and one write per pixel (plus bus transfer) are avoided. Since the menu renderer has to be able to draw on top of everything, it is also granted direct VGA VRAM access.\\



\subsubsection{Assembly Renderer Optimization}
Not only \cw{R\_DrawColumn} and \cw{R\_DrawSpan} were given direct access to the VGA bank, they were hand optimized with gorgeous assembly using all the tricks in the book. Taking a look at \cw{R\_DrawColumn} from \cw{planar.asm} is eloquant. The function uses self-modifying code (see reserved \cw{12345678h} meant to contain scaling factor which is patched). We can also see the loop was unrolled to process two pixels at at time (and therefore avoid emptying the i486 prefetch queue because of the \cw{jnz} instruction). Notice the cool trick where register \cw{eax} is reused three times, once as a pointer to the texture source, then as a \cw{al} texel storage, then as \cw{al} translated texel (lightmapped) storage again. Overall this method yields an amortized 7 instructions per pixel for drawing column which is remarkable.\\
\par
All together these three optimization improved performance by a substantial 15\%.






\acode{R_DrawColumn.asm}

\vspace{-1.25cm}
\section{Performance tuning}
Despite all the care and optimizations, most gamers could not get more than 10 fps with the out-of-the-box game. To help reaching a decent framerate, two tuning mechanisms helped to reduce the number of pixels written.
\begin{enumerate}
\item High detail/Low detail toggle.
\item Adjust size of 3D canvas.
\end{enumerate}
\par
\trivia{These tradeoff were not peculiar to PC. All console ports from the "weak" Super Nintendo to the "strong" PlayStation used a combination of these two settings.}\\
\par



\section{High/Low details}
The first tuning option was to lower horizontal resolution using column doubling. In low resolution mode the engine only renders one out of very two columns but it writes it to the framebuffer twice. This results in a tremendous performance improvement since less pixel value are generated but they also don't have to transit on the bus. Given the 3D renderer domination of the frame rate, performance gains is almost a x2 multiplier as the following benchmark proves.\\
\par
 \vspace{0.5cm}
\begin{figure}[H]
\centering  
\begin{tabularx}{\textwidth}{ C{1} C{1} C{1} C{1} } 
  \toprule
  \textbf{High details resolution} & \textbf{High details fps} &  \textbf{Low details resolution}  & \textbf{Low details fps}\\
  \toprule 
320x200  &  19   & 160x200  &       29 \\
320x168  &  20   & 160x168  &       30 \\
288x144  &  23   & 144x144  &       32 \\
256x128  &  25   & 128x128  &       35 \\
224x112  &  28   & 112x112  &       38 \\
192x096  &  31   & 096x096  &       41 \\
160x080  &  35   & 080x080  &       45 \\
128x064  &  40   & 064x064  &       49 \\
096x048  &  45   & 048x048  &       54 \\
  \toprule
\end{tabularx}
\caption{Platform code specific.\protect\footnotemark}
\end{figure}
\footnotetext{Benchmark machine: Unisys CWD 4001 (486DX2-66 Cirrus Logic VLB).}
\par
Using low details instead of high details yields a linear 2x performance improvement regardless of the resolution selected.



\fullimage{high_res.png}\\
\par
Above the high resolution is 320x168. Below, in low resolution, it is dropped to 160x168 with uneven columns duplicated. The resolution drop is particularly noticeable on the non-magnified sergeant and the door but only because diminished lightning and CRT scaling had been disabled for demonstration purposes. In practice the difference was subtle.\\
\par
With direct access to the VGA banks, low is completely free optimization without need to write twice the same pixel since the vga mask is setup to write the same pixel in two banks.\\
\par
\fullimage{low_res.png}\\






\section{3D Canvas size adjustment}
An other option to improve the framerate was to lower the size of the 3D canvas. The player had access to a sliding bar allowing to select between eight dimensions. The slider was a multiplier affecting variable \cw{numblocks} to produce a value within range [0x3,0xB]. Value 11 was special and hardcoded to be recognized as "full-screen" 320x200.\\
\par
\vspace{4mm}
\ccode{R_ExecuteSetViewSize.c}
\par
It is unknown if anybody had the misfortune and courage to play the game in mode 3. This would have been an achievement in itself.\\
\par
\drawing{lod}{The nine 3D canvas configurations.}
\par
\trivia{id shipped \doom~ in High details with a canvas size \cw{0x9} default configuration.}

\begin{figure}[H]
\centering  
\begin{tabularx}{\textwidth}{ R{1}  R{1} R{1} R{1}  R{1.4}  R{0.6} }
  \toprule
  \textbf{numblocks} & \textbf{Width} & \textbf{Height} & \textbf{\# Pixels} & \textbf{\# Pixels \%} & \textbf{fps}\\
  \toprule 

 0xB & 320 & 200 & 64,000 & 100 & 19 \\
 0xA & 320 & 168 & 53,760 &  84 & 20 \\
 0x9 & 288 & 144 & 41,472 &  64 & 23 \\
 0x8 & 256 & 128 & 32,768 &  52 & 25 \\
0x7 & 224 & 112 & 25,088 &  39 & 28 \\
 0x6 & 192 & 96 & 18,432 &  28 & 31  \\
 0x5 & 160 & 80 & 12,800 &  20 & 35  \\
 0x4 & 128 & 64 &  8,192 &  12 & 40  \\
 0x3 &  96 & 48 &  4,608 &   7 & 45  \\
   \toprule
\end{tabularx}
\caption{Benchmark of performance gain vs 3D canvas size.}
\end{figure}


\par
Gains\footnote{Benchmarked on a miniPC Unisys CWD 4001 (486DX2-66/CirrusLogic-GD5424, 8MiB RAM).} are not as substantial as the details level. Reducing the 3D canvas size by 50\% yields only a 31\% improvement yet the visible area is so small it its almost not worth it.\\

\par
\fullimage{shipped_config}

