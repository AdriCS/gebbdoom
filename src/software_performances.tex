\section{Performances}
There is a convenient and portable way to assess performances of the engine. Thanks to its fixed tics architecture, a demo can be replayed with frame skipping disabled. This replay mode (called timedemo) will produce the exact same game session and a series of exact same 3D views on any machine.\\
\par
The gold standard of benchmarks was created by Anton Ertl around 1994. Over the last twenty five years, he has gathered metrics for hundreds of configuration\footnote{Source: https://www.complang.tuwien.ac.at/misc/doombench.html}.  It is simple to run and only requires Doom shareware v1.9.\\
\par
\fakedosoutput{timedemo.txt}
\par
These parameters run the prerecorded game session \#3 (probably recorded by John Romero). Because no frames are skipped, the session takes a variable time to finish. Upon completion two values are displayed.\\
\par
\fakedosoutput{timedemo_result.txt}
\par

\cw{gameticks} is the number of frames encountered. For \cw{demo3} it is always equal to 2134 since it comes from the record lump. The second value, \cw{realticks}, represents the walltime in ticks. It represent how fast the machine managed to replay the record.\\
\par
 The average frame per second is obtained with the following formula: $$ fps = walltime/realticks * 35 $$ 
In the example above\footnote{On a miniPC Unisys CWD 4001 (486DX2-66/CirrusLogic-GD5424).}, the game ran at an average of $1065/2134*35 = 17$ fps.\\
\par
Foone Turing was kind enough to volunteer his fleet of machine to assess performances upon release. Varying CPU and graphic cards gave fluctuating results.\\
\par

\begin{figure}[H]
\centering  
\begin{tabularx}{\textwidth}{ L{1}  R{0.3} R{2.4} R{1} R{0.3} }
  \toprule
   \textbf{CPU} & \textbf{Frequency} & \textbf{Graphic card} & \textbf{Bus} & \textbf{fps}\\
  \toprule 
  386DX & 33 & Tseng Labs ET3000    & ISA-8  &  4\\
  386DX\protect\footnotemark & 33 & Cirrus Logic CL-G05420 & ISA-16 &  7\\
  \toprule 
  486SX & 33 & Tseng Labs ET3000                & ISA-8  &  7\\
  486SX & 33 & Cirrus Logic CL-G05420           & ISA-16 & 11\\ 
  486SX & 33 & Diamond Stealth (Tseng ET4000)   & VLB    & 15\\
  \toprule 
  486DX2 & 66 & Tseng Labs ET3000               & ISA-8  &  8\\
  486DX2 & 66 & Cirrus Logic CL-G05420          & ISA-16 & 13\\
  486DX2 & 66 & Diamond Stealth (Tseng ET4000)  & VLB    & 24\\
   \toprule
 \end{tabularx}
\caption{Benchmark with out-of-the-box Doom shareware}
\end{figure}
\footnotetext{For reference, this configuration was able to run Wolfenstein 3D at 20 fps.}
\par
Without tweaking the configuration a top of the line machine could barely reach 25 fps\footnote{It would not be until Pentium with PCI buses came out that \doom could be maxed out at 35 fps}. Notice the importance of the bus, which yields a 3x performance hit/boost. At equal frequency a 486 provides twice the framerate of a 386.\\
\par






\subsection{Profiling}
Even without any special tools, it is possible to gain insight on which part of the engine are responsible for resources consumptions thanks to built-in command line parameters.\\
\par
Parameter \cw{-nodraw} skips rendering all together (but does blit).\\
\par
\par
\fakedosoutput{timedemo_result_nodraw.txt}
\par
Without drawing, the game ran at close to 202fps.\\
\par
Parameter \cw{-noblit} renders to RAM but doesn't transfer the content to VRAM which is a good way to assess the impact of the bus speed. Because of optimization described later this parameter was only available on non-DOS versions.\\
\par
\tcode{timedemo_result_noblit.txt}
\par
An excellent performance visualizer are flamegraphs. They are built by repeatedly interrupting a process, unwinding the stack starting from the Program Counter to generate a backtrace. The process is repeated hundreds of time over the course of the demo. When the demo is done, all backtrace are merged together.\\
\par
 This produces a tree where the width represents 100\% of the time and each level is a function call. It provides a  visual representation of were was the CPU during each part of a frame. On NeXTStep, thank to the multiprocess capability of the OS is easy to use \cw{gdb} from a second terminal and gather backtraces. The result looks as follow.\\
\par

\drawing{flamegraph}{\doom demo1 on NeXT, flamegraph stats}
\par
Without too much surprise, \cw{D\_DoomLoop} acccounts for 100\% of the time, with \cw{D\_Display} overwhelmingly dominating\footnote{But keep in mind NeXTStep port did not implement the audio system.}. Gameplay (\cw{TryRunTics}) (on the very right) is barely visible.\\
\par
On DOS generating a flamegraph is slightly more difficult since it is a mono-threaded operating system. However with a little bit of instrumentation it is possible to get something similar.\\
\par
\drawing{dos_flamegraph}{}
\par
We see that sound doesn't consume much since the only work to do is to retrieve sfx and music data from the wad and place them in the RAM for the audio card to use. Something weird though is that \cw{I\_FinishUpdate} seems instantaneous which leads us to DOS optimizations.\\
\par


\subsection{Optimizations}
For obvious reasons, the DOS version received special care. Two area of bottlnecks were identified in the math function and in the renderer.\\
\par
Since Fixed point operation were performed a lot, \cw{FixedMul} and  \cw{FixedDiv2}. \cw{FixedMul} optimized with inline assembly.\\
\par



Drawing walls and flats had been corretly identified as the \#1 bottleneck. The DOS version had hand crafted XXX routines which improved performances by 15\%. These were the only assembly in doom. The compiler and \cw{libc} were becoming so good that there was no more need to. Only the drawing routines for the walls and flats were optimized wich resulted in a 15\% performance books. John Carmack even declared that "The days of assembly are counted". That was without counting on Intel's Pentium which super-scalar architecture would require extra care by Michael Abrash for Quake.\\
\par
 \fixme{Graph ASM vs NO ASM}.
 \par
 STUDY ASM vs C version. Seems to write 8 pixels at at time, reusing offset, saving store instructions. Trying to stress registers. On PC the rules separating the kernel and the video system are bent. The kernel's drawing routines such as \cw{R\_DrawColumn} and \cw{R\_DrawSpanFAB} are given privileged access to the VGA framebuffer and write in it directely. The status bar and the auto-map keep are not that lucky.\\
 \par
 \cw{FixedMul} is called 124 times in the source code. Good candidate for optimization. Benchmark disabling it.
 \par

\subsection{Look up tables}
Bla
\section{Reduce window size}
Bla. Usage of dirtybox here.
\par
\drawing{lod}{The ten 3D canvas configurations.}
\par
11 numblocks, [3,11]\\
\par
\ccode{R_ExecuteSetViewSize.c}
\par


\begin{figure}[H]
\centering  
\begin{tabularx}{\textwidth}{ C{1}  R{1} R{1} R{1}  R{1.4}  R{0.6} }
  \toprule
  \textbf{numblocks} & \textbf{Width} & \textbf{Height} & \textbf{\# Pixels} & \textbf{Percentage} & \textbf{fps}\\
  \toprule 

11 & 320 & 200 & 64,000 & 100 & 19 \\
10 & 320 & 168 & 53,760 &  84 & 20 \\
 9 & 288 & 144 & 41,472 &  64 & 23 \\
 8 & 256 & 128 & 32,768 &  52 & 25 \\
 \protect\footnotemark7 & 224 & 112 & 25,088 &  39 & 28 \\
 6 & 192 & 96 & 18,432 &  28 & 31  \\
 5 & 160 & 80 & 12,800 &  20 & 35  \\
 4 & 128 & 64 &  8,192 &  12 & 40  \\
 3 &  96 & 48 &  4,608 &   7 & 45  \\
   \toprule
\end{tabularx}
\caption{Platform code specific.}
\end{figure}

\par

\footnotetext{limit of playability}

\section{High/Low details}

\begin{figure}[H]
\centering  
\begin{tabularx}{0.7\textwidth}{ C{1}  C{1} C{1}}
  \toprule
  \textbf{numblocks} & \textbf{High details fps} &  \textbf{Low details fps}\\
  \toprule 

            
11 &  19     &        30 \\
10  &  XX      &      XX \\
9  &  XX      &       XX \\
8  &  XX      &       XX \\
7  &  XX      &       XX \\
6  &  XX      &       XX \\
5  &  XX      &       XX \\
4  &  XX      &       XX \\
3  &  45      &       54 \\
  \toprule
\end{tabularx}
\caption{Platform code specific.}
\end{figure}

\fullimage{high_res.png}\\
\fullimage{low_res.png}\\


\subsection{bla}
\trivia{
Doom build time\\ 
Full build = 9:37 (among it: 19s to link)\\
Partial build (r\_sky.c) = 27s (among it 19s to link).\\}
\par

Abstraction layer (compare VGA in wolf3d to doom) incured small overhead but also unlocked new algorithm and allowed easy port -> even a fridge runs doom.\\
\par
NextStep dithering.\\
\\ 


\fq{An exercise that I try to do every once in a while is to "step a frame" in the game, starting at some major point like common->Frame(), game->Frame(), or renderer->EndFrame(), and step into every function to try and walk the complete code coverage. This usually gets rather depressing long before you get to the end of the frame. Awareness of all the code that is actually executing is important, and it is too easy to have very large blocks of code that you just always skip over while debugging, even though they have performance and stability implications.}{John Carmack}
\par
\trivia{Last minute bug with lock only John Carmack could have figured it out.}




\trivia{The source code released \cw{linuxdoom-1.10} was considably altered. XX created many more headers files which make the architecture clearer and also increased compilation speed.}
\par


Doom ASM improved performances by 16\%. Detail here the method (timedemo) and realtick. FPS 14 to 16.\\
\par


\par

The kernel is made of 45 translation units and is common to all version of \doom.\\
\par

\trivia{What are the dot during startup?}


