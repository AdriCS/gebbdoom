\section{Source Code}
The source code of \doom was released roughtly four years after the commercial release of the game, on December 23, 1997. More than twenty years later, the zip archive is still where it was originally placed, on id Software ftp server.\\
\par
\tcode{doom_src_zip_url.c}
\par
 In the long series of id software source code release\footnote{From 1993 to 20XX, id Software released the code for all nine games it produced}, Doom source code stands apart since what was released was not what was used to ship the game. The source code is targeted at neither MS-DOS nor NextStep. There is a little bit of a back story about it.\\
 \par
 In early 1997, Bernd Kreimeier approached id Software with a business offer. He wanted to write a book explaining the internals of the game engine, how to compile, and how to modify it. The idea was to release the book along with the source code.\\
 \par
  People at id, especially John Carmack and John Romero thought it was a great idea. They promptly sent him the source code. However, upon receiving the source code, Bernd had to make a few important decisions. Dave Taylor had been hard at work and \doom was able to run on many platforms.  The engine worked on no less than five operating systems. Linux, NextStep, IRIX SGI, and of course MS-DOS were supported. To make the code easier to understand, Bernd decided to pick one platform and delete everything unrelated.\\
  \par
  Because the dominant operating system was MS-DOS, it would have made sense to pick that target. However, due to copyright issues, the DMX source library needed by the MS-DOS build could not be included. MS-DOS was a no-go.\\
   An other option would have been to release the NextStep version which had seen the most usage. However since NeXT had stopped manufacturing workstation in 1994 and sold less than 50,000 units over its lifetime this was also a dead end. Also the sound and music system had never been implemented. NeXTStep was also a no go. The third available solution was the Linux build.\\
\par
 As he was cleaning up the code from everything not related to Linux and writing the book at the same time, hardware and software kept on evolving. Things evolved faster than he could write and before he could finish, interest in Doom was decreasing in favor of Quake and Duke Nukem 3D.\\
 \par
  With the profitability of the project compromised the book was abandoned\footnote{What surfaced, "A Brief Summary of DOOM style Rendering by Robert Forsman July and Bernd Kreimeier" was of great quality.}. With the blessing of id Software, Kreimeier decided to release the code he had cleaned up. This port became the base of all hundreds of ports ever generated since.\footnote{The original MS-DOS code can largely be reconstructed thanks to Raven which was much less conservative about DMX overlaps. Large portions of previously censured system of the game such as \cw{i\_sound.c} and \cw{i\_ibm.c} can be found in Heretic, and Hexen source code.}.\\
 \par  


