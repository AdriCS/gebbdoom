\section{Source Code}
\doom developers have been kind to entousiast programmers. Shortly after the release of the game in December 1993, id Software started to open the source of the tools necessary to build the assets. The nodebuilder (\cw{doombsp}) in May 1994, the dmtools in 1998, and DoomED in 2015. The most interesting part, the game engine and its secretes was released on December 23, 1997\footnote{Only four years after the game}. More than twenty years later, the zip archive is still where it was originally placed, on id Softare ftp server.\\
\par
\tcode{doom_src_zip_url.c}
\par
 In the long series of id software source code release\footnote{From 1993 to 20XX, id Software released the code for all nine games it produced}, Doom source code stands apart since what was released was not what was used to ship the game. The source code is targeted at neither MS-DOS nor NextStep. There is a little bit of a backstory about how this code ended up being released.\\
 \par
 In early 1997, Bernd Kreimeier (a programmer and professional author) approached id software in order to write a book explaining the Doom game engine internals and how to extend/modify the game. The idea was to release the book with the source code. After obtaining the source code, he had to make a few decisions. Dave Taylor had ported the code to many platforms and as a result \doom could be compiled for DOS, NextStep, IRIX SGI, SVGA, .... He decided to clean things up to make it easier for the reader to understand the codebase. It would have been ideal to delete everything but what was needed for DOS. Due to copyright issues, the DMX source library could not be included. An other option would have been to release the NextStep version. However since NeXT had stopped manufacturing workstation in 1994 and sold less than 50,000 units over its lifetime this version would have been useless. His solution to provide something compiling out of the box and reaching sa many readers as possible was to remove the portions of the code targetting MS-DOS, Irix, and NeXT components and port Doom to Linux.\\
\par
 As he was writing both the code required by Linux and the explanations for the book, hardware and software kept on evolving. Things evolved faster than he could write and before he could finish Doom had became obsolete. With the profitability of the project compromised the book was abandonner\footnote{What a pity, what remains, a PDF titled "XXX" is of great quality.}. With the blessing of id Software, Kreimeier released his Linux port. This port became the base of all hundreds of ports ever generated since.\footnote{The original MS-DOS code can largely be reconstructed thanks to Raven which was much less conservative about DMX overlaps. Large portions of previously censured system of the game such as \cw{i\_sounc.c} were found in Heretic, and Hexen source code.}.\\
 \par  


