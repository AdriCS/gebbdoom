\section{Source Code}
The source code of \doom{} was released on December 23, 1997, roughly four years after the commercial release of the game. Originally hosted on id Software's \cw{FTP} server, the code was transitioned to \cw{github.com} where it can still be found to this day.\\
\par
\tcode{doom_src_zip_url.c}
\par
 In the long series of id software source code releases\footnote{From 1993 to 2012, id Software released the code for all games it produced.}, \doom{} stands apart since what was released was not what was used to ship the game. There is a little bit of a back story.\\
 \par
 In early 1997, Bernd Kreimeier approached id Software with a business proposition. He wanted to write a book explaining the internals of the game engine, how to compile, and how to modify it. The idea was to release the book along with the source code.\\
 \par
  People at id, especially John Carmack thought it was the "Right Thing" to do\footnote{"The Right Thing" is a concept mentioned in the book "Hackers: Heroes of the Computer Revolution" by Steven Levy. It was often quoted in John Carmack's finger plans.}. They promptly sent him the source code. Upon reviewing it Bernd realized he had to make a few important decisions. Between the development requirements and Dave Taylor's ports there was code specific to many platforms. The engine could be compiled on no less than five operating systems. Linux, NeXTSTEP, SGI IRIX, and of course MS-DOS were supported. To make the code easier to understand, Bernd decided to pick one platform and delete everything unrelated.\\
  \par
  The ideal choice would have been the MS-DOS version. It was the dominant operating system and it was the version players had experienced the game with. However, there was a copyright issue. id Software had licensed an audio library, DMX, the code to which was proprietary and could not be included with the source. MS-DOS was a no-go.\\
  \par
   Another option would have been to release the NeXSTEP version which had seen the most usage and therefore was the second most stable. However, since \NeXT had stopped manufacturing workstations in 1994 and sold fewer than 50,000 units over its lifetime this was also a dead end. Few people would have had the software to enjoy it. There was a second problem with NeXTSTEP -- the sound and music systems had never been implemented for this platform. NeXTSTEP was also a no-go.\\
   \par
   
    The third available option was the Linux build, and that is what Bernd picked. As he was stripping the code of everything not related to Linux and writing the book, hardware and software kept on evolving. Ultimately, the world of gaming changed faster than he could write and before he was finished, interest in \doom{} had decreased in favor of newer engines such as Quake and Duke Nukem 3D.\\
 \par
  With the profitability of the venture compromised, the publisher withdrew and the book project was abandoned\footnote{What survived, "A Brief Summary of DOOM style Rendering by Robert Forsman and Bernd Kreimeier" was of high quality.}. With the blessing of id Software, Kreimeier released the Linux code he had cleaned up. This port has become the base for hundreds of forks since.\footnote{The original MS-DOS code can largely be reconstructed thanks to Raven who were much less conservative about DMX-related code. Large portions of previously censured portions of the source, such as \cw{i\_sound.c} and \cw{i\_ibm.c} can be found in the Heretic and Hexen source code.}.\\
 \par  


