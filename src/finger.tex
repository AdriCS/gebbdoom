\section{Pubic Relations}
In an era before social media and before democratized Internet, studio had few ways to get directely in touch with their customers. Most of the time, newspapers and magazine relied information which was inrequent, slow, innacurate and most of time left the reader with more questions than answers.\\
\par
id Software quickly noticed the Unix system that came with their NeXT hardware features a tool called \cw{finger}. Entirely text based, \cw{finger} allowed id Software each employee to update a \cw{.plan} text file located in the home directory of their workstations. Anybody from the outside could connect to id server and consult who was there.\\
\par
\tcode{finger_idsoftware}
\par
\trivia{When it was invented, the term "finger" had, in the 1970s, a connotation of "is a snitch": this made "finger" a good reminder/mnemonic to the semantic of the UNIX finger command (a client in the protocol context).}\\
\par

Using finger on an email address brought up the content of the user's \cw{.plan} file. This was the equivalent of what is known today as a blog and allowed direct one way communication with users.\\
\par
Unfortunately most of the early \cw{.plan} have now been lost to history. id Software fans started saving them around 1996. The oldest available on Internet is one of John Carmack's date Feb 18, 1996. The date indicates he was working on Quake at the time.\\
\par
\tcode{finger_johnc.txt}


